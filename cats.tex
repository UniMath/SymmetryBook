\chapter{A categorical interlude}
\label{ch:cats}

We have seen that many types carry a notion of morphism between its elements:
\begin{itemize}
\item We have functions $f : A \to B$ between types $A$ and $B$ in a universe $\UU$
  (\cref{univalent-mathematics}),
\item We have identifications $p : x \eqto y$ between elements $x,y$ of any type $A$
  (\cref{sec:identity-types}),
\item We have pointed functions $f : A \ptdto B$ between pointed types $A$ and $B$ in $\UUp$
  (\cref{def:pointedtypes}),
\item We have fiberwise maps $f : \prod_{x:X}(A(x) \to B(x))$ between families $A,B : X \to \UU$ (\cref{def:fiberwise}),
\item We have homomorphisms $f : \Hom(G, H)$ between groups $G, H : \Group$
  (\cref{def:grouphomomorphism}),
\wip{\item We have maps of cycles $f : (X,t) \to (Y,u)$ and of bicycles $(X,a,b) \to (X',a',b')$,}
\item We have maps of $G$-sets $f : \Hom_G(X,Y)$ for $X,Y : \BG \to \Set$
  (\cref{def:map-of-Gsets}; a special case of fiberwise maps).
\end{itemize}
In all those cases, we have notions of identity morphism and composition of morphisms.
We have also seen that some maps between types are paired with maps on morphisms, for example,
taking underlying symmetries in groups, $\USym : \Group \to \Set$ (\cref{def:group-symmetries}),
comes with a corresponding operation of taking the underlying map of symmetries of a group homomorphism,
\[ \USym : \Hom(G,H) \to (\USymG \to \USymH) \]
(\cref{def:USym-hom}) satisfying $\USym(\id_G) = \id_{\USymG}$ and $\USym(\psi\circ\varphi) = \USym\psi \circ \USym\varphi$ (\cref{cor:USym-compose}).

It's very useful to develop some abstractions for types equipped with a notion of morphism and maps equipped with maps of morphisms like this.
These give the notions of (wild) \emph{categories} and \emph{functors}, respectively,
and \emph{category theory} is the study of these structures.

Here we give a brief primer\footnote{%
  The topic is of course too vast to cover in detail here,
  so we refer to the literature for more details.
  Category theory in univalent foundations is also treated in
  Ch.~10 of the HoTT book\footnotemark{},
  while \citeauthor{AwodeyCat}\footnotemark{}
  and \citeauthor{RiehlContext}\footnotemark{}
  give traditional expositions,
  and \citeauthor{MacLaneWorking}\footnotemark{}
  gives a comprehensive treatment.}%
\footcitetext{hottbook}%
\footcitetext{RiehlContext}%
\footcitetext{AwodeyCat}
on category in order to systematize what we've done so far,
and prepare for the main result of the next chapter, which is to give an
\emph{equivalence of categories} between the categories of concrete and abstract groups.

\section{Brief overview of the chapter}

In~\cref{sec:categories} we define the kinds of categories we need, along with many examples (including the above).
Then we discuss various abstract notions in categories (terminal and initial objects, products and coproducts)
and remark on the importance of \emph{duality} in~\cref{sec:duality}.
In~\cref{sec:naturality} we cover natural transformations and adjunctions; we have already seen some examples of adjunctions, for example in~\cref{xca:adjunction-^*-_*}.

We have also seen an incarnation of the Yoneda lemma in~\cref{xca:TTYoneda}; in~\cref{sec:yoneda} we treat the O.G.\ version. We end with a brief introduction to monoidal categories in~\cref{sec:monoidal-cats}, as we'll see in~\cref{ch:abelian} that abelian groups form an example.

\section{Categories and functors}
\label{sec:categories}

As mentioned above, many types come equipped with notion
of \emph{morphism} or \emph{arrow} between its elements
which is more general than identification or \emph{isomorphism}.
For instance
we have type of functions $A \to B$ for $A,B:\UU$
and the type of pointed functions $A \ptdto B$ for $A,B:\UUp$.
There are identity morphisms and composition of morphisms,
and this motivates the following definition:

\begin{definition}\label{def:wild-cat}
  A \emph{wild precategory}\index{category!wild precategory}
  consists of the following data:
  \begin{enumerate}
  \item\label{struc:cat-ob} A type $\var{Ob}$, called the \emph{type of objects}.
  \item\label{struc:cat-mor} For each pair of objects $A,B : \var{Ob}$,
    a type of \emph{morphisms} $\hom(A,B)$. These are also known as \emph{arrows},
    and written $A \to B$ when there's no danger of confusion.
  \item\label{struc:cat-id} For each object $A : \var{Ob}$,
    an \emph{identity arrow} $\id_A : A \to A$.
  \item\label{struc:cat-comp} For each pair of arrows $f : A \to B$ and $g : B \to C$,
    a \emph{composite arrow} $g\circ f : A \to C$.
  \item\label{struc:cat-unit-laws} For each arrow $f : A \to B$,
    a pair of identifications
    \[
      \lambda : \id_B \circ f \eqto f, \quad
      \rho : f \circ \id_A \eqto f.
    \]
  \item\label{struct:cat-assoc} For each triple of arrows
    $f : A \to B$, $g : B \to C$, and $h : C \to D$, an identification
    \[
      \alpha : (h \circ g) \circ f \eqto h \circ (g \circ f).
    \]
  \end{enumerate}
  If $\mathcal C \jdeq (\var{Ob},\hom,\id,\lambda,\rho,\alpha)$
  is a wild precategory, then we write $A, B : \mathcal C$ instead of $A, B : \var{Ob}$
  to indicate that $A, B$ are elements of the underlying type of objects of $\mathcal C$.
  We may write $f, g : A \to_{\mathcal C} B$ to emphasize where the arrows $f$ and $g$ live, if needed,
  and sometimes $\hom_{\mathcal C}(A,B)$ or $\mathcal C(A,B)$, instead of $\hom(A,B)$.
\end{definition}
\begin{remark}
  With this definition, we readily equip the universes of types $\UU$ and of pointed types $\UUp$
  with the structure of wild precategories.
  In the former case, we can use reflexivities for $\lambda$, $\rho$, and $\alpha$.
  In the latter, we leave their definition as an exercise.

  We use the adjective ``wild'' to highlight a deficiency of this definition as it stands:
  We haven't specified any further laws for the identifications $\lambda$, $\rho$, and $\alpha$.
  For example, it would be sensible to require an identification of $\lambda\eqto\rho : \id_A \circ \id_A \eqto \id_A$, as well as a filler for the pentagonal diagram of $\alpha$'s coming from four composable arrows:
  \[
    \begin{tikzpicture}[commutative diagrams/every diagram]
      \node (P0) at (90:2.3cm) {$((k \circ h) \circ g) \circ f$};
      \node (P1) at (90+72:2cm) {$(k \circ (h \circ g)) \circ f$} ;
      \node (P2) at (90+2*72:2cm) {\makebox[5ex][r]{$k \circ ((h \circ g) \circ f)$}};
      \node (P3) at (90+3*72:2cm) {\makebox[5ex][l]{$k \circ (h \circ (g \circ f))$}};
      \node (P4) at (90+4*72:2cm) {$(k \circ h) \circ (g \circ f)$};
      \path[commutative diagrams/.cd, every arrow, every label]
      (P0) edge node[swap] {${\ap{[\blank\circ f]}(\alpha)}$} (P1)
      (P1) edge node[swap] {$\alpha$} (P2)
      (P2) edge node {$\ap{[k\circ\blank]}(\alpha)$} (P3)
      (P4) edge node {$\alpha$} (P3)
      (P0) edge node {$\alpha$} (P4);
    \end{tikzpicture}
  \]
  Of course, we would on occasion then need a filler for a three dimensional diagram of pentagons
  for five composable arrows, etc., etc., \emph{ad infinitum}.
  The corresponding structure is known to connoisseurs as an $(\infty,1)$-precategory.
  It is an open problem as we write this whether this notion can be defined in our type theory.
  We would certainly hope that types and pointed types would furnish examples.
\end{remark}
% definitions: wild cat, cat, univalent cat, margin:\infty-cat
%monoids

\section{Abstract notions and duality}
\label{sec:duality}

% terminal/initial
% duality
% preorders and monoids
% monos and epis (epi+mono in Set=iso, not in Mon)

\section{Naturality and adjunctions}
\label{sec:naturality}

% functors and natural transformations
% adjunctions
% products and coproducts, pullbacks and pushouts, (co)equalizers, (co)limits

\section{The Yoneda Lemma}
\label{sec:yoneda}

% Yoneda lemma
% exponentials

\section{Monoidal categories}
\label{sec:monoidal-cats}

% monoidal category = Mon(Cat)
% closed category (no need for HITs)


%%% Local Variables:
%%% mode: LaTeX
%%% latex-block-names: ("lemma" "theorem" "remark" "definition" "corollary" "fact" "properties" "conjecture" "proof" "question" "proposition" "exercise")
%%% TeX-master: "book"
%%% TeX-command-extra-options: "-fmt=macros"
%%% compile-command: "make book.pdf"
%%% End:
