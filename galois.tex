\chapter{Galois theory}%
\label{chap:galois-theory}%

%% VERY PRELIMINARY
The goal of Galois theory is to study how the roots of a given polynomial can
be distinguished from one another. Take for example $X^2+1$ as a polynomial
with real coefficients. It has two distinct roots in $\mathbb C$, namely $i$
and $-i$. However, an observer, who is limited to the realm of $\mathbb R$,
can not distinguish between the two. Morally speaking, from the point of view
of this observer, the two roots $i$ and $-i$ are pretty much the same. Formally
speaking, for any polynomial $Q: \mathbb R[X,Y]$, the equation $Q(i,-i) = 0$ is
satisfied if and only if $Q(-i,i) = 0$ also. This property is easily understood
by noticing that there is a automorphism of fields $\sigma: \mathbb C  \to
\mathbb C$ such that $\sigma(i) = -i$ and $\sigma(-i) = i$ which also fixes
$\mathbb R$. The goal of this chapter is to provide the rigorous framework in
which this statement holds.
{\color{red} TODO: complete/rewrite the introduction}

\section{Covering spaces and field extensions}
\label{sec:cover-spac-fields}

\def\fieldstype{\mathbf{Fields}}%
\def\Gal{\mathrm{Gal}}%
\def\fieldshom{\hom_{\fieldstype}}%
\def\isHom{\mathrm{isHom}}%
\def\iso{\mathrm{Iso}}%
\newcommand\restr[1]{{{#1}^\ast}}%
\newcommand\fieldsext[1]{#1\backslash\fieldstype}%
Recall that a field extension is simply a morphism of fields $i: k\to K$ from a
field $k$ to a field $K$. Given a fixed field $k$, the type of fields
extensions of $k$ is defined as
\begin{displaymath}
  \fieldsext k \defequi \sum_{K:\fieldstype}\fieldshom(k,K)
\end{displaymath}

\begin{definition}
  The Galois group of an extension $(K,i)$ of a field $K$, denoted $\Gal(K,i)$
  or $\Gal(K/k)$ when $i$ is clear from context, is the group
  $\Aut_{\fieldsext k}{(K,i)}$.
  \label{def:galois-group}
\end{definition}

\begin{remark}
  \label{rem:sip-univalence}
  The Structure Identity Principle holds for fields, which means that for
  $K,L:\fieldstype$, one has
  \begin{displaymath}
    (K = L) \weq \iso(K,L)
  \end{displaymath}
  where $\iso(K,L)$ denotes the type of these equivalences that are
  homomorphisms of fields. Indeed, if one uses $K$ and $L$ also for the carrier
  types of the fields, one gets:
  \begin{displaymath}
    \begin{split}
      (K = L) \weq \sum_{p:K=_\UU L}  (\trp p (+_K) = +_L)
      \times (\trp p (\cdot_K) = \cdot_L)
      \\ \times (\trp p (0_K) = 0_L)
      \times (\trp p (1_K) = 1_L) 
    \end{split}
  \end{displaymath}
  Any $p : K =_\UU L$ is the image under univalence of an equivalence $\phi: K \weq L$, and then:
  \begin{align*}
    \trp p (+_K) &= (x,y) \mapsto \phi( \inv\phi(x) +_K \inv \phi(y)) \\
    \trp p (\cdot_K) &= (x,y) \mapsto \phi( \inv\phi(x) \cdot_K \inv \phi(y)) \\
    \trp p (0_K) &=\phi( 0_K ) \\
    \trp p (1_K) &=\phi( 1_K )
  \end{align*}
  It follows that:
  \begin{displaymath}
    \begin{split}
      (K=L) \weq \sum_{\phi: K \weq L} (\phi(x +_K y) = \phi (x) +_L \phi (y)) \\
      \times (\phi(x\cdot_K y) = \phi (x) \cdot_L \phi (y)) \\
      \times (\phi(0_K) = 0_L)
        \times (\phi(1_K) = 1_L)
    \end{split}
  \end{displaymath}
  The type on the right hand side is the same as $\iso(K,L)$ by definition.

  In particular, given an extension $(K,i)$ of $K$:
  \begin{align*}
    \US \Gal(K,i) \weq \sum_{p:K=K} \trp p i = i \weq \sum_{\sigma:\iso(K,K)} \sigma \circ i = i
  \end{align*}
  This is how the Galois group of the extension $(K,i)$ is defined in ordinary mathematics.
\end{remark}

Given an extension $(K,i)$ of field $k$, there is a map of interest:
\begin{displaymath}
  \restr i: \fieldsext K \to \fieldsext k,\quad (L,j) \mapsto (L,ji)
\end{displaymath}

\begin{lemma}
  The map $\restr i$ is a set-bundle.
  \label{lem:field-ext-restriction-set-bundle}
\end{lemma}
\begin{proof}
  Given a field extension $(K',i')$ in $\fieldsext k$, one wants to prove that
  the fiber over $(K',i')$ is a set. Suppose $(L,j)$ and $(L',j')$ are
  extensions of $K$, together with paths $p:(K',i') = (L,ji)$ and $p': (K',i')
  = (L',j'i)$. Recall that $p$ and $p'$ are respectively given by equivalences
  $\pi: K' = L$ and $\pi': K' = L'$ such that $\pi i' = ji$ and $\pi' i' =
  j'i$. 
  %
  A path from $( (L,j), p)$ to $( (L',j'), p')$ in the fiber over $(K',i')$ is
  given a path $q: (L,j) = (L',j')$ in $\fieldsext K$ such that $\trp q p =
  p'$. However, such a path $q$ is the data of an equivalence $\varphi : L =
  L'$ such that $\varphi j = j'$, and then the condition $\trp q p = p'$
  translates as $\varphi \pi = \pi'$. So it shows that $\varphi$ is necessarily
  equal to $\pi'\inv\pi$, hence is unique.  
\end{proof}

The fiber of this map at a given extension $(L,j)$ of $k$ is:
\begin{align*}
  \inv{(\restr i)}(L,j) &\weq \sum_{L':\fieldstype}\sum_{j':K \to L'}(L,j) = (L',j'i) \\
  &\weq \sum_{L':\fieldstype}\sum_{j':K\to L'}\sum_{p:L = L'} pj=j'i \\
  &\weq \sum_{j':K\to L} j=j'i \\
  &\weq \hom_k(K,L)
\end{align*}
where the last type denotes the type of homomorphisms of $k$-algebra (the structure of $K$ and $L$ being given by $i$ and $j$ respectively).

In particular, the map $t: \USym \Gal(K,i) \to \inv{(\restr i)}(K,i)$ mapping $g$ to
$\trp g (\id_K)$ identifies with the inclusion of the $k$-automorphisms of $K$
into the $k$-endomorphisms of $K$.

{\color{red} TODO: write a section on polynomials in chapter 12}
%
\begin{definition}
  Given an extension $i:k\to K$, an element $\alpha:K$ is algebraic if $\alpha$ is merely
  a root of a polynomial with coefficients in $k$. That is if the following
  proposition holds:
  \begin{displaymath}
    \Trunc{\sum_{n:\mathbb N}\sum_{a:\bn n + 1 \to k}i(a(0))+i(a(1))\alpha+\cdots+i(a(n))\alpha^n=0}%
  \end{displaymath}
  \label{defn:algebraic-element}
\end{definition}

\begin{definition}
  A field extension $(K,i)$ is said to be algebraic when each $a:K$ is algebraic.
  \label{defn:algebraic-extension}
\end{definition}

\begin{remark}
  Note that when the extension $(K,i)$ is algebraic, then $t$ is an
  equivalence. However, the converse is false, as shown by the non-algebraic
  extension $\mathbb Q \hookrightarrow \mathbb R$. We will prove that every
  $\mathbb Q$-endomorphism of $\mathbb R$ is the identity function. Indeed, any
  $\mathbb Q$-endomorphism $\varphi : \mathbb R \to \mathbb R$ is linear and
  sends squares to squares, hence is non-decreasing. Let us now take an irrational
  number $\alpha:\mathbb R$. For any rational $p,q:\mathbb Q$ such that $p <
  \alpha < q$, then $p = \varphi(p) < \varphi(\alpha) < \varphi(q) = q$. Hence
  $\varphi(\alpha)$ is in any rational interval that $\alpha$ is. One deduces
  $\varphi(\alpha) = \alpha$.
  %
  \label{rem:algebraic-endomorphisms-are-automorphisms}
\end{remark}

\begin{definition}
  A field extension $i:k\to K$ is said finite when $K$ as a
  $k$-vector space, the structure of which is given by $i$, is of finite dimension.
  In that case, the dimension is called the degree of $i$, denoted $[(K,i)]$ or $[K:k]$ when $i$ is clear from context.
  \label{defn:degree-field-extension}
\end{definition} 

\section{Intermediate extensions and subgroups}
%
Given two extensions $i: k \to K$ and $j: K \to L$, the map $\restr i$ can be seen as a pointed map
\begin{displaymath}
  \restr i: \B\Gal(L,j) \to \B\Gal(L,ji),\quad x\mapsto x\circ i.
\end{displaymath}
Then, through \cref{lem:field-ext-restriction-set-bundle}, $\restr i$ presents
$\Gal(L,j)$ as a subgroup of $\Gal(L,ji)$. One goal of Galois theory is to
characterize those extensions $i':k \to L$ for which all subgroups of
$\Gal(L,i')$ arise in this way. 

Given any extension $i:k \to L$, there is an obvious $\Gal(L,i)$-set $X$ given by 
\begin{displaymath}
  (L',i') \mapsto L'.
\end{displaymath}
For a pointed connected set-bundle $g:B \to \B \Gal(L,i)$, one can consider the
type of fixed points of the $\mkgroup B$-set $Xf$:
\begin{displaymath}
  K \defequi (Xg)^{\mkgroup B} \jdeq \prod_{x:B}X(g(x))
\end{displaymath}
It is a set, which can be equipped with a field structure, defined pointwise.
Moreover, if one denotes $b$ for the distinguished point of $B$, and $(L'',j'')$ for $g(b)$, then, because $g$ is pointed, one has a path $p:L=L''$ such that $pi'=j''$. There are
fields extensions $i':k \to K$ and $j':K \to L$ given by:
\begin{displaymath}
  i'(a) \defequi x\mapsto \snd(g(x))(a),\quad
  j'(f) \defequi \inv p f(b)
\end{displaymath}
In particular, for all $a:k$, $j'i'(a) = \inv p \snd(g(b))(a) = \inv p j''(a) = i'(a)$.

Galois theory is interested in the settings when these two constructions are inverse from each other.

\section{separable/normal/etc.}
\label{sec:cover-spac-fields-1}

\section{fundamental theorem}
\label{sec:fundamental-theorem}



