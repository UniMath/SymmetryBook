%% \section{scalar products}
%% \section{euclidean frames, relation to determinants(?)}
%% \section{the euclidean group as a semidirect product}
%% \section{euclidean properties (length, angle, etc.)}


In this chapter we study Euclidean geometry.  We assume some standard linear
algebra over real numbers, including the notion of finite dimensional vector
space over the real numbers and the notion of inner product.

\section{Euclidean spaces}

\begin{definition}\label{def:EuclideanSpace}
  A {\em Euclidean space} $E$ is a torsor $\Points E$ for the additive abelian group underlying
  a real vector space $\Vectors E$ of finite dimension $\dim E$ equipped with an inner product
  $(x,y) \mapsto \langle x,y \rangle$.
\end{definition}

We denote the type of all Euclidean spaces by $\ES$.

For each natural number $n$, we may construct the {\em standard} Euclidean
space $\EE^n$ of dimension $n$ as follows.  For $\Vectors E$ we take the vector space $\RR^n$,
$$ \Vectors E \defeq \RR^n, $$
equipped with the standard inner product given by the dot product
$$ \langle x , y \rangle \defeq x \cdot y, $$
where the dot product is defined as usual as
$$ x \cdot y \defeq \sum_i x_i y_i . $$
Then for $\Points E$ we take the corresponding principal torsor:
$$ E \defeq \princ V. $$

\begin{theorem}\label{thm:GramSchmidt}
  Any Euclidean space $E$ is merely equal to $\EE^n$, where $n$ is $\dim E$.
\end{theorem}

\begin{proof}
  For $n$ we (must) take $n$ to be $\dim E$.  Since we are proving a
  proposition and any torsor is merely trivial, we may assume that $\Points E =
  \princ {\Vectors E}$.  Since any finite dimensional vector space merely has a
  basis, we may assume we have chosen a basis for $\Vectors E$.  Finally, by
  Gram-Schmidt orthonormalization we may assume that $\Vectors E$ is $\RR^n$
  equipped with the standard inner product.
\end{proof}

\begin{lemma}\label{lem:EuclideanSpace1Type}
  The type $\ES$ is a $1$-type.
\end{lemma}

\begin{proof}
  Given two Euclidean spaces $E$ and $E'$, we must show that $E=E'$ is a set.
  We may assume $\dim E = \dim E'$, since otherwise, $E=E'$ is empty.  Since
  being a set is a proposition, we may assume $E$ and $E'$ are both standard,
  of dimension $n$, say.  The equalities $\Vectors E = \Vectors E'$ are
  correspond to orthogonal $n \times n$ real matrices, which form a set, so we
  may assume that $E'$ is $E$.  The equalities $\princ{\Vectors E} =
  \princ{\Vectors E}$ between trivial torsors correspond to translations by
  elements of the corresponding group $\Vectors E$, and they form a set, so we
  are done.
\end{proof}

In light of that lemma, we may introduce the following family of groups.

\begin{definition}\label{def:EuclideanGroup}
  Given a natural number $n$, we define the {\em Euclidean group} $\EucGp n$ to be
  the connected component of $\ES$ containing the point $\EE^n$.
\end{definition}
