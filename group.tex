%\usepackage{\xspace}
\newcommand{\ie}{i.e.,\xspace}
%\newcommand{\refl}{\mathrm{refl}}
The identity type is not just any type.  In the previous sections we have seen that the identity type $a=_Aa$ reflects all the ``symmetries'' of a term $a$ in a type $A$.  Symmetries have special properties; for instance if you rotate a square by $90^o$, or by $-90^o$, undoing the first rotation.  Symmetries can also be composed.  
We have also seen that the order with which you compose symmetries can matter (see Example~]ref{}).

With inspiration with geometric and algebraic origins, it became clear to mathematicians at the end of the 19'th century that the properties of such symmetries could be codified by saying that they form a {\em group}.  Conversely, once one has given an abstract definition of a group it emerges that they all are identity types (see Lemma~\ref{}).  This chapter is about groups, and their basic properties; everything from the point of view of identity types which arguably is closer to the geometric origins of group theory than the abstract approach which is usual in traditional textbooks.  

See the end of the section \ref{} for a brief summary of the early history of groups.  
\begin{remark}
  The reader may wonder about the status of the identity type $a=_Aa'$ where $a,a':A$.  One problem is of course that there is no obvious ways of composing and another is that $a=_Aa'$ does not have a distinguished element such as $\mathrm{refl{}_a}:a=_Aa$.
If $f:a=_Aa'$ we can use transport along $f$ to compare $a=_Aa'$ with $a=_Aa$ (much as affine planes can be compared with the standard plane or a finite dimensional real vector space is isomorphic to some Euclidean space), but absent existence and choice of such an $f$ there is not much one generally can say about the behavior of $a=_Aa'$.
\end{remark}

\subsection{The identity type as a group}

((list the axioms informally))

Fix a type $A$ and let $a,b,c:A$ be terms in $A$.

The first property is already present at the very foundations: there is a preferred term
$$\refl{}_a:a=_Aa,$$
which we think of the identity symmetry of $a$, not doing anything.

The existence of the inverse is presented as follows.  
\begin{lemma}
  For every $p:a=_Ab$ there is a $p^{-1}:b=_Aa$ such that $\refl{}_a{}^{-1}\equiv\refl{}_a$.  More precisely, there is a term 
$$\mathrm{inv}:\prod_{a,b:A}((a=_Ab)\to(b=_Aa))$$
such that $\mathrm{inv}(a,a,\refl{}_a)=\refl{}_a$. In this formulation, $p^{-1}$ is shorthand for $\mathrm{inv}(a,b,p)$.
\end{lemma}

\begin{remark}
  Notice that the last statement  (``More precisely\dots'')  not only asserts that there {\em exist} inverses, but that there actually is a (preferred and consistent) way to produce them.  

Classically this was in many instances unnecessay to say because there was a unique inverse, and the distinction is not mentioned in introductory texts.  However, then this very point had to be revisited later on.  In our proof relevant setting it is obvious that the ultimate statement will have to go beyond an assertion that inverses exist.
\end{remark}