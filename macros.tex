%% document class
\documentclass[10pt,oneside,leqno,openright]{memoir}
%% packages
\usepackage{microtype}
\usepackage[T2A,T1]{fontenc}
\usepackage[utf8]{inputenc}
\usepackage[osf,helvratio=.9]{newpxtext}     % largsc
\renewcommand{\rmdefault}{pplx} % use FPL text for better sc
\usepackage[vvarbb]{newpxmath}
\DeclareSymbolFont{largesymbols}  {OMX}{zplm}{m}{n}
\makeatletter
\re@DeclareMathSymbol{\sumop}{\mathop}{largesymbols}{"50}
\re@DeclareMathSymbol{\prodop}{\mathop}{largesymbols}{"51}
\re@DeclareMathSymbol{\intop}{\mathop}{largesymbols}{"52}
\makeatother
\renewcommand*{\coprod}{\amalg}
\usepackage[cal=cm,scr=zapfc,scrscaled=1.2,frak=euler]{mathalfa}
\let\pxopenbox\openbox % pxmath already has openbox but amsthm also defines it
\let\openbox\relax
\usepackage{csquotes}
\usepackage[russian,ngerman,main=english,shorthands=off]{babel}
\usepackage{amsthm}
\renewcommand*{\openbox}{\pxopenbox}
\usepackage{mathtools}          %to get \vcentcolon
\usepackage{thmtools}
\usepackage{xspace}
\usepackage{booktabs}
\usepackage{ragged2e}           % to hyphenate marginals
\usepackage{marginfix}          % to get \mparshift
\usepackage{scalerel}           % to scale \div
\usepackage[all]{xy}
\usepackage{pgfplots,tikz,tikz-cd,tikz-3dplot}
\pgfplotsset{compat=1.11}
\usepackage{enumitem}
\usepackage{xifthen}

\input tikzsetup

%% mathtools does not define an extensible equal sign, so we do so here:
\makeatletter
\def\Equalfill@{\arrowfill@\Relbar\Relbar\Relbar}
\MHInternalSyntaxOn
\providecommand*\xEqual[2][]{%
  \ext@arrow 0055{\Equalfill@}{#1}{#2}}
\MHInternalSyntaxOff
\makeatother
\mathtoolsset{mathic}

% bibliography
\usepackage{hyphenat}
\usepackage[backref=true,
            style=verbose,
            autocite=footnote,
            autolang=hyphen,
            useprefix=false,
            backend=biber,
            bibencoding=utf8]{biblatex}
\addbibresource{papers.bib}
\newcommand*{\citeerror}{\PackageError{cas}{Don't use \protect\cite\space: %
    use \protect\footcite\space instead,\MessageBreak %
    perhaps together with \protect\citeauthor}{}}
\let\cite\citeerror
\DefineBibliographyStrings{english}{%
  backrefpage = {page},% originally "cited on page"
  backrefpages = {pages},% originally "cited on pages"
}
\newcommand*{\arxiv}[1]{Preprint available at arXiv: \href{http://arxiv.org/abs/#1}{\nolinkurl{#1}}}
\newbibmacro*{related:samevol}[1]{%
  \renewcommand*{\newunitpunct}{\space}%
  \entrydata{#1}{\usebibmacro{doi+eprint+url}}}

% hyperref should be the package loaded last
\usepackage[colorlinks,
            bookmarksnumbered=true,
            bookmarksdepth=2,
            bookmarksopenlevel=2,
            citecolor=linkcolor,
            linkcolor=linkcolor,
            urlcolor=linkcolor,
            pdfborderstyle=,
            unicode]{hyperref}
% To get unicode in pdfinfo, we need to use hypersetup, not package options
\hypersetup{
  pdfauthor={Marc Bezem, Ulrik Buchholtz, Pierre Cagne, Bjørn Ian Dundas, Daniel R. Grayson},
  pdftitle={Symmetry},
  pdfsubject={Mathematics},
  pdfkeywords={type theory, group theory, univalence axiom}}
% - except for cleveref!
\usepackage[capitalize,noabbrev]{cleveref}

%% Colors: link color, and a red/blue pair that remains contrasting
%% for colorblind individuals (https://davidmathlogic.com/colorblind)
\definecolor{linkcolor}{rgb}{0,0,0.5}
%\definecolor{casred}{rgb}{0.863,0.196,0.125}
%\definecolor{casblue}{rgb}{0.0,0.353,0.71}
\definecolor{casred}{rgb}{0.831,0.067,0.349}
\definecolor{casblue}{rgb}{0.102,0.522,1.0}
\definecolor{casgreen}{rgb}{0.608,0.761,0.039}


%% Enable synctex
\pdfadjustspacing=1
\brokenpenalty=10000 %%% No hyphenation across page breaks
\synctex=1

%% redefine \em to use \slshape
\makeatletter
\DeclareRobustCommand\em
  {\@nomath\em \ifdim \fontdimen\@ne\font >\z@
    \eminnershape \else \slshape \fi}%
\makeatother

%% To mark a chapter or section as optional
\newcommand*{\titledagger}{ (\texorpdfstring{$\dagger$}{†})}
%% macros
\newcommand{\DELETE}[1]{} % no star on purpose: argument may contain \par

%%% Lists (uses enumitem)
\setlist{itemsep=0.5ex}
\setlist[1]{labelindent=\parindent}
\setlist[1]{leftmargin=*}
\setlist[enumerate,1]{label=(\arabic*),ref=(\arabic*)}
\setlist[description]{font=\normalfont\scshape}

%%%%%%%%%%%%%%%%%%%%%%%%%%%%%%%%%%%%%%%%%%%%%%%%%%%%%%%%%%%%%%%%%%%%%%%%%%%%%
%%% THEOREMS
\newcommand*\qefsymbol{\ensuremath{\lrcorner}}
\declaretheoremstyle[headfont=\normalfont\scshape,bodyfont=\slshape]{cas-thm}
\declaretheoremstyle[headfont=\normalfont\scshape,qed=\qefsymbol]{cas-def}
\declaretheorem[sibling=subsection,style=cas-thm]{theorem}
\declaretheorem[sibling=theorem,style=cas-thm]{lemma}
\declaretheorem[sibling=theorem,style=cas-thm]{corollary}
\declaretheorem[sibling=theorem,style=cas-thm]{conjecture}
\declaretheorem[sibling=theorem,style=cas-thm]{axiom}
\declaretheorem[sibling=theorem,style=cas-thm]{construction}
\declaretheorem[sibling=theorem,style=cas-def]{definition}
\declaretheorem[sibling=theorem,style=cas-def]{remark}
\declaretheorem[sibling=theorem,style=cas-def]{example}
\declaretheorem[sibling=theorem,style=cas-def]{exercise}
\declaretheorem[sibling=theorem,style=cas-def,name=Exercise]{xca}
\declaretheorem[sibling=theorem,style=cas-def,name=Principle]{principle}
\def\implementation#1{\proof[Implementation of \cref{#1}]}
\def\endimplementation{\endproof}
\numberwithin{equation}{section}
\Crefname{xca}{Exercise}{Exercises}
\Crefname{construction}{Construction}{Constructions}

%% end
%%%%%%%%%%%%%%%%%%%%%%%%%%%%%%%%%%%%%%%%%%%%%%%%%%%%%%%%%%%%%%%%%%%%%%%%%%%%%

%%% Palatino line width
\setlength{\normalrulethickness}{0.59pt}

%%% The occasional Russian
\newcommand{\casrus}[1]{\foreignlanguage{russian}{%
    \fontfamily{Tempora-TLF}\selectfont #1}}

%%% Tufte font size
%%
% Set the font sizes and baselines to match Tufte's books
\makeatletter
\renewcommand\normalsize{%
   \@setfontsize\normalsize\@xpt{14}%
   \abovedisplayskip 10\p@ \@plus2\p@ \@minus5\p@
   \abovedisplayshortskip \z@ \@plus3\p@
   \belowdisplayshortskip 6\p@ \@plus3\p@ \@minus3\p@
   \belowdisplayskip \abovedisplayskip
   \let\@listi\@listI}
\normalbaselineskip=14pt
\normalsize
\renewcommand\small{%
   \@setfontsize\small\@ixpt{12}%
   \abovedisplayskip 8.5\p@ \@plus3\p@ \@minus4\p@
   \abovedisplayshortskip \z@ \@plus2\p@
   \belowdisplayshortskip 4\p@ \@plus2\p@ \@minus2\p@
   \def\@listi{\leftmargin\leftmargini
               \topsep 4\p@ \@plus2\p@ \@minus2\p@
               \parsep 2\p@ \@plus\p@ \@minus\p@
               \itemsep \parsep}%
   \belowdisplayskip \abovedisplayskip
}
\renewcommand\footnotesize{%
   \@setfontsize\footnotesize\@viiipt{10}%
   \abovedisplayskip 6\p@ \@plus2\p@ \@minus4\p@
   \abovedisplayshortskip \z@ \@plus\p@
   \belowdisplayshortskip 3\p@ \@plus\p@ \@minus2\p@
   \def\@listi{\leftmargin\leftmargini
               \topsep 3\p@ \@plus\p@ \@minus\p@
               \parsep 2\p@ \@plus\p@ \@minus\p@
               \itemsep \parsep}%
   \belowdisplayskip \abovedisplayskip
}
\renewcommand\scriptsize{\@setfontsize\scriptsize\@viipt\@viiipt}
\renewcommand\tiny{\@setfontsize\tiny\@vpt\@vipt}
\renewcommand\large{\@setfontsize\large\@xipt{15}}
\renewcommand\Large{\@setfontsize\Large\@xiipt{16}}
\renewcommand\LARGE{\@setfontsize\LARGE\@xivpt{18}}
\renewcommand\huge{\@setfontsize\huge\@xxpt{30}}
\renewcommand\Huge{\@setfontsize\Huge{24}{36}}
\makeatother

%%% Layout
\stockaiv
\pageaiv
\settypeblocksize{*}{26pc}{*}
\setlrmargins{1in}{*}{*}
\setulmarginsandblock{1in}{1in}{*}
\setheadfoot{\baselineskip}{2\baselineskip}
\setheaderspaces{*}{2\baselineskip}{*}
\setmarginnotes{2pc}{11pc}{\baselineskip}
\checkandfixthelayout
\raggedbottom
\setlength{\parindent}{1em}

%%% Tufte margins
\marginparmargin{right}
\footnotesinmargin
\renewcommand*{\thefootnote}{\arabic{footnote}}
\setlength{\footmarkwidth}{-1sp}
\setlength{\footmarksep}{0pt}
\renewcommand{\foottextfont}{\footnotesize\RaggedRight}
\newlength{\extrawidth}
\setlength{\extrawidth}{\marginparsep}
\addtolength{\extrawidth}{\marginparwidth}
\newenvironment{fullwidth}{\begin{adjustwidth}{0mm}{-\extrawidth}%
  \blockmargin}%
  {\unblockmargin\end{adjustwidth}}
\setmpjustification{\RaggedLeft}{\RaggedRight}%
\newcommand*{\marginnote}[2][]{%
  \ifthenelse{\isempty{#1}}{\relax}{\mparshift{#1}}%
  \marginpar{\mpjustification\normalfont\footnotesize #2}}
\setmarginfloatcaptionadjustment{figure}{\captionnamefont{\footnotesize\scshape}%
  \captiontitlefont{\footnotesize}\captionstyle{\mpjustification}}
\setmarginfloatcaptionadjustment{table}{\captionnamefont{\footnotesize\scshape}%
  \captiontitlefont{\footnotesize}\captionstyle{\mpjustification}}
\sidecapmargin{right}
\setsidecappos{t}
\renewcommand*{\sidecapstyle}{\captionnamefont{\footnotesize\scshape}%
  \captiontitlefont{\footnotesize}\captionstyle{\mpjustification}}

%%% Space after period
\DeclareMathSymbol{.}{\mathpunct}{letters}{"3A}
\DeclareMathSymbol{\decimalperiod}{\mathord}{letters}{"3A}

%%% Title page
\newcommand{\thetitlepage}{{%
    \thispagestyle{empty}
    \noindent{\huge \textls[250]{SYMMETRY}}

    \vspace*{10ex}

    \noindent{\itshape\Large\foreignlanguage{ngerman}{Am Anfang war die Symmetrie}
      --
      In the beginning was symmetry!}\\

    \vspace{2ex}

    \noindent\hfill\parbox{.62\textwidth}{Werner Heisenberg,
      \emph{\foreignlanguage{ngerman}{Der Teil
          und das Ganze: Gespr\"ache im Umkreis der Atomphysik}}, 1969,
      English translation, \emph{Physics and Beyond}, 1971.}

    \vfill

    \begin{center}
      \tdplotsetmaincoords{45}{135}
      \begin{tikzpicture}[tdplot_main_coords,scale=2.0,opacity=0.5]
        \draw (-1.00000, -1.61803, 0.00000) -- (-0.00000, -1.00000, -1.61803);
        \draw (-1.61803, 0.00000, -1.00000) -- (-0.00000, -1.00000, -1.61803);
        \draw (-1.00000, -1.61803, 0.00000) -- (-1.61803, -0.00000, -1.00000);
        \draw (1.00000, -1.61803, 0.00000) -- (-0.00000, -1.00000, -1.61803);
        \draw (1.00000, -1.61803, 0.00000) -- (-1.00000, -1.61803, 0.00000);
        \draw (-1.61803, 0.00000, 1.00000) -- (-1.00000, -1.61803, -0.00000);
        \draw (-1.61803, 0.00000, -1.00000) -- (0.00000, 1.00000, -1.61803);
        \draw (-1.61803, 0.00000, 1.00000) -- (-1.61803, 0.00000, -1.00000);
        \draw (0.00000, 1.00000, -1.61803) -- (0.00000, -1.00000, -1.61803);
        \fill[casgreen] (0.00000, 0.00000, 0.00000) -- (0.00000, -1.61803, 0.00000) -- (-1.00000, -1.61803, 0.00000) -- (-1.00000, 0.00000, 0.00000) -- cycle;
        \fill[casblue] (0.00000, 0.00000, 0.00000) -- (0.00000, 0.00000, -1.61803) -- (0.00000, -1.00000, -1.61803) -- (0.00000, -1.00000, 0.00000) -- cycle;
        \fill[casred] (0.00000, 0.00000, 0.00000) -- (-1.61803, 0.00000, 0.00000) -- (-1.61803, 0.00000, -1.00000) -- (0.00000, 0.00000, -1.00000) -- cycle;
        \draw (0.00000, -1.00000, 1.61803) -- (-1.00000, -1.61803, 0.00000);
        \draw (1.61803, 0.00000, -1.00000) -- (0.00000, -1.00000, -1.61803);
        \draw (-1.00000, 1.61803, 0.00000) -- (-1.61803, 0.00000, -1.00000);
        \fill[casred] (0.00000, 0.00000, 0.00000) -- (-1.61803, 0.00000, 0.00000) -- (-1.61803, 0.00000, 1.00000) -- (0.00000, 0.00000, 1.00000) -- cycle;
        \fill[casblue] (0.00000, 0.00000, 0.00000) -- (0.00000, 0.00000, -1.61803) -- (0.00000, 1.00000, -1.61803) -- (0.00000, 1.00000, 0.00000) -- cycle;
        \fill[casgreen] (0.00000, 0.00000, 0.00000) -- (0.00000, -1.61803, 0.00000) -- (1.00000, -1.61803, 0.00000) -- (1.00000, 0.00000, 0.00000) -- cycle;
        \draw (0.00000, -1.00000, 1.61803) -- (1.00000, -1.61803, 0.00000);
        \draw (0.00000, -1.00000, 1.61803) -- (-1.61803, 0.00000, 1.00000);
        \draw (1.61803, 0.00000, -1.00000) -- (1.00000, -1.61803, 0.00000);
        \draw (-1.61803, 0.00000, 1.00000) -- (-1.00000, 1.61803, 0.00000);
        \draw (0.00000, 1.00000, -1.61803) -- (1.61803, 0.00000, -1.00000);
        \draw (-1.00000, 1.61803, 0.00000) -- (0.00000, 1.00000, -1.61803);
        \fill[casred] (0.00000, 0.00000, 0.00000) -- (1.61803, 0.00000, 0.00000) -- (1.61803, 0.00000, -1.00000) -- (0.00000, 0.00000, -1.00000) -- cycle;
        \fill[casgreen] (0.00000, 0.00000, 0.00000) -- (0.00000, 1.61803, 0.00000) -- (-1.00000, 1.61803, 0.00000) -- (-1.00000, 0.00000, 0.00000) -- cycle;
        \fill[casblue] (0.00000, 0.00000, 0.00000) -- (0.00000, 0.00000, 1.61803) -- (0.00000, -1.00000, 1.61803) -- (0.00000, -1.00000, 0.00000) -- cycle;
        \draw (0.00000, 1.00000, 1.61803) -- (-1.61803, 0.00000, 1.00000);
        \draw (1.61803, 0.00000, 1.00000) -- (1.00000, -1.61803, 0.00000);
        \draw (1.00000, 1.61803, 0.00000) -- (0.00000, 1.00000, -1.61803);
        \fill[casgreen] (0.00000, 0.00000, 0.00000) -- (0.00000, 1.61803, 0.00000) -- (1.00000, 1.61803, 0.00000) -- (1.00000, 0.00000, 0.00000) -- cycle;
        \fill[casblue] (0.00000, 0.00000, 0.00000) -- (0.00000, 0.00000, 1.61803) -- (0.00000, 1.00000, 1.61803) -- (0.00000, 1.00000, 0.00000) -- cycle;
        \fill[casred] (0.00000, 0.00000, 0.00000) -- (1.61803, 0.00000, 0.00000) -- (1.61803, 0.00000, 1.00000) -- (0.00000, 0.00000, 1.00000) -- cycle;
        \draw (0.00000, -1.00000, 1.61803) -- (1.61803, 0.00000, 1.00000);
        \draw (0.00000, 1.00000, 1.61803) -- (0.00000, -1.00000, 1.61803);
        \draw (1.61803, 0.00000, 1.00000) -- (1.61803, 0.00000, -1.00000);
        \draw (1.00000, 1.61803, 0.00000) -- (1.61803, 0.00000, -1.00000);
        \draw (0.00000, 1.00000, 1.61803) -- (-1.00000, 1.61803, 0.00000);
        \draw (-1.00000, 1.61803, 0.00000) -- (1.00000, 1.61803, 0.00000);
        \draw (0.00000, 1.00000, 1.61803) -- (1.61803, 0.00000, 1.00000);
        \draw (0.00000, 1.00000, 1.61803) -- (1.00000, 1.61803, 0.00000);
        \draw (1.61803, 0.00000, 1.00000) -- (1.00000, 1.61803, 0.00000);
      \end{tikzpicture}
    \end{center}

    \vfill

    by\\[5ex]
    Marc Bezem\\
    Ulrik Buchholtz\\
    Pierre Cagne\\
    Bjørn Ian Dundas\\
    Daniel R.~Grayson

    \vfill

    \indent\hfill{\Large Book version: \texttt{\OPTcommit} (\OPTdate)}

    \vfill\clearpage}}

\def\BibTeX{{B\kern-.05em{\scshape i\kern-.025em b}\kern-.08em
    T\kern-.1667em\lower.7ex\hbox{E}\kern-.125emX}}

%%% Copyright page
\newcommand{\thecopyrightpage}{{%
    \begin{fullwidth}
    \noindent Copyright \textcopyright{} 2025 by Marc Bezem, Ulrik Buchholtz,
    Pierre Cagne,\\
    Bjørn Ian Dundas, and Daniel R.~Grayson.
    All rights reserved.

    \vspace{10ex}

    \noindent
    \begin{minipage}[c]{0.20\linewidth}\centering
      \includegraphics[width=0.9\linewidth]{cc-by-sa.pdf}
    \end{minipage}
    \begin{minipage}{0.75\linewidth}
      This work is licensed under the
      Creative Commons Attribution-ShareAlike\\ 4.0 International License. To view a copy of this license, visit:\\
      \url{http://creativecommons.org/licenses/by-sa/4.0/}
    \end{minipage}

    \vspace{10ex}

    \noindent This book is available at:
    \url{https://unimath.github.io/SymmetryBook/book.pdf}

    \vspace{10ex}

    \noindent To cite the book, the following \BibTeX{} code may be useful:

    \vspace{5ex}

    \noindent{\ttfamily%
      \halign{{}\hspace{2em} ## \hfil & ## & ## \hfil \cr
        \noalign{@misc\{Symmetry,}
        title        &=& \{Symmetry\},\cr
        author       &=& \{Marc Bezem and Ulrik Buchholtz and Pierre Cagne\cr
                     & &  {}~and Bjørn Ian Dundas and Daniel R.~Grayson\},\cr
        date         &=& \{\OPTdate\},\cr
        howpublished &=& \{\textbackslash url%
                          \{https://github.com/UniMath/SymmetryBook\}\},\cr
        note         &=& \{Commit:~\textbackslash texttt\{\OPTcommit\}\}\cr
        \noalign{\}}}}

    \vfill\end{fullwidth}\clearpage}}

%%% Table of contents
\renewcommand*{\cftchapterfont}{\itshape\LARGE}
\renewcommand*{\cftchapterleader}{\kern1pc\textperiodcentered}
\renewcommand*{\cftchapterformatpnum}[1]{\kern1pc\normalfont\LARGE #1}
\renewcommand*{\cftchapterafterpnum}{\cftparfillskip}
\renewcommand*{\chapternumberline}[1]{\llap{#1\kern2pc}}
\renewcommand*{\cftsectionfont}{\normalfont}
\renewcommand*{\cftsectionleader}{\enskip\textperiodcentered}
\renewcommand*{\cftsectionformatpnum}[1]{\enskip\normalfont #1}
\renewcommand*{\cftsectionafterpnum}{\cftparfillskip}
\setrmarg{0pt plus 1fil}
\setlength{\cftsectionnumwidth}{3em}
\setlength{\cftsectionindent}{0pt}
\setpnumwidth{2pc}
\maxtocdepth{section}

%%% Headers
\setlength{\headwidth}{\textwidth}
\addtolength{\headwidth}{\marginparsep}
\addtolength{\headwidth}{\marginparwidth}
\makepagestyle{cas}
\makerunningwidth{cas}{\headwidth}
\makeheadposition{cas}{flushright}{flushleft}{}{}
\newcommand*{\headtitle}{\normalfont\scshape\textls{symmetry}}
\makeevenhead{cas}{\thepage\kern2pc\normalfont\scshape\textls{\leftmark}}{}{}
\makeoddhead{cas}{}{}{{\normalfont\scshape\textls{\rightmark}}\kern2pc\thepage}
\def\memUChead{\MakeTextLowercase}
\makeatletter
\makepsmarks{cas}{%
  \createmark{chapter}{both}{nonumber}{\@chapapp\ }{. \ }
  \createplainmark{toc}{both}{\contentsname}
  \createplainmark{lof}{both}{\listfigurename}
  \createplainmark{lot}{both}{\listtablename}
  \createplainmark{bib}{both}{\bibname}
  \createplainmark{index}{both}{\indexname}
  \createplainmark{glossary}{both}{\glossaryname}
}
\makeatother
\pagestyle{cas}

%%% Chapters
\makeatletter
\makechapterstyle{cas}{%
  \setlength{\beforechapskip}{2\onelineskip}%
  \setlength{\midchapskip}{3pt}%
  \setlength{\afterchapskip}{4\onelineskip \@plus .1\onelineskip
    \@minus 0.167\onelineskip}%
  \renewcommand*{\printchaptername}{}%
  \renewcommand*{\chapternamenum}{}%
  \renewcommand*{\chapnumfont}{\raggedright\normalfont\huge\itshape}%
  \renewcommand*{\chaptitlefont}{\raggedright\normalfont\huge\slshape}%
  \renewcommand*{\printchaptertitle}[1]{%
    \begin{adjustwidth}{}{-\extrawidth}
      \chaptitlefont ##1\par\nobreak
    \end{adjustwidth}}%
  \renewcommand*{\printchapternum}{\chapnumfont \thechapter}%
  \renewcommand*{\printchapternonum}{\raggedright}}
\makeatother
\chapterstyle{cas}

%%% Sectioning
\setsecnumdepth{subsection}
\maxsecnumdepth{subsection}
\setsecheadstyle{\normalfont\Large\slshape}
\setsubsecheadstyle{\normalfont\large\slshape}
\setsubsubsecheadstyle{\normalfont\slshape}

%%% Symbol index (Glossary)
\renewcommand*{\glossaryname}{Symbol index}
\renewcommand{\glossitem}[4]{#1\enskip\textperiodcentered\enskip #2, #3 #4\\}
%\twocolglossary

% for the glossary item
\newcommand*{\bang}{!}
\newcommand*{\mthroot}[2]{\!\sqrt[\uproot{2}{#1}]{#2}} % \! added by MB 30012024

% deprecated!!
\renewcommand*{\sc}{\scshape}

% save old div and equiv
\let\olddiv\div
\renewcommand*{\div}{{\mathchoice%
  {\olddiv}%
  {\olddiv}%
  {\vcenter{\hbox{\fontsize{.5em}{0}\selectfont$\olddiv$}}}%
  {\vcenter{\hbox{\fontsize{.45em}{0}\selectfont $\olddiv$}}}%
}}%

% replace exists with a bigop version
\let\oldexists\exists
\let\exists\relax
\DeclareMathOperator*\exists{%
  \mathop{\vphantom\sum\mathchoice%
    {\vcenter{\hbox{\fontsize{1.75em}{0}\selectfont$\oldexists$}}}%
    {\vcenter{\hbox{\fontsize{1.25em}{0}\selectfont$\oldexists$}}}%
    {\vcenter{\hbox{\fontsize{0.9em}{0}\selectfont$\oldexists$}}}%
    {\vcenter{\hbox{\fontsize{0.75em}{0}\selectfont$\oldexists$}}}%
  }}%
\DeclareMathOperator*\existsuniq{%
  \mathop{\vphantom\sum\mathchoice%
    {\vcenter{\hbox{\fontsize{1.75em}{0}\selectfont$\oldexists!$}}}%
    {\vcenter{\hbox{\fontsize{1.25em}{0}\selectfont$\oldexists!$}}}%
    {\vcenter{\hbox{\fontsize{0.9em}{0}\selectfont$\oldexists!$}}}%
    {\vcenter{\hbox{\fontsize{0.75em}{0}\selectfont$\oldexists!$}}}%
  }}%

% Meta-macros for:
\newcommand*{\constant}[1]{\mathrm{#1}} % defined constants / functions : roman
\newcommand*{\constructor}[1]{\mathrm{#1}} % constructors : roman
\newcommand*{\typeformer}[1]{\mathrm{#1}} % (uppercase) typeformers : roman
\newcommand*{\var}[1]{\mathit{#1}} % variable : italics
\newcommand*{\UU}{{\mathscr{U}}} % universes (special case) : calligraphic
\newcommand*{\UUp}{{\mathscr{U}_*}}

\newcommand*{\fakeslant}[1]{%
  \pdfliteral{1 0 0.167 1 0 0 cm}#1\pdfliteral{1 0 -0.167 1 0 0 cm}}
\newcommand*{\bn}[1]{{%
  \if!\ifnum9<1#1!\else_\fi%
  \mathbb{#1}\else\fakeslant{\mathbb{#1}}\fi}}

% our version of \operatorname with a font change (so we can use the appropriate meta-macro)
\makeatletter
\newcommand*{\casop}[1]{\mathop{\newmcodes@\kern\z@ #1}\nolimits@}
\makeatother

% Typeformers
\newcommand*{\bool}{\typeformer{Bool}}
\newcommand*{\charstring}{\typeformer{String}}
\newcommand*{\integer}{\typeformer{Int}}
\newcommand*{\real}{\typeformer{Real}}
\newcommand*{\true}{\typeformer{True}}
\newcommand*{\false}{\typeformer{False}}
\newcommand*{\Prop}{\typeformer{Prop}}
\newcommand*{\Set}{\typeformer{Set}}
\newcommand*{\Groupoid}{\typeformer{Groupoid}}
\newcommand*{\FinSet}{\typeformer{FinSet}}
\newcommand*{\Cyc}{\typeformer{Cyc}}% cycles
\newcommand*{\InfCyc}{\typeformer{InfCyc}}% infinite ditto
\newcommand*{\FinCyc}{\typeformer{FinCyc}}% finite ditto
\newcommand*{\CycGe}[1]{\typeformer{Cyc}_{#1|}}% cycles greater than
\newcommand*{\CycLe}[1]{\typeformer{Cyc}_{|#1}}% cycles less than
\newcommand*{\Bicyc}{\typeformer{Bicyc}}% bicycles
\newcommand*{\GSet}[1][G]{\mathord{#1\textrm{-}\Set}}
\newcommand*{\absGSet}[1][{\agp G}]{\mathord{#1\textrm{-}\Set^{\abstr}}}
\newcommand*{\Group}{\typeformer{Group}}
\newcommand*{\AbGroup}{\typeformer{AbGroup}}
\newcommand*{\Bunch}{\typeformer{Bunch}}
\newcommand*{\AbBunch}{\typeformer{AbBunch}}
\newcommand*{\Band}{\typeformer{Band}}
\newcommand*{\AbBand}{\typeformer{AbBand}}
\newcommand*{\Monoid}{\typeformer{Monoid}}
\newcommand*{\Tors}{\typeformer{Torsor}}
\newcommand*{\Copy}{\casop{\typeformer{Copy}}}
\newcommand*{\Fin}{\casop{\typeformer{Fin}}}
\newcommand*{\Heap}{\typeformer{Heap}}

\newcommand*{\zet}{\typeformer{Z}} % the SET of integers
\newcommand*{\QQ}{\mathbb{Q}}
\newcommand*{\ZZ}{\mathbb{Z}}
\newcommand*{\NN}{\mathbb{N}}
\newcommand*{\NNN}{\mathbb{N}^{-}} % negated natural numbers
\newcommand*{\CC}{\mathbb{C}}
\newcommand*{\RR}{\mathbb{R}}
\newcommand*{\ii}{\constant{i}} % unit complex imaginary
\newcommand*{\ee}{\constant{e}} % base of natural logarithm
\newcommand*{\emptytype}{\emptyset}
\newcommand*{\UUscone}{\UU_\ast^{=1}} % pointed connected groupoids
\newcommand*{\UUsctwo}{\UU_\ast^{=2}} % pointed 1-connected 2-types
\newcommand*{\UUpconn}{\UU_\ast^{>0}} % pointed connected types
\newcommand*{\UUppone}{\UU_{{\ast}{\ast}}^{=1}} % bipointed conn gpds

% Constructors
\newcommand*{\yes}{\constructor{yes}}
\newcommand*{\no}{\constructor{no}}
\newcommand*{\triv}{\constructor{triv}}
\newcommand*{\refl}[1]{\constructor{refl}_{#1}}
\newcommand*{\rrfl}{\constructor{rrfl}}
\newcommand*{\inl}[1]{\casop{\constructor{inl}}_{#1}}
\newcommand*{\inr}[1]{\casop{\constructor{inr}}_{#1}}
\newcommand*{\inc}[1]{\casop{\constructor{in}}_{#1}}
\newcommand*{\zeq}{\constructor{zeq}}

% Functions and defined elements
\newcommand*{\refloi}[1]{\casop{\constant{refl}^{-\mathrm{o}}_{#1}}} % exception
\newcommand*{\fact}{\casop{\constant{fact}}}
\newcommand*{\id}{\mathord{\constant{id}}}
\newcommand*{\pt}{\constant{pt}}
\newcommand*{\shape}{\constant{sh}} % the basepoint of the classifying type of a group G, referred to as the designated shape of G.  Note: "\sh" is already in use.
\newcommand*{\ad}{\constant{ad}}
\newcommand*{\symm}{\casop{\constant{symm}}}
\newcommand*{\trans}{\casop{\constant{trans}}}
\newcommand*{\trp}[2][]{\casop{\constant{trp}^{#1}_{#2}}}
\newcommand*{\fst}{\casop{\constant{fst}}}
\newcommand*{\snd}{\casop{\constant{snd}}}
\newcommand*{\len}{\casop{\constant{len}}}
\newcommand*{\hd}{\casop{\constant{hd}}}
\newcommand*{\tl}{\casop{\constant{tl}}}
\newcommand*{\rev}{\casop{\constant{rev}}}
\newcommand*{\zpos}{\casop{\constant{pos}}}
\newcommand*{\zneg}{\casop{\constant{neg}}}
\newcommand*{\zzero}{\casop{\constant{zero}}}
\newcommand*{\preim}{\casop{\constant{preim}}}
\newcommand*{\fold}{\casop{\constant{fold}}}
\newcommand*{\tot}{\casop{\constant{tot}}}
\newcommand*{\Tot}{\casop{\constant{Tot}}}
\newcommand*{\funext}{\casop{\constant{funext}}}
\newcommand*{\ptw}{\casop{\constant{ptw}}}
\newcommand*{\wdg}{\casop{\constant{wdg}}}
\newcommand*{\ap}[1]{\casop{\constant{ap}}_{#1}}
\newcommand*{\apd}[1]{\casop{\constant{apd}}_{#1}}
\newcommand*{\apap}[3]{\casop{\constant{apap}_{#1}(#2)(#3)}}
\newcommand*{\apc}{\casop{\constant{ap}{\ct}}}
\newcommand*{\ns}{\casop{\constant{ns}}}
\newcommand*{\ev}{\casop{\constant{ev}}}
\newcommand*{\secfun}{\casop{\constant{sec}}}
\newcommand*{\pow}[1]{\casop{\constant{pow}}_{#1}} % power bundle/map
\newcommand*{\ve}{\casop{\constant{ve}}} % cute: the inverse of \ev
\newcommand*{\out}{\casop{\constant{out}}} % projection from copy
\newcommand*{\permgrp}[1]{\Sigma_{{#1}}}%
\newcommand*{\cast}{\casop{\constant{cast}}}
\newcommand*{\etop}[1]{\bar {#1}}   % equivalence to path
\newcommand*{\overetop}[1]{\overbracket[0.76pt][-1pt]{#1}}
\newcommand*{\casoverline}[1]{\overbracket[0.59pt][-1pt]{#1}}
\newcommand*{\ptoe}[1]{\tilde {#1}} % path to equivalence
\newcommand*{\ua}{\casop{\constant{ua}}} % univalence
\newcommand*{\cst}[1]{\casop{\constant{cst}}_{#1}} % constant function at
\newcommand*{\N}{\constant{N}} % normalizer
\renewcommand*{\ker}{\casop{\constant{ker}}}
\newcommand*{\lcm}{{\casop{\constant{lcm}}}}
\newcommand*{\typekernel}{{\constant{Ker}}}
\newcommand*{\Img}{\casop{\constant{Im}}}
\newcommand*{\img}{\casop{\constant{im}}}
\newcommand*{\image}{\Img}
\newcommand*{\Fact}{\casop{\constant{Fact}}}
\newcommand{\prj}{\constant{pr}}
\newcommand{\prjim}{\prj^{\img}}
\newcommand*{\incl}{\constant{in}}%the homomorphism in an element #1 in \Mono_G
\newcommand*{\Order}{\constant{Order}}
\newcommand*{\ord}{\casop{\constant{ord}}}% order of cycle/group element
\newcommand*{\Sub}{\casop{\constant{Sub}}}% subtype through predicates
\newcommand*{\Inj}{\casop{\constant{Inj}}}% subtype through injections
\newcommand*{\Coker}{\casop{\constant{Coker}}}
\newcommand*{\coker}{\casop{\constant{coker}}}
\newcommand*{\Nor}{\constant{Nor}}
\newcommand{\nor}{{\casop{\constant{nor}}}}
\newcommand*{\Sym}{\casop{\constant{Sym}}}
\newcommand*{\USym}{\constant{U}} % underlying symmetries (no space!) formerly \sh_G=\hs_G
\newcommand*{\US}[1]{\USym #1}%changes e.g f^\abstr changed to to Uf
\newcommand*{\ISym}{\constant{I}} % underlying identities
\newcommand*{\Card}{\casop{\constant{\#}}}
\newcommand*{\Cay}{\casop{\constant{Cay}}}
\newcommand*{\Con}{\casop{\constant{Con}}}
\newcommand*{\bunch}{\casop{\constant{bunch}}}
\newcommand*{\Ad}{\casop{\constant{Ad}}}
\newcommand*{\AC}{\casop{\constant{AC}}} % axiom of choice
\newcommand*{\lAC}[1]{\casop{#1\textrm{-}\constant{AC}}} % local axiom of choice
\newcommand*{\band}{\casop{\constant{band}}}
\newcommand*{\fiber}{\casop{\constant{fiber}}}
\newcommand*{\zs}{\casop{\constant{s}}} % successor on the integers
\newcommand*{\Succ}{\casop{\constant{succ}}}
\newcommand*{\Pred}{\casop{\constant{pred}}}
\newcommand*{\El}{\casop{\constant{El}}}
\newcommand*{\clf}{\constant{B}} % classifying space operator (no space)
\newcommand*{\B}{\constant{B}}            % without extra space
\newcommand*{\BB}{\B^2}
\newcommand*{\PP}{\constant{PP}}
\newcommand*{\grpcenter}{\casop{\constant{Z}}}
\newcommand*{\im}{\casop{\constant{im}}}
\newcommand*{\iscontr}{\casop{\mathrm{isContr}}}
\newcommand*{\isprop}{\casop{\mathrm{isProp}}}
\newcommand*{\isset}{\casop{\mathrm{isSet}}}
\newcommand*{\isgrpd}{\casop{\mathrm{isGrpd}}}
\newcommand*{\isfinset}{\casop{\mathrm{isFinSet}}}
\newcommand*{\isEq}{\casop{\mathrm{isEquiv}}}
\newcommand*{\isAb}{\casop{\mathrm{isAb}}}
\newcommand*{\isonetype}{\casop{\mathrm{is1Type}}}
\newcommand*{\isconn}{\casop{\mathrm{isConn}}}
\newcommand*{\iszeroconn}{\casop{\mathrm{is0Conn}}}
\newcommand*{\iszerotrunc}{\casop{\mathrm{is0Trunc}}}
\newcommand*{\isnconn}{\casop{\mathrm{is\mathit{n}Conn}}}
\newcommand*{\isntrunc}{\casop{\mathrm{is\mathit{n}Trunc}}}
\newcommand*{\istrans}{\casop{\mathrm{isTrans}}}
\newcommand*{\isinj}{\casop{\mathrm{isInj}}}
\newcommand*{\issurj}{\casop{\mathrm{isSurj}}}
\newcommand*{\ismono}{\casop{\mathrm{isMono}}}
\newcommand*{\isepi}{\casop{\mathrm{isEpi}}}
\newcommand*{\cy}[1]{\casop{\mathrm{cy}}_{#1}}
\newcommand*{\flt}{\casop{\constant{flt}}} % map in flattening lemma/construction


\newcommand*{\fundgrp}{\casop{\pi_1}}
\newcommand*{\fundgrpd}{\casop{\Pi_1}}

\newcommand*{\dg}[1]{\delta_{#1}} % degree m map on circle
\newcommand*{\cdg}[1]{\rho_{#1}} % degree m map on cycles (root)
\newcommand*{\mathdegree}{\textup{\textdegree}}

% Operators that produce groups typically come in pairs
% The delooping and the group itself
\makeatletter
\newcommand*{\newBcommand}{\@dblarg\@newBcommand}
\newcommand*{\newBvariable}{\@dblarg\@newBvariable}
\def\@newBcommand[#1]#2{%
  \expandafter\newcommand\csname #2\endcsname{\casop{\constant{#1}}}%
  \expandafter\newcommand\csname B#2\endcsname{\casop{\constant{B#1}}}%
  \expandafter\newcommand\csname U#2\endcsname{\casop{\constant{U#1}}}%
}
\def\@newBvariable[#1]#2{%
  \expandafter\newcommand\csname B#2\endcsname{{\mathit{B}\mathit{#1}}}%
  \expandafter\newcommand\csname USym#2\endcsname{{\mathit{U}\mathit{#1}}}%
}
\makeatother
\newBcommand{AUT}
\newBcommand{Aut}
\newBcommand{Out}
\newBcommand{OUT}
\newBcommand{INN}
\newBcommand{Inn}
\newBcommand{inn}
\newBcommand{Ker}
\newBcommand{Hom}
\newBcommand{Iso}
\newBcommand{sgn} % sign homomorphism
\newBcommand{conj} % conjugation homomorphism MB c := conj 040124
\newBcommand{Z} % center
\newBcommand[F]{FG} % free group
\newBcommand[V]{VG} % Vierer group
\newBcommand[C]{CG} % cyclic group
\newBcommand{D} % dihedral group
\newBcommand{Dic} % dicyclic group
\newBcommand[A]{AG} % alternating group
\newBcommand[O]{OG} % orthogonal/octahedral group
\newBcommand[U]{UG} % unitary group
\newBcommand{GL} % general linear group
\newBcommand{PGL} % projective general linear group
\newBcommand{SL} % special linear group
\newBcommand{SO} % special orthogonal group
\newBcommand{SU} % special unitary group
\newBcommand[\Sigma]{SG} % symmetric group


% Common multi-letter variables
% A bit of a hack. If we use \var{...} we get a big box
% with the wrong sub-/super-script positions
\newBvariable{G}
\newBvariable{H}
\newBvariable{K}
\newBvariable{L}
\newBvariable{M}
\newBvariable{N}
\newBvariable{W}
\newBvariable{A}
\newBvariable{f}
\newBvariable{g}
\newBvariable{h}
\newBvariable{i}
\newBvariable{j}
\newBvariable{k}
\newBvariable{p}
\newBvariable{q}

% Latin abbreviations and initialisms
\newcommand*{\cf}{cf.~}
\newcommand*{\ie}{i.e., }
\newcommand*{\eg}{e.g., }
\newcommand*{\viz}{viz., }

% disputed phrases
\newcommand*{\nonempty}{nonempty\xspace}
\newcommand*{\covering}{set bundle\xspace}
\newcommand*{\coverings}{set bundles\xspace}
\newcommand*{\Covering}{Set bundle\xspace}
\newcommand*{\Coverings}{Set bundles\xspace}
\newcommand*{\inftygp}{$\infty$-group\xspace}
\newcommand*{\aninftygp}{an $\infty$-group\xspace}
\newcommand*{\inftygps}{$\infty$-groups\xspace}
\newcommand*{\gporder}{cardinality\xspace}
\newcommand*{\gporders}{cardinalities\xspace}
\renewcommand*{\th}{\textsuperscript{th}\xspace}
\newcommand*{\st}{\textsuperscript{st}\xspace}
\newcommand*{\nd}{\textsuperscript{nd}\xspace}
\newcommand*{\rd}{\textsuperscript{rd}\xspace}

% relations
\let\oldequiv\equiv
\renewcommand*{\equiv}{\simeq}
\newcommand*{\weq}{\simeq}
\def\ordinaryequals{\mathchar`\=}
\newcommand*{\eq}{\mathrel{\ordinaryequals}}
% \newtocommand makes a new command for an arrow with a symbol
% on top that looks better than \xrightarrow
\newcommand{\doalign}[2]{%
 {\vbox{\offinterlineskip\ialign{\hfil##\hfil\cr$#1$\cr$#2$\cr}}}%
}
\newcommand{\newtocommand}[2]{%
  \expandafter\newcommand\csname #1to\endcsname{\mathrel{\mathchoice%
    {\doalign{\scriptstyle #2\,}{\displaystyle\to}}%
    {\doalign{\scriptstyle #2\,}{\textstyle\to}}%
    {\doalign{\scriptscriptstyle #2\,}{\scriptstyle\to}}%
    {\doalign{\scriptscriptstyle #2\,}{\scriptscriptstyle\to}}%
}}}
\newtocommand{eq}{\ordinaryequals}
\newtocommand{equiv}{\simeq}
\newtocommand{isom}{\cong}
\newtocommand{sim}{\simeq}
\newcommand*{\we}{\simto}
\newcommand*{\isom}{\cong}
\newcommand*{\liff}{\weq}
\newcommand*{\jdeq}{\oldequiv}
\newcommand*{\defeq}{\vcentcolon\mathrel{\mkern-0.8mu}\jdeq}
\newcommand*{\defequi}{\defeq} % equal by definition
\newcommand*{\defis}{\coloneqq}

% make colon active (an active topic of discussion)
\makeatletter
\def\ordinarycolon{\mathchar`\:}
\def\colon{\,\mathord\ordinarycolon\,}
\def\idxcolon{\mathrel\ordinarycolon}
\protected\edef\tikz@nonactivecolon{%
  \noexpand\ifmmode\noexpand\colon\noexpand\else:\noexpand\fi
}
\begingroup\lccode`\~=`\:\lowercase{\endgroup
  \protected\def~}{\new@ifnextchar={\coloneqq\@gobble}{\colon}}
\AtBeginDocument{\mathcode`\:="8000 }
% use prefix for displaying name prefixes, but not for sorting names
\AtBeginDocument{\toggletrue{blx@useprefix}}
\AtBeginBibliography{\togglefalse{blx@useprefix}}
\makeatother

% binary operations
\newcommand*{\ct}{*}
\newcommand*{\cto}{*_{\constant{o}}}
\newcommand*{\dblslash}{\mathbin{/\kern-3pt/}}

% paired delimiters (can change size via optional argument)
\DeclarePairedDelimiter\Trunc{\lVert}{\rVert} % truncated type
\DeclarePairedDelimiter\trunc{\lvert}{\rvert} % truncation constructor
\DeclarePairedDelimiter\merely{\lVert}{\rVert_{-1}}
\DeclarePairedDelimiter\angled{\langle}{\rangle}
\DeclarePairedDelimiter\FinOrd[] % constructor for nonempty finite ordinals
\DeclarePairedDelimiter\set\lbrace\rbrace
\DeclarePairedDelimiter\sem\llbracket\rrbracket % semantic interpretation
\DeclarePairedDelimiterX\setof[2]\lbrace\rbrace{\, #1 \,\delimsize\vert\, #2 \,}

% special truncations and truncation constructors
\newcommand*{\myTrunc}[2]{{\Trunc{#1}}{}_{#2}}
\newcommand*{\mytrunc}[2]{{\trunc{#1}}{}_{#2}}
\newcommand*{\setTrunc}[1]{{\Trunc{#1}}_0}
\newcommand*{\settrunc}[1]{{\trunc{#1}}_0}
\newcommand*{\grpdTrunc}[1]{{\Trunc{#1}}_1}
\newcommand*{\grpdtrunc}[1]{{\trunc{#1}}_1}

% footnote without number, no star: may contain \par
\newcommand\blfootnote[1]{%
  \begingroup
  \renewcommand\thefootnote{}\footnote{#1}%
  \addtocounter{footnote}{-1}%
  \endgroup}

% deprecated macros (but still in active use! MB 20250422)
\newcommand*{\invprinc}[1]{{\color{blue}\invpathsp{\sh_{#1}}}}
\newcommand*{\princ}[1]{{\color{blue}\pathsp{\sh_{#1}}}}
\newcommand*{\absprtor}[1][\agp G]{{\color{red}\pathsp{#1}^{\abstr}}}
\newcommand*{\absGTor}[1][\agp G]{{\color{red}\Tors_{#1}^{\abstr}}}
\newcommand*{\absHom}{\Hom^{\abstr}}
\newcommand*{\absGSetvar}[1][X]{\mathcal{#1}}
\newcommand*{\pathsp}[1]{\constant{P}_{\!#1}} % NB negative thin space
\newcommand*{\invpathsp}[1]{\constant{P}_{\!#1}^{-1}} % NB negative thin space
\newcommand*{\uc}[1]{{\pathsp{#1}}}%universal set bundle
\newcommand*{\abstr}{\casop{\constant{abs}}}
\newcommand*{\agp}[1]{\mathcal #1} %generic abstract group
\newcommand*{\grpcenterinc}[1]{\mathrm z_{{#1}}} %

\newcommand*{\pre}{\constant{pre}}%these may be open for discussion
\newcommand*{\preinv}{\constant{preinv}}
\newcommand*{\post}{\constant{post}}
\newcommand*{\adjoint}{\ad}
\newcommand*{\concr}{\constant{concr}}

%some special types
\newcommand*{\TG}{{1\!\!1}} % trivial group, no B symbol, experimental MAB
\newcommand*{\SetBundle}{\constant{SetBundle}}
\newcommand*{\typegroup}{\Group}
\newcommand*{\typeabgroup}{\AbGroup}
\newcommand*{\typesubgroup}{\constant{Sub}}%"gp" removed - is evident from the type of the subscript G
\newcommand*{\typemono}{\constant{Mono}}%monomorphisms into a group (subscript)
\newcommand*{\typeepi}{\constant{Epi}}%epimorphisms out of a group (subscript)
\newcommand*{\typenormal}{\constant{Nor}}
\newcommand*{\typeset}{\Set}
\newcommand*{\typeinftygp}{{\infty}\Group}
\newcommand*{\typemonoid}{\Monoid}
\newcommand*{\typetorsor}{\Tors}
\newcommand*{\pttype}{\UUp}
\newcommand*{\typeabsgp}{\Group^{\abstr}}
\newcommand*{\BSigma}{\B\Sigma}%previously \Set_{(S)} - the component of S:\Set
\newcommand*{\twist}[1]{^{\curvearrowright{#1}}}%twist by action
\newcommand*{\swap}{\constant{swap}}%loop in BC_2
\newcommand*{\Sc}{{\typeformer{S}^1}}%the circle
%\newcommand{\sbt}{\begin{picture}(-1,1)(-1,-3)\circle*{2}\end{picture}}%
\newcommand*{\sbt}{\mathchoice%
  {\vcenter{\hbox{\scriptsize\textbullet}}}%
  {\vcenter{\hbox{\scriptsize\textbullet}}}%
  {\vcenter{\hbox{\tiny\textbullet}}}%
  {\vcenter{\hbox{\fontsize{4}{5}\selectfont\textbullet}}}}% {\mathop{\bullet}}
%{\tikz[anchor=base,baseline]{\node[scale=.7,inner
%    sep=0, outer sep=0, circle]%
%    {$\bullet$};}}%
\newcommand*{\base}{{\sbt}}%point in circle
\newcommand*{\Cloop}{\mathop\circlearrowleft}% loop in circle \InfCycSet
\newcommand*{\Sloop}{\mathop\circlearrowleft}% loop in circle \Sc
\newcommand*{\qedge}{\mathop\curvearrowright}% edge in graph quotient

\newcommand*{\conncomp}[2]{{{#1}_{\left(#2\right)}}}%
\newcommand*{\univcover}[2]{{{#1}^0_{\left(#2\right)}}}%

\newcommand*{\blank}{\_}%
\newcommand*{\op}{^{\constant{op}}}%
\newcommand*{\inv}[1]{#1^{-1}}%
%% \newcommand*{\invo}[1]{#1^{-1\mathop{\constant{o}}}}%mathop deliberately to center the o
\newcommand*{\ptdto}{\to_\ast}%
\newcommand*{\ptdweto}{\equivto_\ast}%
\newcommand*{\mono}{\hookrightarrow}%
\newcommand*{\epi}{\twoheadrightarrow}%
\newcommand*{\loops}[1][\null]{\Omega^{#1}} % no space
\newcommand*{\mkgroup}[1][\null]{\underline{\Omega}^{#1}}
\newcommand*{\mkhom}[1][\null]{\underline{\Omega}^{#1}}
\newcommand*{\mkheap}{\underline{\constant{I}}}
%% \newcommand*{\cdoto}{\cdot^{\constant{o}}} % concat in pathover
\DeclareMathOperator*{\pathovercomp}{\underline\circ}

%% paths over paths
\newcommand*{\pathover}[4]{\mathchoice%
  {#1 \xrightarrow[#3]{=} #4}%
  {#1 \xrightarrow[#3]{=} #4}%
  {#1 \xrightarrow[#3]{=} #4}%
  {#1 \xrightarrow[#3]{=} #4}}
\newcommand*{\po}{\casop{\constant{po}}}
\newcommand*{\pair}{\casop{\constructor{pair}}}
\newcommand*{\rec}{\casop{\constant{rec}}}
\newcommand*{\ind}{\casop{\constant{ind}}}
\newcommand*\pathpair[2]{\casoverline{({#1},{#2})}}

%% Euclidean geometry

\newcommand*{\EE}{\mathbb E}
\newcommand*{\VV}{\mathbb V}
\newcommand*{\ES}{{\tilde \EE}}
\newcommand*{\EucObj}{\casop{\constant{EucObj}}}
\newcommand*{\OS}{{\tilde \VV}}
\newcommand*{\OrthGp}[1]{\constant{O}(#1)}
\newcommand*{\EucGp}[1]{\constant{E}(#1)}
\newcommand*{\Vectors}{\casop{\constant{Vec}}}
\newcommand*{\Points}{\casop{\constant{Pts}}}
\newcommand*{\typeRealVectorSpace}{\constant{Vect}_{\RR}}

%%%%%%%%%%%%%%%%%%%%%%%%%%%%%%%%%%%%%%%%%%%%%%%%%%%%%%%%%%%%%%%%%%%%%%%%%%%%

% Peter & Benedikt's macros for referring to coqdoc
% d2c4e86
% see https://tex.stackexchange.com/a/35314/ for help understanding the following:

\newcommand{\longhash}{e47ce20acce953129e34e021a10976ed27948a39}
\newcommand{\shorthash}{e47ce20}

%fragile, better to freeze with stable hash
\newcommand{\coqdocbasebaseurl}{https://unimath.github.io/doc/UniMath/\shorthash/}

%\coqident call are relative to this long path
\newcommand{\coqdocbaseurl}{\coqdocbasebaseurl UniMath.}
\newcommand{\urlhash}{\#}

\newcommand{\coqdocurl}[2]{\coqdocbaseurl #1.html\urlhash #2}

%nolinkurl from url or hyperref package
\newcommand{\nolinkcoqident}[1]{\nolinkurl{#1}} % TODO: give better def for this?
\makeatletter
\newcommand{\coqident}{\begingroup\@makeother\#\@coqident}
\newcommand{\@coqident}[3][]{% empty default first and optional argument
  \ifthenelse{\isempty{#2}}%
  {\nolinkcoqident{#3}}%           [optional]{}{printme}
  {\ifthenelse{\isempty{#1}}%
  {\href{\coqdocurl{#2}{#3}}{\nolinkcoqident{#3}}}% []{file}{identifier+printed}
  {\href{\coqdocurl{#2}{#3}}{\nolinkcoqident{#1}}}}% [printme]{file}{identifier}
\endgroup}
% optional argument allows link text to differ from link url
\newcommand{\coqfile}[2]{%
  \ifthenelse{\isempty{#1}}%
  {\href{\coqdocbaseurl #2.html}{#2.v}}%
  {\href{\coqdocbaseurl #1.#2.html}{#2.v}}}
\makeatother

%sususe is a replacement for subsubsection.  sususe is the same as subsubsection but numbers correctly bid
%\newenvironment{sususe}[1]{\refstepcounter{theorem}%
%\vspace{.5\baselineskip}\par\medskip\noindent%
%{\normalfont\normalsize\bfseries{\thetheorem. #1}}%
%\vspace{.5\baselineskip}\newline}
\let\sususe\subsubsection


%To get correct fraktur (BID)
\usepackage{yfonts}

% Pierre Cagne macros
  \newcommand{\loopspace}[1][]{\constructor{Aut}^2_{#1}}

%%% Local Variables:
%%% mode: latex
%%% TeX-master: "book"
%%% End:
