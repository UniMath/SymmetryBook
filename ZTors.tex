\documentclass[a4,12pt]{amsart}

\newcounter{chapter}% HACK since amsart has no chapter counter like amsbook
%and there is \numberwithin{theorem}{chapter} in macros.tex,
%seemingly overruled by \numberwithin{theorem}{section} in top.tex

\input macros
\begin{document}
\input topZTors

\section{The integers}
\label{sec:integers}

We define the type of integers in one of the many possible ways.

\begin{definition}\label{def:integers}
Let $\zet$ be the inductive type with the following three constructors:
\begin{enumerate}
\item $z_0: \zet$ for the integer number zero, 
$0 \defeq z_0$
\item $pos: \NN \to \zet$ for positive numbers,
$1 \defeq pos(0),\ldots$.
\item $neg: \NN \to \zet$ for negative numbers, 
$-1 \defeq neg(0),\ldots$
\end{enumerate}

The \emph{embedding} function $i:\NN\to\zet$ is defined by induction,
setting $i(0)\defeq z_0$, $i(S(n))\defeq pos(n)$.
Like the type $\NN$, the type $\zet$ is a set with decidable equality
and ordering relations,
and we denote its elements often in the usual way as $\ldots,-1,0,1,\ldots$.

One well-known equivalence is \emph{negation} ${-}:\zet\to\zet$, 
also called \emph{complement}, inductively defined by setting 
$-z_0\defeq z_0$, 
$-pos(n)\defeq neg(n)$, 
$-neg(n)\defeq pos(n)$.
Negation is its own inverse.

The \emph{successor} function $s:\zet\to\zet$ is defined inductively setting 
$s(z_0)\defeq pos(0)$, 
$s(pos(n))\defeq pos(S(n))$,
$s(neg(n))\defeq -i(n)$. For example, we have
$s(-1)\jdeq s(neg(0))\jdeq -i(0) \jdeq z_0 \jdeq 0$.
By induction on $n:\NN$ one proves $s(i(n))=i(S(n))$, 
so that one can say that $s$ extends $S$
on the $i$-image of $\NN$. 

The successor function $s$ is an equivalence.
It is instructive to depict iterating $s$ in both directions as 
a doubly infinite sequence containing all integers:
\[
\ldots \mapsto neg(1) \mapsto neg(0) \mapsto z_0 \mapsto pos(0) \mapsto pos(1) \mapsto \ldots
\]

The inverse $s^{-1}$ of $s$ is called the \emph{predecessor} function.
We denote the $n$-fold iteration of $s$ as $s^n$, and
the $n$-fold iteration of $s^{-1}$ as $s^{-n}$.

Addition of integers is defined inductively by setting
$z + z_0\defeq z$, 
$z + pos(n)\defeq s^{n+1}(z)$, 
$z + neg(n)\defeq s^{-(n+1)}(z)$.
Again, addition extends $+$ on the $i$-image of $\NN$,
see \cref{xca:addition-on-Z-and-N}. 
From addition and unary $-$ one can define a binary
\emph{substraction} function setting $z-y \defeq z+(-y)$.
\end{definition}

\begin{xca}\label{xca:addition-on-Z-and-N}
Show that $i(n+m)=i(n)+i(m)$ for all $n,m:\NN$.
\end{xca}


\section{Digression: the integers as a group}
\label{sec:integers-group}

In the previous section we first defined the \emph{set}
of integers $\zet$. Then we defined functions that add
algebraic stucture to $\zet$ (the structure of a free group 
with one generator). In this section we
show that this structure can also be obtained from the
symmetries of a certain element in a certain type.
This is one of the keypoints of this book, 
therefore pointed out early on.

\begin{lemma}\label{lem:one-orbit-int}
Let $s$ be as in \cref{def:integers}, and 
let $f:\zet\to\zet$ be such that $f\circ s = s\circ f$. 
  \begin{enumerate}
  \item\label{item-one-orbit} For all $z:\zet$ there is a unique $n:\NN$
such that either $z=s^{-(n+1)}(0)$, or $z=s^{n}(0)$.
  \item\label{item-f(0)-nonneg} For all $n:\NN$, if $f(0)=s^{n}(0)$, then $f=s^{n}$.
  \item\label{item-f(0)-nonpos} For all $n:\NN$, if $f(0)=s^{-n}(0)$, then $f=s^{-n}$.
  \end{enumerate}
\end{lemma}
\begin{proof}
From $f\circ s = s\circ f$ we get $f\circ s^n = s^n\circ f$
and $f\circ s^{-n} = s^{-n}\circ f$ by induction on $n:\NN$.

(\ref{item-one-orbit}) Induction on $n:\NN$ proves $s^{n}(0)=n$, 
as well as $s^{-n}(0)=-n$. Uniqueness is easy.

(\ref{item-f(0)-nonneg}) Assume $f(0)=s^{n}(0)$.  
Given $z:\zet$, let $m$ be such that either $z=s^{m}(0)$, 
or $z=s^{-(m+1)}(0)$. In the first case we calculate
\[
f(z)=f(s^{m}(0))=s^{m}(f(0))=s^{m}(s^{n}(0))=s^{n}(s^{m}(0))= s^{n}(z),
\]
so that $f=s^{n}$ by function extensionality. 
The second case is very similar;
(\ref{item-f(0)-nonpos}) goes like (\ref{item-f(0)-nonneg}).
\end{proof}

\begin{corollary}\label{cor:pre-torsor-int}
We have $\zet\equiv \sum_{f:\zet\to\zet} (f\circ s = s\circ f)$.
\end{corollary}
\begin{proof}
First observe that $f\circ s = s\circ f$ is a true proposition
for all $f=s^n$ and $f=s^{-n}$. Recall that proofs of propositions
may be left out from dependent pairs. Thus we
define $e : \zet\to \sum_{f:\zet\to\zet} (f\circ s = s\circ f)$ 
inductively by setting 
$e(z_0)\defeq \id_\zet$, 
$e(pos(n))\defeq s^{n+1}$,
$e(neg(n))\defeq s^{-(n+1)}$.
By \cref{lem:one-orbit-int}, $e$ is a well-defined equivalence.
\end{proof}

Again by \cref{lem:one-orbit-int}, if $f\circ s = s\circ f$,
then $f: \zet\to\zet$ is an equivalence. 
Using UA, we get by \cref{cor:pre-torsor-int} the equivalence
\[
\zet\equiv \sum_{f:\zet=\zet} (f\circ s \circ f^{-1} = s).
\]
Recall from \cref{sec:heavy-transport} 
that $f\circ s \circ f^{-1}$ is transport of $s$ by
conjugation with $f$. Using the characterization of equality 
in $\Sigma$-types from \cite[Theorem 2.7.2]{hottbook} we get the equivalence
\[
\zet\equiv ((\zet,s) = (\zet,s)).
\]
One particular equivalence is $e$ from the proof
of \cref{cor:pre-torsor-int}, with inverse $e^{-1}$.
We have $e^{-1}(\id_\zet) = 0$.
The type $(\zet,s) = (\zet,s)$ of symmetries of $(\zet,s)$
has a natural algebraic structure induced by
$\trans$ and $\symm$ from \cref{def:eq-symm}.
%if $f,g:(\zet,s) = (\zet,s)$, then $f\circ g:(\zet,s) = (\zet,s)$.
%Moreover, elements of $(\zet,s) = (\zet,s)$ have inverses.
This algebraic structure is transported by $e^{-1}$ to $\zet$
and gives exactly the group structure defined by ${+},{-},0$.
The proof of this observation is postponed until the notion
of a group has been defined in \cref{ch:groups}.

One important issue has been ignored up to now:
What is the type of $(\zet,s)$? 
One possible answer is: $\sum_{X:\UU}(X=X)$.
The following exercise shows that we do not get the 
property that $(X,f)=(X,f)$ is equivalent to
$(\zet,s) = (\zet,s)$ for all $(X,f) : \sum_{X:\UU}(X=X)$.
It is for this reason that in the next section
another choice will be made.

\begin{xca}\label{xca:zet-symmetries}
Figure out the symmetries of $(\zet,\id_\zet)$ (easy) and 
of $(\zet,s^2)$ (hard).
\end{xca}

\section{Z-Torsors}\label{sec:ZTorsors}




\bibliographystyle{amsplain}
\bibliography{papers}
% \printindex
\end{document}
% Local Variables:
% fill-column: 144
% latex-block-names: ("lemma" "theorem" "remark" "definition" "corollary" "fact" "properties" "conjecture" "proof" "question" "proposition")
% TeX-master: t
% End:
