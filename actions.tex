\chapter{Group actions}
\label{ch:actions}

Historically, groups have appeared because they can ``act'' on a set
(or more general objects), that is to say, they collect some of the
symmetries of the set. This is a point of view that we will return to
many times and we give the basic theory in \cref{sec:gsets}.
This section should remind the reader of the material in \cref{cha:circle},
where we dealt with the special case of the group of integers.
More generally, connected \coverings now reappear in the guise of
``transitive $G$-sets'', and these are intimately related to
the set of subgroups of a group.

Also discussed in \cref{sec:gsets} is the notion of ``$G$-torsor''.
A $G$-torsor is a $G$-set that is merely equal to the universal \covering.
The type of $G$-torsors recovers the classifying type of the group $G$,
and this idea is used in~\cref{ch:absgroup} to build the equivalence between our definition of a group and the abstract version taught in most algebra classes.

\section{Brief overview of the chapter}

After setting things up in~\cref{sec:gsets}, and
studying subgroups in~\cref{sec:subgroups},
we introduce the important
operations of taking \emph{invariant maps} and \emph{orbits} of an action
in~\cref{sec:fixpts-orbits}.
We then construct in~\cref{sec:torsors}
the fundamental equivalence between the classifying type $\BG$ of a group $G$
and the type of $G$-torsors.

In~\cref{sec:orbit-stabilizer-theorem,sec:burnsides-lemma} we
begin the study of the combinatorics of group actions.
This allows us to count for instance how many ways there are of ``coloring''
objects acted on by groups,
and it lays the groundwork for the combinatorics of finite groups
we'll be looking at in~\cref{ch:fingp}.

\section{Group actions ($G$-sets)}
\label{sec:gsets}

One of the goals of \cref{sec:Gsetforabstract} below
is to prove that the types of groups and abstract groups are equivalent.
In doing that, we are invited to explore how elements of
abstract groups should be
thought of as symmetries and introduce the notion of a $G$-set.
However, this takes a pleasant detour where we have to explore a
most important feature of groups: they can \emph{act} on things
(giving rise to manifestations of symmetries)!

%\MB{Before we handle the more complex case of abstract groups,
%let us see what this looks like for groups. Leave out?}

\begin{definition}\label{def:Gset}
  For $G$ a group, a \emph{$G$-set} is a function
  \index{group!acting on a set}\index{group action!of $G$-set}
  \[
    X : \BG\to\Set,
  \]
  and $X(\sh_G)$ is referred to as the \emph{underlying set}.
  If $p:x\eqto y$ in $\BG$,
  then the transport function $X(x)\to X(y)$ induced
  by $X(p)\defeq\trp{X}(p) : X(x)\eqto X(y)$ is also denoted by $X(p)$.
  We denote $X(p)(a)$ by $p\cdot_X a$.
  The operation $\cdot_X$ is called the \emph{group action} of $X$.
  When $X$ is clear from the context we may leave out the
  subscript $X$.\footnote{%
    Note that in this case $\cdot: (x\eqto y) \to X(x) \to X(y)$.
    See \cref{def:principaltorsor} for a special case
    where $\cdot_X$ is indeed path composition.}
  In particular, if $g:\USymG$,
  then $X(g)$ is a permutation of the underlying set of $X$.

  The type of $G$-sets is
  \glossary{GSet}{\protect{$\GSet$}}{type of $G$-sets}
  \[
    \GSet\defequi(\BG\to\Set).\qedhere
  \]
\end{definition}
\marginnote{
  Much of what follows will work equally well for $\infty$-groups;
  if $G$ is (a group or) an infinity group,
  a \emph{$G$-type} is a function $X : \BG\to\UU$,
  with \emph{underlying type} $X(\sh_G)$.
  This is an \emph{action in $\UU$}, and
  more generally, an action of $G$ on an element of type $A$
  is a function $X : \BG\to A$, see~\cref{sec:actions} below.}

\begin{example}\label{def:trivGset}
  If $G$ is a group and $X$ is a set, then $\triv_G X$ defined by
  \[\triv_G X(z)\defequi X, \quad\text{for all $z:\BG$,}\]
  is a $G$-set.
  Examples of this sort (regardless of $X$) are called \emph{trivial $G$-sets}.
\end{example}
\begin{remark}
  \label{rem:G-set-vs-set-bundle}
The reader will notice that the type of $G$-sets is equivalent to the
type of \coverings over $\BG$.
The reason we have allowed ourselves two names is that our focus is different: for a $G$-set $X:\BG\to\Set$ we focus on the sets $X(z)$, whereas when talking about \coverings the first projection $\sum_{z:\BG}X(z)\to \BG$ takes center stage.  Each focus has its advantages.
\end{remark}

\begin{example}\label{def:principaltorsor}
  If $G$ is a group, then
  \[
    \princ G:\BG\to\Set,
    \qquad\princ G(z)\defequi\pathsp{\sh_G}(z)\defequi(\sh_G \eqto z)
  \]
  is a $G$-set called the \emph{principal $G$-torsor}.\footnote{%
    The term ``$G$-torsor'' will reappear several times and will mean nothing but a $G$-set in the component of $\princ G$ -- a ``twisted'' version of $\princ G$.}
  We've seen this family before in the guise of the (preimages of the) ``universal \covering'' of \cref{def:universalcover}!

  There is nothing sacred about starting the identification
  $\sh_G \eqto z$ at $\sh_G$.
  Define more generally
  \begin{equation}\label{eq:pathsp}
    \pathsp{\blank}:\BG\to\GSet,
    \qquad
    \pathsp{y} \defeq (z \mapsto (y\eqto z)),
  \end{equation}
  Applying $\pathsp{\blank}$ to a path $q:y\eqto y'$
  induces an equivalence from $\pathsp y$ to $\pathsp {y'}$ that sends $p:y \eqto z$
  to $pq^{-1}:y'\eqto z$.
  As a matter of fact, \cref{lem:BGbytorsor} will identify $\BG$ with the type of
  $G$-torsors via the map $\pathsp{\blank}$, simply denoted as $\pathsp{}$,
  using the full transport structure of the identity type $\pathsp y(z)\jdeq(y \eqto z)$.
\end{example}

Note that the underlying set of $\princ G$ is
\[
  \princ G(\sh_G) \jdeq
  \pathsp{\sh_G}(\sh_G) \jdeq
  (\sh_G \eqto \sh_G) \jdeq \USymG,
\]
the underlying symmetries of $G$.
If we vary both ends of the identifications simultaneously,
we get another $G$-set:
\begin{example}\label{def:adjointrep}
  If $G$ is a group, then
  \[
    \Ad_G:\BG\to\UU,\qquad\Ad_G(z)\defequi(z\eqto z)
  \]
  is a $G$-set (or $G$-type) called
  the \emph{adjoint $G$-set (or $G$-type)}.\footnote{%
    Note that $\Ad_G$ also makes sense for $\infty$-groups.
    The name ``adjoint'' comes from how transport works in this case; if $p:y \eqto z$,
    then $\Ad_G(p):(y\eqto y) \equivto (z\eqto z)$ is given by conjugation:
    \[
      \Ad_G(p)(q)\eqto pqp^{-1} \text{ in } z \eqto z.
    \]
    The picture
    \[
      \begin{tikzcd}[ampersand replacement=\&]
        y \ar[r,eqr,"p"]\ar[d,eql,"q"'] \& z \ar[d,eqr,"\Ad_G(p)(q)"] \\
        y \ar[r,eql,"p"'] \& z
      \end{tikzcd}
    \]
    is a mnemonic device illustrating that it couldn't have been different,
    and should be contrasted with the picture for
    $\princ G (p):(\sh_G\eqto y)\equivto (\sh_G\eqto z)$:
    \[
      \begin{tikzcd}[ampersand replacement=\&]
        \sh_G \ar[r,eqr,"{\refl{\sh_G}}"]\ar[d,eql,"q"']
          \& \sh_G \ar[d,eqr,"\princ G(p)(q)"] \\
        y \ar[r,eql,"p"'] \& z.
      \end{tikzcd}
    \]
  }\label{ft:adjoint-transport}
Notice that by the induction principle for the circle,
\[
  \sum_{z:\BG}\Ad_G(z) \jdeq \sum_{z:\BG}(z \eqto z)
\]
is equivalent to the type of (unpointed!) maps $\Sc\to\BG$,
known in other contexts as the \emph{free loop space} of $\BG$,
an apt name given that it is the type of ``all symmetries in $\BG$.''
The first projection $\sum_{z:\BG}\Ad_G(z)\to \BG$ correspond to the function $(\Sc\to\BG)\to\BG$ given by evaluating at $\base$.
\end{example}
\begin{example}
  \label{ex:HomHGasGset}
  Let $G$ and $H$ be groups. Recall that $\Hom(H,G)$ is a set
  (\cref{lem:hom-is-set}). We will define group actions on such
  sets of homomorphisms by moving the shapes of $G$ and $H$ as
  in \cref{exa:conj-concrete}. Reusing the notation
  $\Hom(H,G)$, define for any $x:\BH$ and $y:\BG$
  \[
    \Hom(H,G)(x,y)\defequi \Hom(\mkgroup(\BH_\div,x),\mkgroup(\BG_\div,y)).
  \]
  Alternatively, by \cref{def:pointedtypes} and
  \cref{def:grouphomomorphism}, we have
  \[
    \Hom(H,G)(x,y)\jdeq
%\Copy_{\mkgroup}((\BH_\div,x) \ptdto (\BG_\div,y)). 
    \Copy_{\mkgroup}\bigl(\sum_{f:\BH_\div\to \BG_\div}(y\eqto f(x))\bigr).
  \]
  Thus the type $\Hom(H,G)$ may also be considered to be a $(H\times G)$-set
  \[
    \Hom(H,G) : (\BH\times \BG)\to\Set.
  \]
  
  We shall be particularly interested in the restriction to $G$,
  giving a $G$-set for which we again reuse the notation:
  \[
    \Hom(H,G)(y)\defequi\Hom(H,G)(\sh_H,y) %\jdeq avoid overflow
%\Copy_{\mkgroup}(\sum_{f:\BH_\div\to \BG_\div}(y \eqto f(\sh_H)))
    .\qedhere
  \]
\end{example}
\begin{xca}
  \label{xca:HomZGvsAdG}
  Provide an identification between the $G$-sets
  $\Ad_G$  and $\Hom(\ZZ,G)$
  of \cref{def:adjointrep,ex:HomHGasGset}.\footnote{%
    Hint: This is similar to \cref{ex:Zinitial}:
    identify $\Hom(\ZZ,G)(y)$ with $\sum_{z:\BG}\sum_{p:z\eqto z}(y \eqto z)$
    and use~\cref{lem:contract-away}.}
\end{xca}

\begin{definition}\label{def:map-of-Gsets}
  If $G$ is a group and $X,Y$ are $G$-sets,\footnote{%
  This definition generalizes to \inftygps and $G$-types.}
  then a
  \emph{map from $X$ to $Y$} is an element of the type
  \[
    \Hom_G(X,Y) \defeq \prod_{z:\BG}(X(z)\to Y(z)).
  \]
  When $f$ is such a map, we may write $f_z$ for $f(z)$.
  \index{map!of $G$-sets}\index{$G$-subset}
\end{definition}

\begin{remark}\label{rem:map-of-Gsets}
  Given $G$-sets $X,Y$ and a map $f$ from $X$ to $Y$,
  we have $f_w(g\cdot_X x) = g\cdot_Y f_z(x)$ for all $z,w:\BG$,
  $x:X(z)$, $g:z\eqto w$. In other words, the diagram on the right commutes:
\[
\begin{tikzcd}
  z\ar[d,eql,"g"'] &X(z) \ar[r,"f_z"] \ar[d,eql,"g\cdot_X\,\blank"']
                  &Y(z) \ar[d,eqr,"g\cdot_Y\,\blank"] \\
  w               &X(w) \ar[r,"f_w"']                & Y(w)
\end{tikzcd}
\]
An important special case is when $Y$ is the $G$-set that
is constant $\Prop$: Given a map $P$ from $X$ to $\triv_G\Prop$,
we have $P_w(g\cdot x)$ if and only if $g\cdot P_z(x)$
for all $z,w:\BG$, $x:X(z)$, $g:z\eqto w$.
This applies to the following definition.
\end{remark}

\begin{definition}\label{def:Gsubset}
  A \emph{$G$-subset of $X$} is a map from $X$ to the $G$-set that is
  constant $\Prop$. The type of all such maps is denoted by%
  \glossary(SubGX){$\protect\Sub_G(X)$}{set of $G$-subsets of $X$}
  \[
   \Sub_G(X)\defeq \Hom_G(X,\triv_G\Prop) \jdeq \prod_{z:\BG}(X(z)\to\Prop).
  \]
  Similarly to \cref{cor:subtype-same-level}, $\Sub_{G}(X)$ is a set.\footnote{%
  \label{ft:SubTotX} The type $\Sub_{G}(X)$ can be uncurried 
  (\cref{xca:Sigma-curry}) as $\Tot(X)\to\Prop$, 
  the type of subtypes of $\Tot(X)\jdeq\sum_{z:\BG}X(x)$ (\cref{def:subtype}).}
  If $P$ is a $G$-subset of $X$, then the \emph{underlying $G$-set of $P$},
  denoted by $X_P$, is defined by
  \[
  X_P(z) \defeq \sum_{x:X(z)}{P(z,x)},\quad\text{for all $z:\BG$}.\qedhere
  \]
  \glossary(Gset){$X_P$}{underlying $G$-set of $P$}
\end{definition}

\begin{xca}\label{xca:SubGX-closedSubXshG}
Show that evaluation at $\sh_G$ is an equivalence from $\Sub_G(X)$ to
\[
  \sum_{Q:\Sub(X(\sh_G))}
  \prod_{x:X(\sh_G)}
  \Bigl(Q(x)\to\prod_{g:\USymG} Q(g\cdot x)\Bigr).
\]
The latter type is the type of all
subsets of $X(\sh_G)$ that are closed under the group action.
\end{xca}

\begin{remark}
  \label{remark:GsetsareGsets}
  A $G$-set $X$ is often presented by focusing on the underlying set $X(\sh_G)$
  and providing it with a structure relating it to $G$ determining
  the entire function $X : \BG\to\Set$.

  More precisely, since $\BG$ is connected, a $G$-set $X : \BG\to\Set$ factors
  through the component
  $\conncomp \Set {X(\sh_G)} \jdeq \sum_{Y:\Set}\Trunc{X(\sh_G) \eqto Y}$
  that contains the point $X(\sh_G)$.
  Since $\BSG_{X(\sh_G)}\jdeq(\conncomp \Set {X(\sh_G)},X(\sh_G))$,
  the $G$-set $X$ can,
  without loss of information, be considered as a homomorphism from $G$ to
  the permutation group $\SG_{X(\sh_G)}$ of $X(\sh_G)$,
  classified by a pointed map
  \[
    \BG\ptdto\BSG_{X(\sh_G)}.
  \]

  The constructions in the previous two paragraphs yields the following equivalences:
  \[
    \GSet \equivto \sum_{X:\Set} (\BG \ptdto \BSG_X)
     \equivto \sum_{X:\Set}\Hom(G,\SG_X).\qedhere
   \]
\end{remark}

\begin{definition}\label{def:Gaction}
  If $G$ is a group and $X$ is a set, then an \emph{action}
  of $G$ on $X$
  is a homomorphism from $G$ to the permutation group of $\SG_X$ of $X$.%
  \index{actions!of a group on a set}
\end{definition}
By the construction in~\cref{remark:GsetsareGsets} we identify $G$-sets
and sets with an action of $G$ on a set.

\begin{xca}
Show that if $X$ is a type family with parameter type $\BG$ and $X(\sh_G)$ is a set,
then $X$ is a $G$-set.
\end{xca}

\begin{xca}\label{xca:Ad-triv-abelian}
  Prove that a group $G$ is abelian if and only if the $G$-sets $\Ad_G$ and
  $\triv_G(\USymG)$ are identical.
\end{xca}

\begin{xca}\label{xca:Ad-princ-trivial}
  Prove that a group $G$ is the trivial group if and only if the $G$-sets $\Ad_G$ and
  $\princ G$ are identical.
\end{xca}

\begin{definition}\label{def:finite-G-set}
Let $G$ be a group and $X:\BG\to\Set$ a $G$-set.
We say $X$ is \emph{finite} if the underlying set $X(\sh_G)$ is finite.
\index{finite $G$-set}
(If $X(\sh_G)$ is an $n$-element set, then so is $X(z)$, for any $z:\BG$.)
For any finite $G$-set $X$ we denote the number of elements
in $X(\sh_G)$ by $\Card(X)$, also called the \emph{cardinality} of $X$.
\index{cardinality!of finite $G$-set}
\end{definition}

\subsection{Transitive $G$-sets}
\label{sec:transitiveGsets}
We saw in~\cref{cha:circle} that connected \coverings play a special role:
In the case of the circle, classifying the group of integers $\ZZ$,
they correspond to cycles (\cref{thm:cycset-connS1cover}).

We hinted there that they are connected to subgroups, so
we now study them over a general group $G$.
As $G$-sets they are called transitive $G$-sets.
Classically, a $\abstr(G)$-set (a notion \emph{we} have yet not defined) $\mathcal X$ is said to be \emph{transitive} if there exists some $x:\mathcal X$ such that for all $y:\mathcal X$ there exists a $g:\mathcal X$ with $x=g\cdot y$.  In our world this translates to:
\begin{definition}\label{def:transitiveGset}
  A $G$-set $X:\BG\to\Set$ is \emph{transitive}\index{transitive $G$-set} if the proposition
  \[
    \istrans(X) \defequi
    \exists_{x:X(\sh_G)} \prod_{y:X(\sh_G)} \exists_{g:\USymG} x=g\cdot y
  \]
  holds.
\end{definition}
\begin{remark}
  In other words, $X$ is transitive if and only if there exists
  some $x:X(\sh_G)$ such that the map $\blank\cdot x:\USymG\to X(\sh_G)$ is
  surjective.

  Note also that by connectedness (cf.~\cref{xca:component-connected})
  it is equivalent to demand this over all $z:\BG$:
  \begin{equation}\label{eq:Gset-trans-gen}
    \prod_{z:\BG}\exists_{x:X(\sh_G)}
    \prod_{y:X(\sh_G)}\exists_{g:z\eqto z}x=g\cdot y.
  \end{equation}

  Yet another equivalent way of expressing that $X$ is transitive is to say
  that $X(\sh_G)$ is nonempty and for any $x,y:X(\sh_G)$ there
  exists some $g:\USymG$ with $x = g\cdot y$.
\end{remark}

\begin{lemma}
  \label{lem:conistrans}
  A $G$-set is transitive if and only if the associated \covering is connected.
\end{lemma}
\begin{proof}
  Consider a $G$-set $X:\BG\to\Set$ and the associated \covering
  $f:\tilde X\to\BG$ where $\tilde X\defequi\sum_{y:\BG}X(y)$ and $f$
  is the first projection.  Now, $\tilde X$ is connected if and only
  if there exists a $z:\BG$ and a $x:X(z)$ such that for
  all $w:\BG$ and $y:X(w)$ there exists some $g:z\eqto w$ such that $y=g\cdot x$.
  Since $\BG$ is connected, this is equivalent to asserting that there
  exists some $x:X(\sh_G)$ such that for all $y:X(\sh_G)$ there exists
  some $g:\USymG$ such that $x=g\cdot y$.
\end{proof}

The next lemma is an analog of~\cref{cor:ConnCycles},
but for a general group and transitive \covering
we only get injectivity, not an equivalence.
The action in \cref{fig:not-normal,fig:not-normal-graph}
illustrates what can go wrong.
We'll study exactly when we get surjectivity in~\cref{sec:normal}
on ``normal'' subgroups.
\begin{marginfigure}
  \noindent\begin{tikzpicture}[scale=.1]
    \coordinate (two)   at (0, 10);
    \coordinate (one)   at (0, 6);
    \coordinate (zero)  at (0, 2);
    \coordinate (base)  at (0,-5);

    \pgfmathsetmacro\cc{.55228475}% = 4/3*tan(pi/8)
    \pgfmathsetmacro\cy{2*\cc}%
    \pgfmathsetmacro\cx{10*\cc}%
    \pgfmathsetmacro\intx{3.5}%
    \pgfmathsetmacro\inty{1.5}%
    \pgfmathsetmacro\ay{.35165954}%

    % right 3-cycle
    \draw[casblue] (zero) .. controls ++(0,-\cy+\ay) and ++(-\cx,-\ay)
    .. (10,1) .. controls ++(\cx,+\ay) and ++(0,-\cy-\ay)
    .. (20,4)
    \foreach \y in {4,8} {
      .. controls ++(0,\cy + \ay) and ++(\cx,-\ay)
      .. (10,3 + \y) .. controls ++(-\cx,\ay) and ++(0,\cy-\ay)
      .. (0,2 + \y) .. controls ++(0,-\cy+\ay) and ++(-\cx,-\ay)
      .. (10,1 + \y) .. controls ++(\cx,\ay) and ++(0,-\cy-\ay)
      .. (20,4 + \y) }
    .. controls ++(0,+\cc) and ++(\cx,\ay)
    .. (10+\intx,12 + \inty) .. controls ++(-\cx,-\ay) and ++(\cx,\ay)
    .. (10-\intx,2 + \inty) .. controls ++(-\cx,-\ay) and ++(0,\cc)
    .. (zero);

    % left 2-cycle
    \draw[casred] (one) .. controls ++(0,-\cy+\ay) and ++(\cx,-\ay)
    .. (-10,5) .. controls ++(-\cx,+\ay) and ++(0,-\cy-\ay)
    .. (-20,8) .. controls ++(0,\cy + \ay) and ++(-\cx,-\ay)
    .. (-10,11) .. controls ++(+\cx,\ay) and ++(0,\cy-\ay)
    .. (two) .. controls ++(0,-\cy+\ay) and ++(\cx,-\ay)
    .. (-10,9) .. controls ++(-\cx,\ay) and ++(0,-\cy-\ay)
    .. (-20,12) .. controls ++(0,+\cc) and ++(-\cx,\ay)
    .. (-10-\intx,12 + \inty) .. controls ++(\cx,-\ay) and ++(-\cx,\ay)
    .. (-10+\intx,6 + \inty) .. controls ++(\cx,-\ay) and ++(0,\cc)
    .. (one);

    % left 1-cycle
    \draw[casred] (zero) .. controls ++(0,\cy) and ++(\cx,0)
    .. (-10,4) .. controls ++(-\cx,0) and ++(0,\cy)
    .. (-20,2) .. controls ++(0,-\cy) and ++(-\cx,0)
    .. (-10,0) .. controls ++(\cx,0) and ++(0,-\cy)
    .. (zero);

    % base right
    \draw (base) .. controls (0,-5+\cy) and ++(-\cx,0)
    .. (10,-3) .. controls ++(\cx,0) and ++(0,\cy)
    .. (20,-5) .. controls ++(0,-\cy) and ++(\cx,0)
    .. (10,-7) .. controls ++(-\cx,0) and ++(0,-\cy) .. (base);
    % base left
    \draw (base) .. controls (0,-5 + \cy) and (-10+\cx,-3)
    .. (-10,-3) .. controls (-10-\cx,-3) and (-20,-5 + \cy)
    .. (-20,-5) .. controls (-20,-5 - \cy) and (-10-\cx,-7)
    .. (-10,-7) .. controls (-10+\cx,-7) and (0,-5 - \cy)
    .. (base);

    % draw dots last
    \node[dot,label=above:$x$] (ntwo) at (two) {};
    \node[dot] (none)   at (one) {};
    \node[dot] (nzero)  at (zero) {};
    \node[dot] (nbase)  at (base) {};
  \end{tikzpicture}
  \caption{A $\mkgroup(\Sc\vee\Sc)$-set for which $\protect\ev_x$ is not
   surjective. At the bottom the type $\Sc\vee\Sc$ is visualized as
   two circles with a common base point. }
  \label{fig:not-normal}
\end{marginfigure}

\begin{marginfigure}
  \noindent\begin{tikzpicture}
    \pgfmathsetmacro{\len}{1}
    \node[vertex] (n1) at (0:\len) {};
    \node[vertex,label=above:$x$] (n2) at (120:\len) {};
    \node[vertex] (n3) at (240:\len) {};
    \begin{scope}[every to/.style={bend right=22}]
      % generator a
      \draw[gena] (n1) to (n2);
      \draw[gena] (n2) to (n3);
      \draw[gena] (n3) to (n1);
    \end{scope}
    % generator b
    \draw[genb] (n1) to[out=-30,in=30,looseness=25] (n1);
    \draw[genb,out=205,in=155] (n2) to (n3);
    \draw[genb,out=45,in=-45] (n3) to (n2);
  \end{tikzpicture}
  \caption{Alternative representation of the $\mkgroup(\Sc\vee\Sc)$-set
    from~\cref{fig:not-normal},
    using colors and arrows to represent which
    parts lies over which circle in which orientation.}
  \label{fig:not-normal-graph}
\end{marginfigure}

\begin{lemma}
  \label{lem:evisinjwhentransitive}
  Let $X,Y:\BG\to\Set$ be $G$-sets. Let $z:\BG$ and $x:X(z)$.
  If $X$ is transitive, then the evaluation map
  \[
    \ev_x:\Hom_G(X, Y)\to Y(z),\qquad \ev_x(f)\defequi f_z(x)
  \]
  is injective.\footnote{%
    Recall that for type families $X,Y:T\to\UU$, and
    $f:\prod_{z:T}(X(z)\to Y(z))$, we may write $f_z:(X(z)\to Y(z))$
    (instead of the more correct $f(z)$) for its evaluation at $z:T$.}
\end{lemma}
\begin{proof}
  Fix a value $y:Y(z)$, and consider an $f:\Hom_G(X,Y)$ with $f_z(x)=y$.
  We will show that $f$ is uniquely determined by this.
  Let $w:\BG$ and  $x':X(w)$. It suffices to show that the value
  of $f_w(x')$ is independent of $f$.
  For any $g:z\eqto w$ such that $g\cdot_X x=x'$
  (which exists by the transitivity of $X$, using \cref{lem:conistrans})
  we have
  \[
    f_w(x')=f_w(g\cdot_X x)=g \cdot_Y f_z(x)=g \cdot_Y y,
  \]
  using~\cref{rem:map-of-Gsets},
  and the latter value indeed doesn't depend on $f$.
  Since we're proving a proposition, we are done.
\end{proof}

Via function extensionality,
the identity type $X \eqto Y$, for $G$-sets $X,Y$
is a subtype of the type $\Hom_G(X,Y)$.
Hence we likewise have that evaluation at some $x:X(z)$ is an
injection
\[
  \ev_x:(X \eqto Y)\to Y(z).
\]
\begin{xca}\label{xca:not-normal}
Reverse engineer the $\mkgroup(\Sc\vee\Sc)$-set in \cref{fig:not-normal,fig:not-normal-graph}.
Let's call it $X$. Show that $X\eqto X$ is contractible.
Conclude that $\ev_x$ cannot be surjective.
(Hint: the induction principle for $\Sc\vee\Sc$ is a generalization
of the induction principle for the circle to two loops.)
\end{xca}

\subsection{Actions in a type}
\label{sec:actions}
Oftentimes it is interesting not to have an action on a set, but on an element in any given type (not necessarily the type of sets).  For instance, a group can act on another, giving rise to the notion of the semidirect product in \cref{sec:Semidirect-products}.  We will return these more general types of actions many times.

\begin{definition}\label{action}
  If $G$ is any group\footnote{%
  Even an $\infty$-group in the sense of \cref{sec:inftygps}.}
  and $A$ is any type of objects,
  then we define an \emph{action of $G$ in $A$} as a function
  \[
    X : \BG \to A.
  \]
  The particular object of type $A$ being acted on is $X(\sh_G):A$,
  the \emph{underlying object},
  and the action itself is given by transport.%
  \index{action!of a group in a type}

  Fixing $a:A$ as the underlying object, we define an \emph{action of $G$ on $a$}
  to be a homomorphism from $G$ to $\Aut_A(a)$.%
  \index{action!of a group on an element}
\end{definition}
This generalizes our earlier definition of $G$-sets
from~\cref{def:Gset}, $X : \BG \to \Set$,
and harmonizes with~\cref{remark:GsetsareGsets}, relating $G$-sets and
actions of $G$ on a set.
Indeed, we identify
an action of $G$ in $A$ with a pair of an underlying object
$a:A$ and an action of $G$ on $a$:
\[
  (\BG \to A) \equivto \sum_{a:A}\Hom(G,\Aut_A(a))
\]
This equivalence maps an action $X:\BG\to A$
to the pair consisting of $a \defeq X(\sh_G)$
and the homomorphism represented by the pointed map
from $\BG$ to the pointed component $\conncomp A a$ given by $X$.

\begin{definition}\label{std-action}
  The \emph{standard action} of $G$ on its designated shape $\sh_G$ is obtained by
  taking $A \defeq \BG$ and $X \defeq \id_{\BG}$.
\end{definition}

\begin{example}\label{ex:S2-acts-on-C3}
  The symmetric group $\SG_2$ acts on the cyclic group $\CG_3$ as follows.
  Given a $2$-element set $S$ consider the
  type $\sum_{X:\Set}S \to X\to X$ of pairs $(X,f)$ of a set $X$
  and a ``pair'' of functions $f_s:X\to X$ (one for each $s:S$).
  Within this type we have the pair $(\bn 1 \amalg S,f)$,
  where
  \begin{align*}
    f_s(\inl 0)     &\defeq \inr s,\\
    f_s(\inr s)        &\defeq \inr{\swap(s)},\\
    f_s(\inr{\swap(s)}) &\defeq \inl 0.
  \end{align*}
  Then $G(S) \defeq \Aut_{\sum_{X:\Set}S\to X\to X}(\bn1\amalg S,f)$ defines an action
  $\BSG_2 \to \Group$.\footnote{%
    If $S$ is $\set{s,s'}$, then we can picture the
    designated shape as follows,
    where the blue and red arrows denote $f_s$ and $f_{s'}$,
    respectively:\par
    \begin{tikzpicture}
    \draw (-.1,1) ellipse (.35 and .35);
    \node (X) at (0,1.5) {$\bn 1$};
    \draw (1,1) ellipse (.4 and 1);
    \node (Y) at (.9,2.2) {$S$};
    \node[dot,label=left:$0$] (x) at (0,1) {};
    \node[dot,label=above:$s$]  (s1) at (1,1.5) {};
    \node[dot,label=below:$s'$] (s2) at (1,.5) {};
    \draw[dashed] (0.6,1.1) ellipse (1.2 and 1.6);
    \begin{scope}[every to/.style={bend left=30}]
      % generator a
      \draw[gena] (x) to (s1);
      \draw[gena] (s1) to (s2);
      \draw[gena] (s2) to (x);
    \end{scope}
    % generator b
    \draw[genb] (x) to (s2);
    \draw[genb] (s2) to (s1);
    \draw[genb] (s1) to (x);
    \node (XY) at (-0.75,2.35) {$\bn1\amalg S$};
  \end{tikzpicture}}
  Furthermore, we identify $G(\bool)$ with $\BCG_3$ by mapping
  a shape $(X,f)$ in $\BG(\bool)$ to the $3$-cycle $(X,f_\yes)$
  and identifying the $3$-cycle $(\bn1\amalg\bool,f_\yes)$, for the $f$ defined above,
  with the standard $3$-cycle $(\bn3,\zs)$, correlating $\inl 0$ with $0:\bn 3$.
\end{example}
\begin{xca}\label{xca:AutC3}
  Show that action of $\SG_2$ on $\CG_3$ from~\cref{ex:S2-acts-on-C3}
  gives an identification $\SG_2 \eqto \Aut(\CG_3)$.
\end{xca}

\begin{example}
  By composing constructions we can build new actions
  starting from simple building blocks.
  For example, the standard action of symmetric group $\SG_n$
  is to permute the elements of the standard $n$-element set $\bn n$.
  Composing with the projection $\BSG_n \to \Set$,
  we get the corresponding standard $\SG_n$-set.\footnote{%
    Check that this action is transitive for $n>0$.}
  Composing further with the operation $\blank \to \bool : \Set \to \Set$,
  we get the action of $\SG_n$ on the set of decidable subsets of $\bn n$.
\end{example}

\section{Subgroups}
\label{sec:subgroups}
In our discussion of the group $\ZZ\defequi\Aut_{\Sc}(\base)$ of integers
in \cref{cha:circle} we discovered that some of the symmetries of $\base$
were picked out by the degree $m$ function $\dg{m}: \Sc\to\Sc$
(for some particular natural number $m>0$, see \cref{def:mfoldS1cover}).  
On the level of the set $\base\eqto{}\base$, the symmetries picked out are
all the iterates (positive or negative or even zero-fold) of $\Sloop^m$.
The important thing is that we can compose or invert any of the iterates
of $\Sloop^m$ and get new symmetries of the same sort (because of
distributivity $mn_1+mn_2=m(n_1+n_2)$). So, while we do not get all
symmetries of $\base$ (unless $m=1$), we get what we'd like to call 
a subgroup of the group of integers.

The case of $m=0$ is special. The iterates of $\Sloop^0$, \ie of
$\refl{\base}$, can also be composed and inverted, never to give
something else than $\Sloop^0$ itself. This is what we'd like to call
the trivial subgroup of the group of integers. 
We can pick out the single symmetry $\Sloop^0$
by the constant map $\cst{\base} : \bn 1 \to \Sc$.

Both $\dg{m}$ and $\cst{\base}$ can trivially be pointed to make them
into classifying maps of homomorphisms that are injections on the
respective sets of symmetries. Using \cref{cor:dgm-conncov},
each $\dg{m}$ is a pointed connected \covering over the circle, and
$\cst{\base}$ is even the universal \covering by 
\cref{lem:univ-cover-of-groupoid}. Finally, \cref{lem:conistrans}
gives yet another equivalent view, namely the of pointed transitive
$G$-sets. This view will now be used for our first formal definition
of the notion of a subgroup of a group $G$.

\subsection{Subgroups through $G$-sets}

The idea is that a $G$-set $X$ picks out those symmetries in $G$
that keep a chosen point of $X(\sh_G)$ in place. For this to work well
we need to point $X(\sh_G)$ and $X$ must be transitive.
%so that the set of symmetries that are picked out is closed under composition and reverse.

\begin{definition}\label{def:set-of-subgroups}
  For any group $G$, define the type of \emph{subgroups of $G$} as%
  \index{type!of subgroups of a group}%
  \glossary(SubG){$\protect\Sub(G)$}{type of subgroups of $G$}
  \[
    \Sub(G)\defequi\sum_{X:\BG\to\Set}{\,}X(\sh_G)
    \times\istrans(X).
  \]
  The \emph{underlying group} of the subgroup $(X,x) : \Sub(G)$ is\footnote{%
    To lighten the notation, we leave out the proof that $X$ is transitive.
    (Otherwise, we would write $(X,x,!):\Sub(G)$.)
    In~\cref{rem:notationsubgroup} below we'll set out further notational
    conveniences regarding subgroups.}
  \[
    \mkgroup \biggl(\sum_{z:BG}X(z),(\sh_G,x)\biggr).\qedhere
  \]
\end{definition}

\begin{xca}\label{xca:group-Xx!}
Show that $\sum_{z:BG}X(z)$ above is a connected groupoid.
Hint: use \cref{lem:conistrans}.
\end{xca}

As an example, recall from \cref{def:RmtoS1} the $\Sc$-set
$R_m : \Sc\to\Set$ defined by $R_m(\base) \defeq \bn m$ and
$R_m(\Sloop) \defis \etop\zs$. Here $m>0$ so that we can point
$R_m$ by $0: R_m(\base)$.\footnote{Any element of $\bn m$ would do.}
Transitivity of $R_m$ is obvious.
Which symmetries $p: \base\eqto\base$ are picked out by $R_m$?
Those that keep the point $0: R_m(\base)$ in place, that is,
those that satisfy $R_m(p)(0)=0$, \ie $p=\Sloop^{mk}$ for some integer $k$.
Given $\alpha_m$ in \cref{con:psi-alpha-m}, it should not come as a
surprise that these are precisely the symmetries picked out by $\dg{m}$.

The case of $m=0$ connects to another old friend, the $\Sc$-set
$R : \Sc\to\Set$ defined by $R(\base) \defeq \zet$ and
$R(\Sloop) \defis \etop\zs$, see \cref{def:RtoS1}.
Again we point by $0: R(\base)$ and transitivity of $R$ is obvious.
The only symmetry that keeps $0$ in place is $\refl{\base}$,
since $R(\Sloop^k)(0) = \zs^k(0) = k = 0$ if and only if $k=0$.
Again, no surprise in view of the results in \cref{sec:symcirc}
identifying $R$ as the universal \covering over $\Sc$.

The following result is analogous to the fact that $\Sub(T)$ is
a set for any type $T$, see \cref{def:subtype}. It captures
that the essence of picking out symmetries (or picking out elements
of a type), is a predicate, like $R_m(p)(0)=0$ above.

\begin{lemma}
  \label{lem:SubGisset}%
  The type $\Sub(G)$ is a set, for any group $G$.
\end{lemma}
\begin{proof}
Let $G$ be a group, and let $X$ and $X'$ be transitive $G$-sets with
points $x:X(\sh_G)$ and $x':X'(\sh_G)$. If $f,f': X\eqto X'$,
then both $f$ and $f'$ can be viewed as families of equivalences of
type $X(z) \equivto X'(z)$, parameterized by $z:\BG$,
with $f_{\sh_G}$ and $f'_{\sh_G}$ mapping $x$ to $x'$.
Now \cref{lem:evisinjwhentransitive} applies and we get $f=f'$.
Hence $X\eqto X'$ is a proposition and the lemma is proved.
\end{proof}

\begin{example}
  \label{exa:fix1subSGn}%
Consider the symmetric group $\SG_n$ from~\cref{ex:groups}\ref{ex:permgroup},
for some $n>0$. It has a canonical action,
the $\SG_n$-set $X : \BSG_n \to\Set$ given by $X(A,!)\defeq A$
for any $A:\FinSet_n$, which is obviously transitive.
For any $k:\bn n$, we can point $X$ by
$k: X(\sh_{\SG_n}) \jdeq \bn n$.\footnote{The choice of the point
does matter for the symmetries that are picked out.}
Thus we have $(X,k):\Sub(\SG_n)$.
The symmetries that are picked out are those $\pi : \bn n \eqto \bn n$
that satisfy $(\pi \cdot_X k) = k$.\footnote{%
This uses the alternative notation for the group action of $X$
introduced in \cref{def:Gset}.}
In other words, $\pi$ keeps $k$ in place and can be any permutation
of the other elements of $\bn n$.
From the next~\cref{xca:n-is-ptd-n+1}
we get that the underlying group of each $(X,k)$
is isomorphic to $\SG_{n-1}$.
\end{example}

\begin{xca}\label{xca:n-is-ptd-n+1}
  Give an equivalence from the type of
  $n$-element sets to the type of pointed $(n{+}1)$-element sets.
  Hint: use~\cref{xca:finsets-decidable}.
\end{xca}

\begin{xca} \label{xca:A-is-A-1+1}
  For any set $A$ with decidable equality,
  give an equivalence from $A$ to $\sum_{B:\UU}(A\eqto(B+\bn1))$.
\end{xca}

For yet another example, consider the cyclic group $\CG_6$ of order $6$; perhaps visualized as the rotational symmetries of a regular hexagon,  \ie the rotations by $2\pi\cdot m /6$, where $m=0,1,2,3,4,5$.
The symmetries of the regular triangle (rotations by $2\pi\cdot m/3$, where $m=0,1,2$) can also be viewed as symmetries of the hexagon.
Thus there is a subgroup of $\CG_6$ which, as a group, is isomorphic to $\CG_3$.\marginnote{Make a TikZ drawing of the hexagon and triangle inscribe in it.}
%LINK TO CH 3 AND SQUARE ROOT OF 6 BUNDLE

\begin{example}
  \label{exa:C3subC6}%
Recall from \cref{ex:cyclicgroups} the definition 
$\CG_6\defeq\Aut_\Cyc(\bn6,\zs)$.
In order to obtain $\CG_3$ as a subgroup we can define
$F : \CG_6 \to \Set$ defined by $F(X,t) \defeq X/2$ for all $(X,t):\BCG_6$,
where $X/2$ is defined in \cref{sec:mthroot} as the quotient of $X$
modulo identifying elements that are an even power of $t$ away from each other.
Clearly, $F$ is a transitive $G$-set.
On symmetries, $F$ maps $\pi : (\bn6,\zs) \eqto (\bn6,\zs)$ to 
$([k] \mapsto [\pi(k)]) : (\bn6,\zs)/2 \eqto (\bn6,\zs)/2$.\footnote{%
The function $[k] \mapsto [\pi(k)]$ is well-defined since
permutations that commute with $\zs$ preserve distance.}
The symmetries $\pi$ satisfying $F(\pi)([0])=[0]$ are the
even powers of $\zs$.\footnote{%
In view of \cref{cor:id-m-cycle}, these symmetries can be visualized 
by the vertices of the regular triangle above.
The same is true for the symmetries picked out by $F(\pi)([1])=[1]$.
Both $F(\pi)([0])=[1]$ and $F(\pi)([1])=[0]$ give the other inscribed
regular triangle.} 
The subgroup that we have defined above is $(F,[0],!) : \Sub_{\CG_6}$.
The underlying group of $(F,[0],!)$ is 
$\mkgroup(\sum_{(X,t):\Cyc_6} X/2, ((\bn6,\zs),[0]))$.
Using $\rho_2: \BCG_3 \ptdto \BCG_6$ from \cref{lem:deg-m-on-Cyc},
and the equivalence between $X/2$ and $\inv\rho_2 (X,t)$ from
\cref{thm:fiber-cdg}, and the equivalence from \cref{lem:sum-of-fibers},
we get an equivalence of the underlying group of  $(F,[0],!)$ and $\CG_3$.
\end{example}

There are other subgroups of $\CG_6$, and in this example they are accounted for simply by the various factorizations of the number $6$.

\subsection{Subgroups as monomorphisms}

We now give a second, equivalent definition of a subgroup,
generalizing the examples $\dg{m}$ and $\cst{\base}$ from
the introduction of this chapter. Recall that both
$\USym\dg{m}$ and $\USym\cst{\base}$ are injective.
Also recall \cref{cor:fib-vs-path}\ref{set-fib-vs-path-point},
which implies that $\USymf$ is injective iff $\Bf$ is a \covering,
for any homomorphism $f$.


%\newcommand{\typemono}{Mono}
\begin{definition}
  \label{def:typeofmono}
  Let $G$ and $H$ be groups. A homomorphism $i: \Hom(H,G)$ is
  a \emph{monomorphism},\index{monomorphism} denoted by $\ismono(i)$,
  %\glossary(isMono){$\protect{\ismono(i)}$}{proposition stating that
  %$i$ is a monomorphism of groups} gives error: culprit \ismono(i)?
  if $\USymi:\USymH\to \USymG$ is an injection  
  (all preimages of $\USymi$ are propositions).
  
  The \emph{type of monomorphisms into $G$}\footnote{%
  The similarity of this type with the type of subtypes
  $\Sub_T \defeq \sum_{S:\UU}\sum_{f:S\to T}\isinj(f)$ in
  \cref{def:subtype} is not coincidental, and the remarks made 
  there in \cref{ft:incl-vs-inj} apply here as well.
  
  In particular, the identity type of $\typemono_G$
  identifies precisely the triples that define the same subgroup,
  namely when their homomorphisms differ by precomposition by an 
  identification of their underlying groups.
  
  \MB{We should add in \cref{ch:absgroup} an equivalence between
  $\typemono_G$ and the subsets of $\USymG$ with the usual closure
  properties --- the ultimate proof that we have the right 
  notion of concrete subgroup.}
  }
  \index{type! of monomorphisms into a group}
  \glossary(MonoG){$\protect{\typemono_G}$}{type of monomorphisms%
  into the group $G$} 
  is
  \[
  \typemono_G\defequi\sum_{H:\typegroup}\sum_{i:\Hom(H,G)}\ismono(i).
  \]
  
  We call $H$ the \emph{underlying group} of $(H,i,!) : \typemono_G$.
  
  A monomorphism $(H,i,!)$ into $G$ is:
      \begin{enumerate}
      \item \emph{trivial}\index{trivial monomorphism} 
      if $H$ is the trivial group;\footnote{This amounts to $Bi$
      being the universal \covering over $BG$, see \cref{def:univ-cover}}
%.contractible (or, equivalently, if $\USymH$ is contractible),
      \item \emph{proper}\index{proper monomorphism} if $i$ is 
      not an isomorphism.\qedhere
      \end{enumerate}
\end{definition}
    
\begin{example}
  \label{ex:SGninSGn+1}
   \marginnote{
     That $i:\SG_2\to\SG_3$ is a monomorphism can visualized as follows:
     if $\SG_3$ represent all symmetries of an equilateral triangle in the
     plane (with vertices $1$, $2$, $3$), then $i$ is represented by the
     inclusion of the symmetries leaving $3$ fixed; \ie reflection through
     the line marked with dots in the picture.
                         $$\xymatrix{&3\ar@{.}[dd]&\\&&\\
             1\ar@{-}[uur]\ar@{-}[rr]&            &2\ar@{-}[uul]}$$}
We will present the subgroups from \cref{exa:fix1subSGn} with
monomorphisms. For each $n:\NN$, consider the  homomorphism 
$i_n:\Sigma_n\to\SG_{n+1}$ of permutation groups
with $\Bi_n$ sending $A:\BSG_n\defequi \FinSet_n$ to $A+\true:\BSG_{n+1}$.
As pointing path we take the reflexivity path.
This is a monomorphism since $\USymi_n:\USym\SG_n\to\USym\SG_{n+1}$ 
is an injection, extending any permutation $\pi$ of $\bn{n}$ to 
a permutation of $\bn{n}+\bn1$ by adding the last element as a
fixed point.

In the picture in the margin we have taken $n=3$ and
$\set{1,2,3}$ for $\bn{3}$. How can we obtain the other proper,
non-trivial subgroups of $\SG_3$? First of all, one should not
expect to find all subgroups through monomorphisms $j:\SG_2\to\SG_3$,
see \cref{xca:C3subSG3}. Using only $\SG_2$, the
two other subgroups can be obtained by varying the pointing
path of $i_3$. These pointing paths are induced by the permutations 
of $\bn{3}$. In \cref{xca:SG2subSG3} you are asked to elaborate each case.
\end{example}

\begin{xca}\label{xca:SG2subSG3}
Calculate $\im(\USymi_3)$ for each pointing path $\pi:\bn{3}\eqto\bn{3}$.
\end{xca}

\begin{xca}\label{xca:C3subSG3}
Define monomorphisms $j,j':\CG_3\to\SG_3$ such that $\USymj\neq \USymj'$
while $(\CG_3,j,!)$ and $(\CG_3,j',!)$ can be identified.
\end{xca}

\begin{example}
  \label{ex:prodinclismono}
  If $G$ and $H$ are groups, then $i_G : G \ptdto (G\times H)$ with
  $\Bi_G(z) \defeq (z,\sh_H)$, pointed by reflexivity, is a monomorphism:
  $\USymi_G$ maps $g:\USymG$ to $(g,\sh_H)$ and is obviously injective.
  We call $i_G$ the \emph{first inclusion} and we have a similar
  \emph{second inclusion} $i_H : H \ptdto (G\times H)$.
  % More generally, if $i:\Hom(H,G)$ is a homomorphism for which there (merely) exists a homomorphism $f:\Hom(G,H)$ such that $\id_H=fi$, then $i$ is a monomorphism.
  \end{example}


\begin{lemma}\label{lem:SubG=MonoG}
  Let $G$ be a group.
  The map sending $(X,\pt,!) : \Sub_G$ to the monomorphism classified by  
  $\fst: (\sum_{z:\BG}X(z),(\sh_G,\pt))\ptdto\BG$, pointed by
  reflexivity, yields an equivalence from $\Sub_G$ to $\typemono_G$.
\end{lemma}

\begin{proof}
   The inverse equivalence is $E$ defined as follows:
 $$E:\typemono_G\to\Sub_G,\qquad 
   (H,i,!)\mapsto E(H,i,!)\defequi (\Bi_\div^{-1},(\sh_H,\Bi_\pt),!),$$
 %\glossary(E){$E$}{equivalence from $\typemono_G$ to $\\Sub_G$}
  where the monomorphism $i:\Hom(H,G)$ is given by 
  the pointed map $(\Bi_\div,\Bi_\pt):\BH\ptdto\BG$.
  The preimage function $\Bi_\div^{-1}:\BG\to\Set$ is a transitive $G$-set 
  since $i$ is a monomorphism, and $(\sh_H,\Bi_\pt):\Bi_\div^{-1}(\sh_G)
  \defequi \sum_{x:\BH}(\sh_G\eqto{}\Bi_\div(x))$.
\end{proof}

 \begin{example}\label{exa:EforSG3}
  In this example we explain how the equivalence between
  $\typemono_G$ and $\Sub_G$ works in the special case
  $G\jdeq\SG_3$ and with two versions of the same subgroup,
  both elaborated above.
  
  Recall $(\SG_2,i_3,!):\typemono_{\SG_3}$ with
  $i_3: \SG_2\ptdto\SG_3 : B\mapsto (B+\true)$ from \cref{ex:SGninSGn+1}.
  The preimage function $\inv\Bi_3$ maps any $A:\BSG_3$
  to $\sum_{B:\BSG_2}(A\eqto{}(B+\true))$.
  In particular we have $(\bn2,\refl{\bn3}):\inv\Bi_3(\bn3)$
  (recall that $i_3$ is pointed by reflexivity).
   
  We have $E(\SG_2,i_3,!)\jdeq(\inv\Bi_3,(\bn2,\refl{\bn3}),!)$.
  Going back as in \cref{lem:SubG=MonoG} we get 
  $(\sum_{A:\BSG_3}\inv\Bi_3(A),\fst,!)$. Using \cref{lem:sum-of-fibers}
  one sees that, indeed, the latter monomorphism 
  can be identified with $(\SG_2,i_3,!)$.
  
  Why do we say that $(X_3,3,!):\Sub_{\SG_3}$ from \cref{exa:fix1subSGn}
  defines the same subgroup as $(\SG_2,i_3,!):\typemono_{\SG_3}$
  from \cref{ex:SGninSGn+1}? The reason is that they pick out
  the same symmetries in $\SG_3$, as argued in these examples.
  Moreover, $(X_3,3,!)$ and $E(\SG_2,i_3,!)$ can be identified.
  Note that $X_3(A,!) \jdeq A$ and 
  $\inv\Bi_3 \jdeq \sum_{B:\BSG_2}(A\eqto{}(B+\true))$.
  Now apply \cref{xca:A-is-A-1+1} and verify that the points correspond.
  \cref{lem:E-preserves-symms} below offers a general result of this kind.
  \end{example}

\begin{xca}\label{xca:SubG=MonoG}
Complete the details of the proof of \cref{lem:SubG=MonoG} above using 
\cref{cor:fib-vs-path}\ref{set-fib-vs-path-point},
\cref{lem:sum-of-fibers},
\cref{lem:conistrans}.
\end{xca}

\begin{corollary}\label{lem:setofsubgroups}
Let $G$ be a group. Then $\typemono_G$ is a set since $\Sub_G$ is.
\end{corollary}

The following lemma states that the equivalences in \cref{lem:SubG=MonoG} preserve the subsets of symmetries that are picked out.
\begin{lemma}\label{lem:E-preserves-symms}
  Let $G$ be a group and $g:\USymG$ a symmetry.
  For all $(X,\pt,!) : \Sub_G$ and $(H,i,!) : \typemono_G$
  that correspond via the equivalences in \cref{lem:SubG=MonoG},
  we have $X(g)(\pt) = \pt$ if and only if there exists $h:\USymH$
  such that $g = \USymi(h)$.
\end{lemma}
\begin{proof}
It suffices to prove this for one of the equivalences.
We choose $\inv E$, that is, the equivalence in the statement
of \cref{lem:SubG=MonoG}, so we take 
$(H,i,!) \jdeq ((\sum_{z:\BG}X(z),(\sh_G,\pt)),\fst,!)$.
Now we have to prove: $X(g)(\pt)=\pt$ iff there exists an
$h:(\sh_G,\pt)\eqto(\sh_G,\pt)$ such that $g = \USym\fst(h)$.

If $X(g)(\pt)=\pt$, then we can simply take $h\defeq(g,\pt)$.

For the converse, assume there exists an
$h:(\sh_G,\pt)\eqto(\sh_G,\pt)$ such that $g = \USym\fst(h)$.
Then $h = (g,p)$ for some $p: X(g)(\pt)=\pt$.
\end{proof}

\marginnote{%
  Which of the equivalent sets $\typemono_G$ and $\Sub_G$ is allowed to be called ``the set of subgroups of $G$'' is, of course, a choice.  It could easily have been the other way around and we informally refer to elements in either sets as ``subgroups'' and use the given equivalence $E$ as needed.
}%
\marginnote{%
  An argument for our choice can be
  as follows.  In set-based mathematics one has two options for defining subgroup: either as a certain subset (uniquely given by its characteristic function to $\Prop$) or as an equivalence class of injections (taking care of size issues since the class of monomorphisms will not form a small set).  The former is the usual choice and is the one we model here with $\Sub_G$, whereas the other corresponds to $\typemono_G$.
  % that the identity type in $\typesubgroup_G$ seems more transparent than the one in $\typemono_G$  (``more things are equal'' in $\typemono_G$?), just as  $A\to\Prop$ gives more the intuition of picking out a subset by means of a characteristic function than what you get when considering the equivalent type of injections into $A$.
}


Through the equivalence $E$ we can translate the concepts in
\cref{def:typeofmono} to subgroups in $\Sub_G$. 
First, observe that the underlying groups correspond. 
Then, we say that a subgroup $(X,\pt,!):\Sub_G$ is:
      \begin{enumerate}
      \item \emph{trivial}\index{trivial subgroup} if the underlying subgroup
      $(\sum_{z:\BG}X(z),(\sh_G,\pt))$ is trivial;
      \item \emph{proper}\index{proper subgroup} if $X(\sh_G)$ is not
      contractible.
      \end{enumerate}

      \begin{remark}
      \label{rem:notationsubgroup}
      A note on classical notation is in order.
If $(X,\pt,!)$ is a subgroup corresponding to a monomorphism $(H,i,!)$ into a group $G$, tradition would permit us to relax the burden of notation and we could write ``a subgroup $i:H\subseteq G$'', or, if we didn't need the name of $i:\Hom(H,G)$, simply ``a subgroup $H\subseteq G$'' or ``a subgroup $H$ of $G$''.
    \end{remark}
    

    {\large cursor 1}
    
    \subsection{Move to better place}
    
    % the ``subsymmetries'' formed a very organized structure.
% For each natural number $n$ we obtained a set of subsymmetries in the identity type $\base=\base$, namely the set of all the iterates $(\Sloop^{n})^m$ where $m$ varies over the integers.
% When $n$ was positive this was realized as the $n$-fold \covering of $S^1$, when $n=0$ this was given by the universal \covering.


The other extreme of the idea of a subgroup was exposed in \cref{sec:groupssubperm} in the form of the slogan ``any symmetry is a symmetry in $\Set$''.
By this we meant that, if $G \defequi \Aut_A(a)$ is a group, we produced a monomorphism $\rho_G:\Hom(G,\Aut_{\USymG}(\Set))$,
\ie any symmetry of $a$ is uniquely given by a symmetry (``permutation'') of the set $\USymG\defequi (a\eqto{}a)$.

For many purposes it is useful to define ``subgroups'' slightly differently.
A monomorphism into $G$ is given by a pointed connected groupoid  $\BH=(\BH_\div,\pt_H)$, a function $F:\BH_\div\to\BG_\div$ whose fibers are sets (a \covering) and an identification $p_f:\sh_G\eqto{}F(\sh_H)$.  There is really no need to specify that $\BH_\div$ is a groupoid: if $F:T\to \BG$ is a \covering, then $T$ is automatically a groupoid.

On the other hand,  the type of \coverings over $\BG$ is equivalent to the type of $G$-sets: if $X:\BG\to\Set$ is a $G$-set, then the \covering is given by the first projection $\tilde X\to \BG$ where $\tilde X\defequi\sum_{y:\BG}X(y)$ and the inverse is obtained by considering the fibers of a \covering.  Furthermore, we saw in \cref{lem:conistrans} that $\tilde X$ being connected is equivalent to the condition $\istrans(X)$ of \cref{def:transitiveGset} claiming that the $G$-set $X$ is transitive.

Hence, the type (set, really) $\typemono_G$ of monomorphisms into $G$ is equivalent to the type of pointed connected \coverings over $\BG$, which again is equivalent to the type $\typesubgroup_G$ of transitive $G$-sets $X:\BG\to\Set$ together with a point in $X(\sh_G)$.

and the \emph{type of epimorphisms from $G$}\index{type! of epimorphisms from a groups}\glossary(EpiG){$\protect{\typeepi_G}$}{type of epimorphisms from the group $G$} is
  $$\typeepi_G\defequi\sum_{H:\typegroup}\sum_{f:\Hom(G,G')}\isepi(f).$$
  
  
  \begin{exercise}
      \begin{enumerate}
      \item Show that $i:\Hom(H,G)$ is a monomorphism if and only if $\USymi$ is an injection of sets and that $i$ is proper if and only $Ui$ is not a bijection.
      \item Show that $f:\Hom(G,G')$ is an epimorphism if and only if $\USymf$ is an surjection of sets.
      \item Consider a composite $f=f_0f_2$ of homomorphisms.  Show that $f_0$ is an epimorphism if $f$ is and $f_2$ is a monomorphism if $f$ is.\qedhere
      \end{enumerate}
    \end{exercise}
    
    If $G_1$ and $G_2$ are groups, then the first projection from $G_1\times G_2$ is an epimorphism 

\begin{xca}
  \label{xca:SG2=SG2contractible} \MB{Move or shorten?}
  Show that $\Iso(\SG_2,\SG_2)$ is contractible. Hint: use
  \cref{xca:CG2isSG2}. Here comes another interesting solution.
  Let $f:\Iso(\SG_2,\SG_2)$. Then $\USym f: (\bn2\eqto\bn2)\equivto_*
  (\bn2\eqto\bn2)$. There is exactly one such pointed equivalence,
  namely the identity on $\bn2\eqto\bn2$. Now show that 
  \[
  \sum_{p:A\eqto \Bf(A)}\,\prod_{q: A\eqto\bn{2}} 
  (\Bf^{-1}_\pt\cdot \ap{\Bf}(q)\cdot p) = q
  \]
  is contractible for all $A:\FinSet_2$, which induces an
  identification of type $\id_{\SG_2}\eqto f$.
  
  This argument is a special case of a technique that we
  will develop in \cref{sec:homabsisconcr} and which is called `delooping'.
\end{xca}


\begin{example}
  \label{ex:prodinclisGset}
  We saw in \cref{ex:prodinclismono} that the first inclusion $i_1:G\to G\times G'$ is a monomorphism.
  The corresponding $G\times G'$-set is the composite of the first projection $\mathrm{proj}_1:\BG_\div\times\BG'_\div\to \BG_\div$ followed by the principal $G$-torsor $\princ G:\BG\to\Set$.\MB{Later!}

  More generally, if $i:\Hom(H,G)$ and $f:\Hom(G,H)$, and $fi\eqto{}\id_H$, then $(H,i,!):\typemono_G$, corresponding to the subgroup with $G$-set given by the composite of $\Bf$ with the princial $H$-torsor $\princ H$.
\end{example}



{\large cursor 2}

    % commented out by BID 211117 Some examples and references should be included when the cyclic subgroups are fully developed
    % \subsection{The geometry of subgroups: some small examples}\footnote{this subsection is not touched: it needs attention}
% \label{smallsubgpex}

% As a teaser, and in order to get a geometric feel for the subgroups and their intricate interplay, it can be useful to have some fairly manageable examples to stare at.
% Some of the main tools for analyzing the geometry of subgroups are collected in \cref{sec:fingp} on finite groups, and we hope the reader will be intrigued by our mysterious claims and go on to study \cref{sec:fingp}.
% That said, the examples we'll present are possible to muddle through by hand without any fancy machinery, but brute force is generally not an option and even for the present examples it is not something you want to show publicly.

% When presenting the subgroups of a group $G$, three types are especially revealing: the set of subgroups $\typesubgroup_G(\sh_G)$, the \emph{groupoid of subgroups} $\typesubgroup(G)\defequi\sum_{y:\BG}\typesubgroup_G(y)$ and what we for now call the ``set of normal subgroups'' $\prod_{y:\BG}\typesubgroup_G(y)$.   Our local use of ``normal subgroup'' is equivalent to the official definition to come.

% The first projection $\typesubgroup(G)\to \BG$ is referred to as the \emph{\covering of subgroups}.

% \footnote{Write out and fix the concrete examples (cyclic groups and $\Sigma_3$) commented out}
% % \begin{remark}
% % In  \cref{cha:circle} we studied the subgroups of the group of integers $G\eqto{}\ZZ$ through \coverings over the circle $S^1$ (which we showed was equivalent to $B\ZZ$).
% % We discovered a subgroup $n\ZZ$ for each natural number $n:\NN$ and in the groupoid $\typesubgroup({\ZZ})$ these sit as elements in separate components.  Each of these components are contractible (because addition is commutative: $\ZZ$ is an abelian group).

% % In general, a component $K$ of the groupoid $\sum_{y:\BG}\typesubgroup_G(y)$ of subgroups of a group $G$ may be much more interesting. For one thing the, $K$ can contain many subgroups in the sense that the preimage of the first projection $K\to \BG$ is a set that may have many different elements; each representing a subgroup.  However, this set of subgroup will be a \emph{conjugacy class} of subgroups: the different subgroups are related by the conjugation action of $G$.

% % If $G$ is abelian this action is trivial, and $\sum_{y:\BG}\typesubgroup_G(y)$ consists of contractible components indexed over the subgroups of $G$.  Otherwise different subgroups may live in the same component of the groupoid of subgroups -- we'll see examples in a moment.

% % In addition, the components will not in general be contractible, revealing the symmetries of the subgroups under the conjugation action.
% % \end{remark}


% % \begin{example}
% %   The trivial group only has itself as a subgroup; the groupoid of subgroups and the set of normal subgroups are singletons.
% % \end{example}
% % \begin{example}
% %   The cyclic group $C_p$ of prime order $p$ has only two subgroups, the trivial and the full subgroup itself and both are normal.  In fact, all subgroups of abelian groups are normal.

% % In general, the cyclic group $C_n$ of order $n$ has exactly one subgroup for each divisor $i$ of $n$.
% % \end{example}


% % \begin{example}
% %   The group $C_2\times C_2$ has has no less than five subgroups: the trivial one, three subgroups that as groups (as opposed as \emph{sub}groups) are equivalent to $C_2$ and the full group $C_4$ itself.
% % \end{example}
% % \begin{remark}
% %   The permutation group $\Sigma_3$ has four nontrivial proper subgroups.  Three conjugate subgroups isomorphic as groups to $C_2$ and one normal one which is as a group is isomorphic to $C_3$.  The component containing the copies of $C_2$ is equivalent to a circle.
% % \end{remark}

\section{Invariant maps and orbits}
\label{sec:fixpts-orbits}
We now return to some important constructions involving $G$-sets for a group $G$.
Some of these make equally good sense for \emph{$G$-types} for \aninftygp
$G$, in which case we add a footnote to this effect.
% perhaps with some warnings $\USymG$, $=$, \cref{lem:conistrans}, ...
\begin{definition}
  \label{def:actiontype} Let $G$ be a group and $X : \BG\to\Set$,\footnote{%
  This definition can be generalized to to \inftygps $G$ and $G$-types $X$.}
  then the \emph{action type}\index{action type}
  is the total type of $X$, denoted\footnote{%
    The superscripts and subscripts are decorated with ``$hG$'',
    following a convention in homotopy theory.
    This helps to distinguish them from other uses, such as powers.
    The action type is sometimes denoted $X \dblslash G$.}
\[
  X_{hG} \defeq \sum_{z:\BG} X(z).
\]
By \cref{def:pathover-trp} and \cref{def:pairtopath}, 
the identity type $(z,x)\eqto_{X_{hG}}(w,y)$
is equivalent to the sum type $\sum_{g:z\eqto w} g\cdot x = y$,
and so are their (often used) propositional truncations.
%We use $(z,y):X_{hG}$ sometimes for $z:\BG,\,y:X(z)$.

The type of \emph{invariant maps}\footnote{%
These are dependent functions $f$ and the reason for the new name
in this context is that $f(z) = g \cdot_X f(z)$ for any $z:\BG$
and $g:z\eqto z$. Cf.~\cref{lem:fixed-char}. Note that there need not be invariant maps: $\prod_{z:\Sc}\base\eqto z$ is empty.}
\index{invariant map type} is
\[
  X^{hG} \defeq \prod_{z:\BG} X(z).
\]

The \emph{set of orbits}\footnote{\MB{Where: terminology homotopy orbit spaces?}}
\index{set!of orbits}\index{orbit set} is the subset of $\Sub_{G,X}$ consisting
of all $G$-subtypes $P$ of $X$ such that the underlying $G$-subtype $X_P$
is transitive:\footnote{See \cref{def:transitiveGset}.}
\[
  X / G \defeq \sum_{P:\Sub_{G,X}}\istrans(X_P).\qedhere
\]
\end{definition}

We have seen many instances of action types before:
When $G$-sets are considered as \coverings $f : A \to \BG$,
they are the domains $A$.
Recall for example~\cref{fig:two-comp-S1-cover},
showing an action of $\ZZ$ on $\set{1,2,3,4,5}$ with no invariant maps
and an action type equivalent to a sum of two circles.
In~\cref{fig:ZZ-set-orbits}, we show a similar $\ZZ$-set,
with underlying set $\set{0,1,2,3,4,5}$, three orbits,
and $5$ corresponding to the only invariant map.\footnote{%
Sending $\base$ to $5$ and $\Sloop$ to $\refl{5}$.}

\begin{marginfigure}
  \begin{tikzpicture}[scale=.15]
    \node (Sc) at (0,-5) {$\B\ZZ$};
    \node[dot,label=left:$5$] (five)   at (-10,30) {};
    \node[dot,label=left:$4$] (four)   at (-10,22) {};
    \node[dot,label=left:$3$] (three)  at (-10,18) {};
    \node[dot]                (base)   at (-10,-5) {};
    \node[label=left:$\Sloop$] (Sloop) at (10,-5) {};

    \pgfmathsetmacro\cc{.55228475}% = 4/3*tan(pi/8)
    \pgfmathsetmacro\cy{2*\cc}%
    \pgfmathsetmacro\cx{10*\cc}%
    \pgfmathsetmacro\intx{3.5}%
    \pgfmathsetmacro\inty{1.5}%
    \pgfmathsetmacro\ay{.35165954}%

    \draw (-10,18) .. controls (-10,18 - \cy + \ay) and (-\cx,17 - \ay)
    .. (0,17) .. controls (\cx,17 + \ay) and (10,20 - \cy - \ay) .. (10,20)
    .. controls (10,20 + \cy + \ay) and (\cx,23 - \ay)
    .. (0,23) .. controls (-\cx,23 + \ay) and (-10,22 + \cy - \ay)
    .. (-10,22) .. controls (-10,22 - \cy + \ay) and (-\cx,21 - \ay)
    .. (0,21) .. controls (\cx,21 + \ay) and (10,24 - \cy - \ay)
    .. (10,24)
    .. controls (10,24 + \cc) and (\intx + \cx, 24 + \inty + \ay)
    .. (\intx,24 + \inty) .. controls (\intx - \cx,24 + \inty - \ay)
    and (-\intx + \cx,20 + \ay)
    .. (-\intx,18 + \inty) .. controls (-\intx - \cx,18 + \inty - \ay)
    and (-10,18 + \cc) .. (-10,18);
    \draw[casblue] (-10,2) .. controls (-10,2 - \cy + \ay) and (-\cx,1 - \ay)
    .. (0,1) .. controls (\cx,1 + \ay) and (10,4 - \cy - \ay)
    .. (10,4)
    \foreach \y in {4,8} {
      .. controls (10,\y + \cy + \ay) and (\cx,3 + \y - \ay)
      .. (0,3 + \y) .. controls (-\cx,3 + \y + \ay) and (-10,2 + \y + \cy - \ay)
      .. (-10,2 + \y) .. controls (-10,2 + \y - \cy + \ay) and (-\cx,1 + \y - \ay)
      .. (0,1 + \y) .. controls (\cx,1 + \y + \ay) and (10,4 + \y - \cy - \ay)
      .. (10,4 + \y) }
    .. controls (10,12 + \cc) and (\intx + \cx, 12 + \inty + \ay)
    .. (\intx,12 + \inty) .. controls (\intx - \cx,12 + \inty - \ay)
    and (-\intx + \cx,4 + \ay)
    .. (-\intx,2 + \inty) .. controls (-\intx - \cx,2 + \inty - \ay)
    and (-10,2 + \cc) .. (-10,2);
    \draw (10,-5) .. controls ++(0,\cy) and ++(\cx,0)
    .. (0,-3) .. controls ++(-\cx,0) and ++(0,\cy)
    .. (-10,-5) .. controls ++(0,-\cy) and ++(-\cx,0)
    .. (0,-7) .. controls ++(\cx,0) and ++(0,-\cy) .. (10,-5);

    \draw (10,30) .. controls ++(0,\cy) and ++(\cx,0)
    .. (0,32) .. controls ++(-\cx,0) and ++(0,\cy)
    .. (-10,30) .. controls ++(0,-\cy) and ++(-\cx,0)
    .. (0,28) .. controls ++(\cx,0) and ++(0,-\cy) .. (10,30);

    \node[dot,label=left:$2$,casred] (two)   at (-10,10) {};
    \node[dot,label=left:$1$,casred] (one)   at (-10, 6) {};
    \node[dot,label=left:$0$,casred] (zero)  at (-10, 2) {};
  \end{tikzpicture}
  \caption{A $\ZZ$-set with three orbits and one invariant map.}
  \label{fig:ZZ-set-orbits}
\end{marginfigure}

In~\cref{fig:ZZ-set-orbits} we have highlighted a single component of
the action type in blue (\ie corresponding to an element of the set of orbits),
and we see that it contains a subset of the underlying set,
the three red elements $\set{0,1,2}$.
Such a set is what is traditionally called an orbit.
This connection is emphasized in the following result,
and in particular in \cref{cor:orbit-equiv}.

Recall from \cref{def:actiontype} the equivalence of
$\Trunc{(z,x)\eqto(w,y)}$ and $\exists_{g:z\eqto w}(g\cdot x = y)$.

\begin{lemma}\label{lem:[]0-surj-on-orbits}\MB{New:}
  Let $G$ be a group and $X : \BG\to\Set$.\footnote{%
  This lemma can be generalized to \inftygps $G$ and $G$-types $X$.}
  Define the map $[\blank]_0$ from the action type of $X$ to the type of
  $G$-subsets of $X$, by $[u]_0(v) \defeq \Trunc{u\eqto v}$
  for all $v:X_{hG}$.\footnote{\MB{New:}%
  Here we use the uncurried form of $\Sub_G(X)$ as
  explained in \cref{ft:SubTotX}.}
  Then the image of $[\blank]_0$ is the set of orbits.
  Moreover, we have a (unique) identification of
  $(X/G,[\blank]_0)$ and $(\setTrunc{X_{hG}},\settrunc{\blank})$
  in the type $\sum_{S:\Set}(X_{hG}\to S)$.
\end{lemma}
\begin{proof}\MB{New:}
  Clearly, $[u]_0$ is a $G$-subset of $X$. 
  By \cref{def:Gsubset}, the underlying $G$-type $X_{[u]_0}$ is
  $(z\mapsto \sum_{x:X(z)}\Trunc{u\eqto(z,x)})$, which we
  must show to be transitive. This follows easily from the properties
  of $\Trunc{u\eqto(z,x)}$. We conclude that
  $[\blank]_0$ is a well defined map from $X_{hG}$ to $X/G$.
  
  For proving surjectivity of $[\blank]_0$ on the set of orbits, 
  consider an orbit $O:X/G$, \ie
  $O$ is a $G$-subtype of $X$ such that $X_O$ is transitive. 
  We have to show that there exists a $u:X_{hG}$ such that $O=[u]_0$.
  By the connectivity of $\BG$ it suffices to show 
  $O(\sh_G)=_{X(\sh_G)\to\Prop}[u]_0(\sh_G)$ for some $u$.
  Transitivity of $X_O$ means that there exists an $x:X(\sh_G)$
  such that $O(\sh_G,x)$ and for all $y:X(\sh_G)$ such that
  $O(\sh_G,y)$ there
  exists a $g:\USymG$ such that $g\cdot x = y$, \ie $[(\sh_G,x)]_0(\sh_G,y)$.
  So we take $u\defeq(\sh_G,x)$ and have to show $O(\sh_G,y)$ if and only if
  $[u]_0(\sh_G,y)$, for all $y:X(\sh_G)$. But this follows directly from
  the observation made just above the lemma
  (see also \cref{rem:equivalents-of-[x]=[y]} below).
  
  The last part of the lemma follows from \cref{rem:set-trunc-as-quotient}. 
\end{proof}

\begin{corollary}\label{cor:orbit-equiv}
Define the map $[\blank] : X(\sh_G) \to X/G$ by $[x]\defeq[(\sh_G,x)]_0$.
Then $[\blank]$ is surjective and induces
by \cref{xca:map-induces-quotient}\ref{it:surj-ind-quot=codomain}
an equivalence between the induced quotient of $X(\sh_G)$ and $X/G$.
Moreover, $[x] = [y]$ is equivalent to $\exists_{g:\USymG}(g\cdot x = y)$.
\end{corollary}
\begin{proof}\MB{New:}
In the proof of surjectivity in \cref{lem:[]0-surj-on-orbits} we used
$u\defeq(\sh_G,x)$ to get $O=[u]_0$, so $[\blank]$ is surjective. 
The last statement follows since
both propositions are equivalent to $\Trunc{(\sh_G,x)\eqto(\sh_G,y)}$.
\end{proof}

\begin{xca}\label{xca:transX-just1orbit}
Show: $X/G$ is contractible if and only if $X$ is transitive.
%That $X/G$ is contractible encodes that there is just one orbit.
%Hint: use \cref{lem:conistrans}.
\end{xca}


\begin{remark}\label{rem:equivalents-of-[x]=[y]}\MB{New:}
Given a group $G$, a $G$-set $X$ and $x,y:X(\sh_G)$,
the following propositions are all equivalent and we may pass from
one to another without mention:
$[x]=_{X/G}[y]$;\quad 
$[x](\sh_G)=_{X(\sh_G)\to\Prop}[y](\sh_G)$;\quad
%$G\cdot x =_{\Sub_{X(\sh_G)}} G\cdot y$;\quad
$\exists_{g:\USymG}(g\cdot x = y)$;\quad
$\Trunc{(\sh_G,x)\eqto(\sh_G,y)}$;\quad 
$[x](\sh_G,y)$;\quad
$[y](\sh_G,x)$. As functions of $x$ and $y$ they all define the equivalence
relation on $X(\sh_G)$ induced be the surjection $[\blank]$. 
\end{remark}

Thus, both the underlying set $X(\sh_G)$ and the action type
$X_{hG}$ have equivalence relations (induced by the surjections $[\blank]$
and $[\blank]_0$, respectively) with quotient set $X/G$.\footnote{%
  \label{ft:orbit-surj}
  This also justifies the notation $X/G$.
  We have a diagram of surjective maps:
  \[
    \begin{tikzcd}[ampersand replacement=\&]
      X(\sh_G) \ar[rr,"{x\mapsto(\sh_G,x)}"]\ar[dr,"{[\blank]}"']
      \& \& X_{hG}\ar[dl,"{[\blank]_0}"] \\
      \& X/G \&
    \end{tikzcd}
  \]}
  We can write $X(\sh_G)$ and $X_{hG}$ as sums of the respective fibers,
  which we will elaborate in the next paragraphs.

Let $O:X/G$ be an orbit and consider 
$\inv{[O]_0} \jdeq \sum_{u:X_{hG}}(O=[u]_0)$.
Recall that $O$ can be uncurried as a predicate on $X_{hG}$,
and that its underlying $G$-set $X_O\jdeq(z:\BG \mapsto \sum_{y:X(z)}O(z,y))$
is transitive. Consequently, for all $u:X_{hG}$,
$O(u)$ holds if and only if $O=[u]_0$.
Therefore, the fiber $\inv{[O]_0}$ is equivalent to the action type
$(X_O)_{hG}\jdeq\sum_{z:\BG} X_O(z)$ by uncurrying.

After the previous paragraph, the elaboration of 
$\inv{[O]} \jdeq \sum_{x:X(\sh_G)}(O=[x])$ is easy.
Recall that $[x]\jdeq[(\sh_G,x)]_0$, so that the fiber $\inv{[O]}$
is equivalent to the underlying set of $X_O$, \ie
$X_O(\sh_G)\jdeq\sum_{x:X(\sh_G)} O(\sh_G,x)$
via identity on first components.
We depict the situation in the diagram%
\footnote{\label{ft:orbit-fibs}%
Along the horizontal arrow, $(O,x)$ maps to $(O,(\sh_G,x))$, for $x:X_O(\sh_G)$.
  \[
    \begin{tikzcd}[ampersand replacement=\&,column sep=tiny]
      \displaystyle\sum_{O:X/G} X_O(\sh_G) \ar[rr]\ar[dr,"\fst"']
      \& \& \displaystyle\sum_{O:X/G} (X_O)_{hG}\ar[dl,"\fst"] \\
      \& X/G \&
    \end{tikzcd}
  \]
}
in the margin. Note how the role of $X$ in \cref{ft:orbit-surj} 
is taken over by $X_O$. 

\begin{definition}\label{def:orbit-stabilizer}
  Let $G$ be a group, $X : \BG \to \Set$ a $G$-set, 
  and $x : X(\sh_G)$ an element.\footnote{%
  This definition can be generalized to to \inftygps $G$ and $G$-types $X$.}
  \begin{enumerate}
  \item Define the group $G_x \defeq \Aut_{X_{hG}}(\sh_G,x)$,
  with (curried) classifying type
  $(\sum_{z:\BG}\sum_{y:X(z)}\Trunc{(\sh_G,x) \eqto (z,y)},(\sh_G,x,!))$.
  Clearly, $\fst:\BG_x \to \BG$ is a set bundle:
  each fiber at $z:\BG$ is a subset of $X(z)$.
  Hence $(G_x,\fst,!):\typemono_G$ is monomorphism into $G$.
  We call the subgroup $G_x$ of $G$ the 
  \emph{stabilizer (sub)group}\index{stabilizer}%
    \index{group!stabilizer} at $x$. The inclusion $\fst$ of $\BG_x$ in
    $\BG$ classifies a monomorphism denoted by $i_x : \Hom(G_x,G)$.
  \item Define $G\cdot x \defeq \setof{y : X(\sh_G)}{[x] =_{X/G} [y]}$
    to be the \emph{underlying set of the orbit through $x$}.\footnote{%
    This is short for the underlying set of the underlying $G$-set of the
    orbit $[x]$ of $X$.}
    \qedhere
  \end{enumerate}
\end{definition}

\begin{remark}\label{rem:orbit-fibs}
In the above definition, the underlying $G$-set $X_{[x]} \jdeq
(z:\BG)\mapsto\sum_{y:X(\sh_G)}\Trunc{(\sh_G,x)\eqto(z,y)}$
of the orbit $[x]$ plays an important double role: On one hand its
action type $(X_{[x]})_{hG}$,
%$\sum_{z:\BG}\sum_{y:X(z)}\Trunc{(\sh_G,x)\eqto(z,y)}$,
pointed at $(\sh_G,x)$, is the classifying type of the stabilizer
group $G_x$. On the other hand it is a transitive $G$-set whose
underlying set $\sum_{y:X(\sh_G)}\Trunc{(\sh_G,x)\eqto(\sh_G,y)}$
is the underlying set of the orbit $[x]$. Thus,
for $O\jdeq[x]$, we have easy identifications of $G\cdot x$ and $X_O(\sh_G)$,
as well as of $\BG_x$ and $(X_O)_{hG}$, using $[x]\jdeq[(\sh_G,x)]_0$.
Applying the maps in \cref{ft:orbit-fibs} in this particular case, 
we obtain \cref{fig:fibs-at-[x]}.
\begin{marginfigure}
  \[\footnotesize %MB20250324: footnotesize should be default
    \begin{tikzcd}[ampersand replacement=\&,column sep=tiny]
      ([x], y) \ar[rr,mapsto]\ar[dr,mapsto,"\fst"']
      \& \& ([(\sh_G,x)]_0,(\sh_g,y))\ar[dl,mapsto,"\fst"] \\
      \& {[x]} \&
    \end{tikzcd}
  \]
  \caption{\label{fig:fibs-at-[x]}Along the horizontal arrow,
  the second component $y:G\cdot x$ is mapped to $(\sh_g,y):\BG_x$.}
\end{marginfigure}

Note furthermore that the base point of $\BG_x$ depends on the choice of $x$,
but the underlying type $(\BG_x)_\div$, being a connected component,
only depends on $[x]:X/G$.
\end{remark}

\begin{xca}\label{xca:[x]=[y]-implies-||Gx=Gy||}
Let $G$ be a group and $X: \BG\to\Set$ a $G$-set. Show:
if $[x]=[y]$, then $\Trunc{G_x\eqto G_y}$, for any $x,y:X(\sh_G)$.
%Solution: conclusion is prop, so we may assume we have a $g:\USymG$
%such that $g\cdot x = y$. Hence we have $p: (\sh_G,x)\eqto(\sh_G,y)$,
%so $\BG_x\eqto\BG_y$ by application on this path $p$.
\end{xca}

\begin{remark}\label{rem:subgrp-is-stabsubgr}
  In fact every subgroup of $G$ is a stabilizer subgroup.
  We can equivalently define the stabilizer subgroup of $x$ 
  by an element of $\Sub_G$, namely
  the transitive $G$-set $X_{[x]}$, pointed by
  $x$ as element of the subset $X_{[x]}(\sh_G)$ of $X(\sh_G)$.
  If $X$ is transitive, then all orbits are equal (\cref{xca:transX-just1orbit})
  and the stabilizer subgroup of $x$ simplifies to $(X,x,!):\Sub_G$,
  a general form defining a subgroup of $G$.
\end{remark}

The following lemma states that the orbits of a $G$-set $X$
sum up to its underlying set, with the sum taken over the set
of orbits $X/G$.

\begin{lemma}
  \label{lem:splitting into orbits}
  The inclusions of the orbits form an equivalence
\[
  (O,x,!)\mapsto x \quad:\quad
  \bigl(\sum_{O:X/G} \inv{[O]} \bigr) \we X(\sh_G).
\]
\end{lemma}
\begin{proof}
Recall that $\inv{[O]}\jdeq \sum_{x:X(\sh_G)}(O=[x])$, 
and then abstract
away the $O$ using \cref{lem:contract-away}.\footnote{%
In fact, for every set $A$ and every equivalence relation on $A$,
the equivalence classes sum up to $A$.}
\end{proof}

There are two possible extreme cases for $G_x$ that are important:
\begin{definition}\label{def:fixed-free}
  Let $G$ be a group, $X$ a $G$-set and $x:X(\sh_G)$ an element of the underlying set.\footnote{%
  This definition can be generalized to to \inftygps $G$ and $G$-types $X$.}
  We say that
  \begin{enumerate}
  \item $x$ is \emph{fixed}\index{fixed}
    if $i_x$ is an isomorphism (so $G_x$ is all of $G$), and
  \item $x$ is \emph{free}\index{free}\index{action!free}
    if $G_x$ is trivial.
  \end{enumerate}
  We say that $X$ itself is \emph{free} if each $x:X(\sh_G)$ is free.
\end{definition}

\begin{lemma}\label{lem:fixed-char}
  Given a group $G$ and a $G$-set $X$, an element $x:X(\sh_G)$ is
 fixed if and only if the orbit $G\cdot x$ is contractible,
  \ie $x = g\cdot x$ for all $g:\USymG$.\footnote{%
  This lemma can be generalized to to \inftygps $G$ and $G$-types $X$.
  In that case $\USymG\defeq\loops\BG$ is the underlying \emph{type} of $G$.}
\end{lemma}
\begin{proof}
  The orbit $G\cdot x$ of $x$ is the fiber of $\Bi_x : \BG_x \ptdto \BG$
  at $\sh_G$. Since $\BG$ is connected,
  this is contractible if and only if all fibers of $\Bi_x$ are contractible,
  \ie $\Bi_x$ is an equivalence, which in turn is equivalent to $i_x$
  being an isomorphism.
\end{proof}

\begin{xca}\label{xca:X_hG-set-iff-Xfree}\MB{New:}
Let $G$ be a \aninftygp and $X$ a $G$-type. Then the
action type $X_{hG}$ is a set if and only if $X$ is free.
\end{xca}
%solution\begin{proof}
%Any type $T$ is a set iff all identity types $t\eqto_T t$ are contractible.
%Since $\BG$ is connected we get that $X_{hG}$ is a set iff each
%identity type $(\sh_G,x)\eqto (\sh_G,x)$ is contractible, \ie
%each stabilizer group $G_x$ is trivial.\end{proof}

When $X : \BG \to \Set$ is a $G$-set for an ordinary group $G$, the subset
\[
  \setof{x : X(\sh_G)}{\text{$x$ is fixed}}
\]
is closely related to the type $X^{hG}$ of invariant maps.
If we evaluate an invariant map $f : \prod_{z:\BG}X(z)$ at $\sh_G$
we do indeed land in this subset:
Letting $x\defeq f(\sh_G)$,
and taking the dependent action on paths,
$\apd{f}(g) : \pathover x X g x$,
we can use~\cref{def:pathover-trp} to conclude
$\trp[X] g(x)\jdeq g\cdot x = x$, for all $g:\USymG$.
The following lemma states that, conversely, each fixed $x$ uniquely
determines an invariant map.

\begin{lemma}\label{lem:fixpts-are-fixed}
  Let $G$ be a group and $X$ a $G$-set,\footnote{\MB{%
  We use the connectivity of $\BG$ and that $X(\sh_G)$ is a set.}}
  with $X^{hG} \jdeq \prod_{z:\BG}X(z)$
  the set of invariant maps. 
  Evaluation $\ev\defeq(f:X^{hG})\mapsto f(\sh_G)$ at $\sh_G$ gives
  \begin{enumerate}
  \item\label{it:ev-is-inj} 
  an injection of type $\bigl(\prod_{z:\BG}X(z)\bigr) \to X(\sh_G)$, which is
  \item\label{it:ev-is-eq-on-inv} an equivalence of type 
  $\bigl(\prod_{z:\BG}X(z)\bigr) \equivto
  \setof{x : X(\sh_G)}{\text{$x$ is fixed}}.$
  \end{enumerate}
\end{lemma}
\begin{proof}
  Let $x : X(\sh_G)$. We prove that the fiber of $\ev$ at $x$,
  \[
    \inv\ev(x) \defeq \sum_{f : \prod_{z:\BG}X(z)} x=f(\sh_G),
  \]
  is a proposition. Let $(f,!),(g,!) : \inv\ev(x)$. 
  Then it suffices to prove $f=g$, which follows by extensionality
  from $f(\sh_G)=x=g(\sh_G)$ since $\BG$ is connected. This proves
  \ref{it:ev-is-inj}.
  
  For \ref{it:ev-is-eq-on-inv}, assume that
  $x : X(\sh_G)$ is fixed, so $i_x\jdeq\fst:\BG_x \to \BG$
  is an equivalence. This means that $\inv\fst(z)$ is contractible
  for all $z:\BG$. Spelling out $\inv\fst(z)$, using \cref{lem:contract-away},
  identifies each fiber $\inv\fst(z)$ with
  $\sum_{y:X(z)}\Trunc{(\sh_G,x)\eqto(z,y)}$.
  Projecting on the first component of each center of contraction
  gives a invariant map $f$ such that $\Trunc{(\sh_G,x)\eqto(\sh_G,f(\sh_G))}$,
  from which the proposition $x=f(\sh_G)$ follows. This proves that
  $\ev$ is surjective, so an equivalence by \ref{it:ev-is-inj}.
\end{proof}

\begin{xca}\label{xca:Gset-A->B}
Let $G$ be the group $\SG_2\times\SG_2$ and $X$ the $G$-set mapping
any pair $(A,B)$ of $2$-element sets to the set $A\to B$.
Elaborate the action of $G$ on $X(\sh_G)$ and determine the
set of orbits and the set of invariant maps. You can
do the same exercise for the following easier cases first:
the $G$-set that is constant $\bn2\times\bn2$, and
the $\SG_2$-sets $X(\blank,\bn2)$ and $X(\bn2,\blank)$.
\end{xca}

\subsection{The Orbit-stabilizer theorem and Lagrange's theorem}

Consider a group $G$, a $G$-set $X$ and an element $x:X(\sh_G)$,
and recall \cref{def:orbit-stabilizer}.
The classifying type of the stabilizer group $G_x$
is the component of $X_{hG}\jdeq\sum_{z:\BG} X(z)$ 
pointed by the shape $(\sh_G,x)$.
The first projection of a symmetry of $(\sh_G,x)$ is a symmetry of 
$\sh_G$, and the second projection is a proof of a proposition.
This suggest the following simple way for $G_x$ to act on the
symmetries of $\sh_G$, by just ignoring the second projection:


\begin{definition}\label{def:Gx-action-on-G}
Let $G$ be a group, $X$ a $G$-set and $x:X(\sh_G)$ an element of 
the underlying set. Recall 
$\BG_x \jdeq \sum_{(z,y):X_{hG}}\Trunc{(\sh_G,x)\eqto(z,y)}$,
the classifying type of the stabilizer group $G_x$. 
Define the $G_x$-set $\tilde G_x : \BG_x \to \UU$ by setting,\footnote{%
In short: $\tilde G_x \defeq \princ G \circ \Bi_x$.\label{ft:restriction}}
for all $(z,y):X_{hG}$ in the same component as $(\sh_G,x)$, 
\[
\tilde G_x(z,y,!)\defeq(\sh_G\eqto z).\qedhere
\]

\end{definition} 

The underlying set of $\tilde G_x$ is $\USymG$. The group action of
$\tilde G_x$ is explored in the following exercise.

\begin{xca}\label{xca:Gx-action-on-G}
Let $s:(\sh_G,x,!)\eqto(z,y,!)$ be a path in $\BG_x$ with first component
$s_1$, and let $g:\USymG$. Show that $s\cdot_{\tilde G_x} g = s_1 g$, \ie
the group action of $\tilde G_x$ is path composition. 
\end{xca}

The action type of $\tilde G_x$ can be identified with the 
underlying set of the orbit through $x$ under $X$. This is 
achieved by a chain of easy equivalences, spelled
out in the following construction.

\begin{construction}[Orbit-stabilizer theorem]
  \label{con:orbit-stabilizer}
  Let $G$ be a group, $X$ a $G$-set $X$,
  $x : X(\sh_G)$ an element of the underlying set of $X$.\footnote{%
  This construction can be generalized to \inftygps $G$ and $G$-types $X$.}
  Recall the $G_x$-set $\tilde G_x$ from \cref{def:Gx-action-on-G}.
  Then we have an equivalence from the action type $(\tilde G_x)_{hG_x}$
  to the underlying set $(G\cdot_X x)$ of the orbit through $x$.
\end{construction}
\begin{implementation}{con:orbit-stabilizer}
  The desired equivalence is the composition of elementary equivalences
  for sums and products, followed by contracting away the variable $z$:
\footnote{\label{ft:action-type-tildeGx-set} 
Note that \cref{xca:Gx-action-on-G} already
implies that $(\tilde G_x)_{hG_x}$ is a set:
given $(\sh_G,x,!,g)$ and $(z,y,!,g')$, there can be at most
one $s : \sh_G \eqto z$ such that $s\cdot_{\tilde G_x} g = sg = g'$. 
}
  \begin{align*}
    (\tilde G_x)_{hG_x}
    &\jdeq \sum_{u : \BG_x}\tilde G_x(u) \\
    &\equivto \sum_{z:\BG}\sum_{y:X(z)}
    \Trunc{(\sh_G,x) \eqto (z,y)}\times (\sh_G \eqto z) \\
    &\equivto \sum_{y:X(\sh_G)}[x] =_{X/G} [y]\quad\jdeq\quad(G\cdot_X x).\qedhere
  \end{align*}
\end{implementation}

The above theorem has some interesting consequences.
One is that $(\tilde G_x)_{hG_x}$ is a set, 
and hence all its components are contractible.
This means that for any $g:\USymG$, the stabilizer group $(G_x)_g$
is trivial. So:

\begin{corollary}\label{cor:action-subgrp-free}
\MB{New:} The $G_x$-set $\tilde G_x$ is free.
\end{corollary}

\cref{lem:free-pt-char} below, applied to $G_x$ and $\tilde G_x$,
will then allow us to conclude that the underlying set 
$(G_x \cdot_{\tilde G_x} g)$ of the orbit through $g$ is equivalent to 
$\USymG_x$, for any $g:\USymG$.

\begin{lemma}\label{lem:free-pt-char}
  Let $G$ be a group and $X$ a $G$-set. Then we have for all $x:X(\sh_G)$
  that $x$ is free if and only if the (surjective) map
  $(\blank \cdot x) : \USymG \to (G\cdot x)$ is injective
  (and hence a bijection).
\end{lemma}
\begin{proof}
  Consider two elements of the orbit, say $g\cdot x,g'\cdot x$ for $g,g':\USymG$.
  We have $g\cdot x=g' \cdot x$ if and only if $x = \inv{g} g'\cdot x$
  if and only if $\inv{g} g'$ lies in $\USymG_x$.
  Hence the map $(\blank \cdot x)$ is injective iff $\USymG_x$ is contractible.
  Now use \cref{xca:connected-trivia} yielding that that 
  $G_x$ is trivial iff $\USymG_x$ is contractible.
\end{proof}

In the case of a subgroup of $G$, we have the following result.

\begin{construction}\label{con:preLagrange}\MB{New:}
  Let $G$ be a group and let $(X,x,!):\Sub_G$ be a subgroup of $G$ as defined
  in \cref{def:set-of-subgroups}. Then we have an equivalence
  $[\blank]_x$ from the underlying set $X(\sh_G)$ of $X$ to 
  $\tilde G_x /G_x$, the set of orbits of $\tilde G_x$. 
\end{construction}
\begin{implementation}{con:preLagrange}
The function $[\blank]_x$ is the composition of three equivalences.
Since $X$ is transitive, $\fst: (G\cdot_X x)\to X(\sh_G)$ is an equivalence.
The orbit-stabilizer \cref{con:orbit-stabilizer} gives us an equivalence $o$
from $(G\cdot_X x)$ to $(\tilde G_x)_{hG_x}$. Since the latter type is a set,
the function $[\blank]_0$ from \cref{lem:[]0-surj-on-orbits} 
is an equivalence from $(\tilde G_x)_{hG_x}$ to $\tilde G_x /G_x$.
Now define $[x']_x \defeq [o(\inv\fst(x'))]_0$ for any $x':X(\sh_G)$.
\end{implementation}

\begin{xca}\label{xca:preLagrange}\MB{New:}
Show that $[\blank]_x$ in the above construction can be elaborated as
\[
[x']_x(z,y,!,h) \defeq \bigl((\sh_G,x',!,\refl{\sh_G})=(z,y,!,h)\bigr):\Prop
\]
for all $x':X(\sh_G)$, $(z,y,!):\BG_x$ and $h:\sh_G\eqto z$.
\end{xca}

The following theorem summarizes several results in this section.

\begin{theorem}\label{thm:summary5.4}
  Let $G$ be a group and $(X,x,!):\Sub_G$ giving a 
  stabilizer subgroup $G_x$ of $G$. Then:
  \begin{enumerate}
  \item The $G_x$-set $\tilde G_x$ from \cref{def:Gx-action-on-G}
  has underlying set $\USymG$ and is free; 
  \item
  The function $[\blank]_x : X(\sh_G)\to\tilde G_x /G_x$
  from \cref{con:preLagrange} is an equivalence;
  \item\label{it:lagrange}
  For all $y:X(\sh_G)$, the underlying set of the orbit $[y]_x$ 
  is \emph{merely} equivalent to $\USymG_x$.
  \end{enumerate}
\end{theorem}

\begin{proof}
Only point \ref{it:lagrange} above has not been proved before.
Using \cref{xca:preLagrange}, the underlying set of $[y]_x$ is 
\[
\sum_{g:\USymG}\bigl((\sh_G,y,!,\refl{\sh_G})=(\sh_G,x,!,g)\bigr),
\]
which can be simplified using \cref{ft:action-type-tildeGx-set} to
$\sum_{g:\USymG} g\cdot_X y = x$. 
Since mere equivalence is a proposition,
we can use the transitivity of $X$ to obtain a $g_0:\USymG_x$ such that
$\fst(g_0)\cdot_X y = x$. Precomposition with $\fst(g_0)$ then yields
an equivalence from $\USymG_x \jdeq \sum_{g:\USymG} g\cdot_X x = x$
to $\sum_{g:\USymG} g\cdot_X y = x$.
\end{proof}

The reader may notice that the above theorem contains some of
the ingredients of the traditional formulation of Lagrange's Theorem:
the (abstract) group $G$, the (abstract) subgroup $G_x$
and the set of orbits (cosets),\footnote{%
\MB{Explain better when cosets are defined!}} 
whose underlying sets are all merely equivalent to $\USymG_x$
(in bijective correspondence with the carrier set of the subgroup).
The following construction brings us one step closer to the
traditional formulation.

\begin{construction}[\MB{Lagrange's Construction}]\label{con:lagrange}
Let $G$ be a group. For all subgroups $H\defeq(X,x,!):\Sub_G$,
we have a function $L_H$ of type
\[
\bigl(\prod_{y:X(\sh_G)}\sum_{g:\USymG} g\cdot_X y = x \,\bigr) \to
\bigl(\USymG \eqto (X(\sh_G) \times \USymG_x)\bigr).
\]
\end{construction}

\begin{implementation}{con:lagrange}
For the $G_x$-set ${\tilde G_x}$, with underlying set $\USymG$, 
the map $[\blank]$ from \cref{lem:[]0-surj-on-orbits} has type 
$\USymG \to {\tilde G_x}/G_x$, and
can be elaborated for all $g:\USymG$ and 
$(z,y,!):\BG_x$, $h:\sh_G\eqto z$ as
\[
[g](z,y,!,h) \jdeq \bigl((\sh_G,x,!,g)=(z,y,!,h)\bigr) : \Prop.
\]
Using the equivalences of \cref{lem:sum-of-fibers} and 
\cref{con:preLagrange}, we get
\[
\USymG
\equivto \sum_{O:{\tilde G_x}/G_x} \inv{[O]}
\equivto \sum_{y:X(\sh_G)} \inv{[[y]_x]} 
\jdeq \sum_{y:X(\sh_G)}\sum_{g:\USymG} [y]_x = [g] .
\]
By extensionality, and using \cref{xca:preLagrange}, 
we have $[y]_x = [g]$ if and only if 
$(\sh_G,y,!,\refl{\sh_G})=(\sh_G,x,!,g)$, which in turn 
is logically equivalent to $g\cdot_X y = x$.

So we get an equivalence from $\USymG$ to 
$\sum_{y:X(\sh_G)}\sum_{g:\USymG} (g\cdot_X y = x)$.
In order to complete the implementation, let a function $f$ of 
type $\prod_{y:X(\sh_G)}\sum_{g:\USymG} g\cdot_X y = x$ be given. 
For any $y:X(\sh_G)$, the type $\sum_{g:\USymG} g\cdot_X y = x$ has been
shown equivalent to $\USymG_x$ in the proof of point \ref{it:lagrange} of \cref{thm:summary5.4},
provided we have a $g_0:\USymG$ such that $g_0\cdot_X y = x$.
Hence we can just take $g_0 \jdeq f(y)$, and get in total an equivalence  
between $\USymG$ and $X(\sh_G) \times \USymG_x$, and we define $L_H(f)$
to be that equivalence.
\end{implementation}

A minor modification of the above implementation, gives the following variation of $L_H$, 
which is sometimes more convenient, \eg in the proof of \cref{lem:burnside}.

\begin{corollary}\label{cor:lagrange-dep-sum}
Conjugation $g\mapsto\inv f(y) g f(y)$ is an equivalence
from $\USymG_x$ to $\USymG_y$, for every $y:X(\sh_G)$.
Thus we get an equivalence $L'_H(f)$ between $\USymG$ and
$\sum_{y:X(\sh_G)}\USymG_y$.
\end{corollary}

\begin{xca}\label{xca:lagrange}
The goal of this exercise is to state and prove the traditional
formulation of  Lagrange's Theorem. 
Let $G$ be a finite group and $(X,x,!):\Sub_G$ giving a 
stabilizer subgroup $G_x$ of $G$.
Assume that $X$ is a finite $G$-set.
Show that $\Card(G) = \Card(X) \times \Card(G_x)$.
\end{xca}

\begin{xca}\label{xca:lagrange-Z-action-Rm}
The goal of this exercise is to illustrate that \cref{con:lagrange}
also can be applied to infinite groups. 
Recall the group of integers $\ZZ\jdeq\mkgroup{(\Sc,\base)}$ and the $\ZZ$-set
$R_m: \Sc\to\Set$ from \cref{def:RmtoS1}, defined by $R_m(\base) \defeq \bn m$
and $R_m(\Sloop) \defis \zs$, for $m>0$. Give an identification of type
$\USym\ZZ \eqto (\bn m \times \USym\ZZ_0)$ using \cref{con:lagrange}.
\end{xca}

\begin{xca}\label{xca:lagrange-if-subgr-not-normal}\MB{TBD: }
The goal of this exercise is to illustrate that \cref{con:lagrange}
also can be applied to infinite groups and a subgroup that is not normal.
Recall \cref{fig:not-normal} ...
\end{xca}






\section{The classifying type is the type of torsors}
\label{sec:torsors}
Recall the definition of the principal $G$-torsor 
$\princ G \jdeq (\sh_G \eqto \blank)$ from \cref{def:principaltorsor}.
In this section we elaborate the concept of torsor and give one example
of its use.
In \cref{sec:Gsetforabstract} we'll use torsors to prove that the type of groups and the type of abstract groups are equivalent by classifying abstract groups via their pointed connected groupoid of torsors.  To see how this might work it is good to start with the case of a (concrete) group $G$.
In the end we want the torsors of $\abstr(G)$ to be equivalent to $\BG$, so to get the right definition we should first explore what the torsors of $G$ look like and prove~\cref{lem:BGbytorsor}, showing that $\BG$ is equivalent to the type of $G$-torsors.
\begin{definition}\label{def:Gtorsor}
  Given a group $G$, the type of \emph{$G$-torsors}%
  \index{torsor@torsor}%
  \glossary(TorsorG){$\protect\typetorsor_G$}{the type of $G$-torsors}%
  \footnote{This works equally well with $\infty$-groups: $G$-torsors are in that case $G$-types in the component of the principal torsor $\princ G:\BG\to\UU$. There is no conflict with the case when the $\infty$-group $G$ is actually a group since then any $G$-type in the component of the principal $G$-torsor will be a $G$-set.}
  is
  \[
    \typetorsor_G\defequi\sum_{X:\GSet}\Trunc{\princ G \eqto X},
  \]
  where $\princ G \jdeq (\sh_G \eqto \blank)$ is the 
  principal $G$-torsor of \cref{def:principaltorsor}.
\end{definition}

\begin{xca}\label{xca:torsor=free+transitive}
  Show that a $G$-set is a $G$-torsor if and only if it is free and transitive.
\end{xca}

\begin{remark}
  For $G$ a group, the type of $G$-torsors is just another name for the component of the type of \coverings over $\BG$ containing the universal \covering.

  Observe that for a group $G$, $\typetorsor_G$ is a connected groupoid\footnote{Admittedly in a higher universe, but we can use the
    Replacement~\cref{pri:replacement} to see that $\typetorsor_G$ is equivalent
    to a type in the same universe as $G$ -- even before we
    have~\cref{lem:BGbytorsor} showing we can take $\BG$.}
  and so -- by specifying the base point $\princ G$ -- it classifies a group.
  Guess which one!\footnote{%
    By the way, the name ``torsor'' is a translation from the French \emph{torseur},
    introduced by \citeauthor{giraud1971},\footnotemark{} who
    related them to “twisting” operations on bundles.
    Since $BG$ is equivalent to the type of $G$-torsors,
    we can also think of shapes $t:\BG$ as giving rise to “twists”.
    Indeed, for a $G$-set $X$,
    we can think of $X(x)$ as a ``twisted'' version of the underlying set,
    $X(\sh_G)$.}\footcitetext{giraud1971}.
\end{remark}


\begin{definition}
  \label{def:BG2TorsG}
Recall from~\cref{def:principaltorsor}\eqref{eq:pathsp}
the definition, for all $y:\BG$, of $\pathsp y:\BG\to\Set$
as the $G$-set with $\pathsp y(z)\jdeq(y\eqto z)$
(so that in particular $\princ G\jdeq\pathsp{\sh_G}$).
Note that $\pathsp y$ is a $G$-torsor, so we can define
  \[
    \pathsp{\blank}:\BG\ptdto(\typetorsor_G,\princ G): y\mapsto P_y,
  \]
  with pointing path $\refl{\princ G}:\princ G\eqto\pathsp{\sh_G}$.\footnote{%
    That is, we have classified a homomorphism from $G$
    to $\Aut_{\GSet}(\princ G)$. It'll turn out to be an isomorphism.}
If $G$ is not clear from the context, we may choose to write $\pathsp{\blank}^G$ instead of $\pathsp{\blank}$.
\end{definition}

\begin{remark}\label{rem:pathsptransport}
  We will use several variants of $\pathsp{\blank}$, in combination with
  some of the conventions introduced back in \cref{ch:univalent-mathematics}.
  In this remark, to avoid confusion, we explain these variants.
  
  First, we also use $\pathsp{\blank}$ to denote its induced action on paths:
  for $y,z:\BG$ we have
  \[
    \pathsp{\blank}:(y\eqto z)\to (\pathsp y\eqto \pathsp z),
  \]
  defined by path induction as in \cref{def:ap}.
  
  Then, as $\pathsp y\eqto \pathsp z$ is an identity between
  families of types, function extensionality (\cref{def:funext}) applies.
  For $q:y\eqto z$, we may also use $\pathsp q$ to denote the corresponding 
  function of type $\prod_{x:\BG}(\pathsp y(x)\eqto \pathsp z(x))$.
  
  Finally, as $\pathsp y(x)$ and $\pathsp z(x)$ are types,
  univalence (\cref{def:univalence}) applies.
  Therefore we may use $\pathsp q(x)$ to denote the corresponding 
  equivalence, \ie transport in the type family $\pathsp{\blank}(x)$,
  sending $p:\pathsp y(x)\jdeq (y\eqto x)$ to
  $pq^{-1}:\pathsp z(x)\jdeq(z\eqto x)$.\footnote{In a commutative diagram,
    \[
      \begin{tikzcd}[ampersand replacement=\&]
        y \ar[rr,eqr,"q"]\ar[dr,eql,"p"'] \& \& z \ar[dl,eqr,"{\pathsp q(p)}"] \\
        \& x.
      \end{tikzcd}
    \]}
\end{remark}

\begin{lemma}\label{lem:pathsptransportiseq}
  Let $G$ be a group. For all $y,z:\BG$ the induced map of identity types
  \[
    \pathsp{\blank}:(y\eqto z)\to (\pathsp y\eqto \pathsp z)
  \]
  is an equivalence.\footnote{%
    For connoisseurs of category theory,
    this is also a corollary of a \emph{type-theoretic Yoneda lemma},
    stating that transport gives an equivalence
    \[
      X(a) \equivto \prod_{b:A}\bigl((a \eqto b) \to X(b)\bigr)
    \]
    for any pointed type $(A,a)$ and type family $X: A \to \UU$.
    Try to prove this yourself!}
\end{lemma}
\begin{proof}
  We craft an inverse $Q:(\pathsp y\eqto \pathsp z) \to (y\eqto z)$ for
  $\pathsp{\blank}$. Given an identity $f:\pathsp y \eqto \pathsp z$, the map
  $f_y: (y\eqto y) \to (z\eqto y)$ maps the reflexivity path $\refl y$ to a path
  $f_y(\refl y):z\eqto y$, and we define
  \[
    Q(f) \defequi \inv{f_y(\refl y)} : (y\eqto z).
  \]
  First we construct an identification of $\pathsp {Q(f)}$ and $f$:
  for any $x:\BG$, 
  $\pathsp {Q(f)}(x)$ maps any $p:\pathsp{y}(x)\jdeq(y\eqto x)$ to
  $p f_y(\refl y)(x):\pathsp{z}(x)\jdeq(z\eqto x)$. Hence we must
  construct an identification of type $p f_y(\refl y)(x)=f_x(p)$,
  which is immediate by induction on $p:y\eqto x$, setting $p\jdeq \refl y$.
  
  Next, we prove the equality of $Q(\pathsp q)$ and $q$
  for every $q:y\eqto z$. Indeed, 
  $Q(\pathsp q)\jdeq\inv{(\pathsp{q})_y(\refl y)} = \inv{(\refl y \inv q)} = q$.
  
  Now apply \cref{lem:weq-iso} to complete the proof of the lemma.
\end{proof}

The following theorem justifies the title of this section, stating 
that the classifying type of a group is the type of its torsors.

\begin{theorem}\label{lem:BGbytorsor}
  If $G$ is a group, then the function
  $\pathsp{\blank}:\BG\to\typetorsor_G$ from~\cref{def:BG2TorsG}
  is an equivalence.\footnote{A similar results holds for $\infty$-groups.}
\end{theorem}

\begin{proof}
  Since both $\typetorsor_G$ and $\BG$ are pointed and connected,
  it suffices by
  \cref{cor:fib-vs-path}\ref{conn-fib-vs-path-point} to show that 
  $\pathsp{\blank}:(\sh_G\eqto\sh_G)\to(\pathsp{\sh_G}\eqto \pathsp{\sh_G})$
  is an equivalence.\footnote{%
  This holds for all variants of $\ap{\pathsp{\blank}}$.}
  This follows directly from \cref{lem:pathsptransportiseq}.
\end{proof}

\subsection{Homomorphisms and torsors}
\label{sec:homotor}
In view of the equivalence $\pathsp{}^G$ between $\BG$ and 
$(\typetorsor_G,\princ G)$ of \cref{lem:BGbytorsor} one might 
ask what a group homomorphism  $f:\Hom(G,H)$ translates to on 
the level of torsors.  Off-hand, the answer is the round-trip 
$(\pathsp{}^H)\Bf(\pathsp{}^G)^{-1}$, but we can be more concrete than that.
We do know that for $x:\BG$ the $G$-torsor $\pathsp x^G$ should be sent to
$\pathsp {\Bf(x)}^H$, but how do we express this for an arbitrary $G$-torsor?
\begin{definition}
  \label{def:restrictandinduce}
  Let $f:\Hom(G,H)$ be a group homomorphism.  If $Y:\BH\to\Set$ is an $H$-set,
  then the \emph{restriction}\index{action!restricted}\index{restriction}
  $f^*Y$ of $Y$ to $G$ is the $G$-set given by precomposition\footnote{%
  \MB{New: }Example: \cref{ft:restriction}.}
  \[
    f^*Y\defequi (Y\circ\Bf) :\BG\to\Set.
  \]

  If $X:\BG\to\Set$ is a $G$-set, we define
  the \emph{induced $H$-set}\index{action!induced}\index{induced action}
  $f_*X : \BH\to\UU$ by setting, for $y:\BH$,
  \[
    f_*X(y)\defeq\myTrunc{\sum_{z:\BG}(\Bf(z) \eqto y)\times X(z)}{0}.\qedhere
  \]
\end{definition}
The following exercise motivates the set-truncation in the definition
of $f_*$ above.\footnote{%
    This situation is common in algebra and is often referred to by saying
    that some construction, in this case the untruncated
    definiens of $f_*X$, is not ``exact''. See also \cref{xca:why-setTrunc_f_*}.}

\begin{xca}\label{xca:why-setTrunc_f_*}
Find groups $G,H$, $f:\Hom(G,H)$ and $G$-set $X$ such that
$\sum_{z:\BG}(\Bf(z) \eqto y)\times X(z)$ is not an $H$-set (but an $H$-type).
\end{xca}
% Solution: $G=\ZZ$, $H=\TG$, $f$ the unique homomorphism $f: \Hom(G,H)$,
% $X$ constant $\bn 1$. Then 
% $\sum_{z:\Sc}(0=0)\times \bn 1$ is a circle, so not a set.

\begin{xca}\label{xca:id_*-is-id}
\MB{New:} Give an equivalence from $f_*\,X$ to $X\circ\inv\Bf$
    if $f$ is an isomorphism. Give an equivalence between the identity
    types $f_*\,X \eqto Y$ and $X \eqto f^*\,Y$, for all $G$-sets $X$
    and $H$-sets $Y$.
\end{xca}


Note that the type $f_*X(y)$ is also the action type $(H^y \times X)_{hG}$,
of the $G$-set $H^y\times X$,
where $(H^y\times X)(x) \defeq (\Bf(x) \eqto y)\times X(x)$ for $x:\BG$,
and whose underlying set is equivalent to $(\sh_H\eqto y)\times X(\sh_G)$.

\begin{remark}
  Dually, there is also a \emph{coinduced $H$-set}\index{action!coinduced}
  $f_!:\BH\to\Set$ given by
  \[
    f_!X(y)\defeq\prod_{z:\BG}\bigl((\Bf(z)\eqto y) \to X(z)\bigr).
  \]
  Note that this always lands in sets when $X$ does.
\end{remark}

When $X$ is the $G$-torsor $\pathsp x^G$, for some $x:\BG$,
the contraction (recall \cref{lem:contract-away})
of $\sum_{z:\BG}(x\eqto z)$ induces an equivalence
\[
  \eta_y:f_*\pathsp x^G(y) \jdeq 
  \myTrunc{\sum_{z:\BG}(\Bf(z) \eqto y)\times(x \eqto z)}{0}
  \equivto (\Bf(x)\eqto y)\jdeq\pathsp{\Bf(x)}^H(y).
\]
Taking $x\jdeq\sh_G$, so $\pathsp x^G\jdeq\princ G$, we get a
path $\eta:f_*\,\princ G\eqto \pathsp{\Bf(\sh_G)}^H$.
We also have the path $\Bf_\pt : \sh_H\eqto\Bf(\sh_G)$,
so that the action of $\pathsp{\blank}^H$ gives us a path
$\pi : \princ H \jdeq\pathsp{\sh_H}^H \eqto \pathsp{\Bf(\sh_G)}^H$.
Combining we get $\inv\eta\pi:\princ H \eqto f_*\,\princ G$.

If $X$ is a $G$-set such that $\Trunc{\princ G \eqto X}$, then $f_*X$
is an $H$-set such that $\Trunc{\princ H \eqto f_*X}$, so that
$f_* : \typetorsor_G \ptdto \typetorsor_H$, pointed by $\inv\eta\pi$.

Summing up, we have implemented the following:
\begin{construction}
  \label{lem:inducedtorsor}
   Let $f:\Hom(G,H)$ be a group homomorphism. Then $f$ induces a 
   pointed map $f_*:\typetorsor_G\ptdto\typetorsor_H$,
   and we have a path of type 
   $f_*\,\pathsp{\blank}^G = \pathsp{\blank}^H\,\Bf \jdeq 
   f^*\,\pathsp{\blank}^H$,
   all represented by the following diagram:
   \[
     \begin{tikzcd}
     \sh_G \ar[ddd,mapsto] \ar[rrr,mapsto] &
     && \Bf(\sh_G)  & \ar[l,eql,"{\Bf_\pt}"'] \sh_H \ar[ddd,mapsto]\\
       &\BG \ar[r,"\Bf"]\ar[d,"{\pathsp{\blank}^G}"'] &
        \BH\ar[d,"{\pathsp{\blank}^H}"] \\
       &\typetorsor_G \ar[r,"f_*"'] & \typetorsor_H \\
     \princ G \ar[rrr,mapsto] &
     && f_*\,\princ G&  \ar[l,eqr,"{\inv\eta\pi}"] \princ H.
     \end{tikzcd}
   \]
\end{construction}

\section{Any symmetry is a symmetry in $\Set$}
\label{sec:groupssubperm}


For abstract groups there is a result, attributed to Cayley,
which is often stated as ``any group is a permutation group''. 
In our parlance this translates to ``any symmetry is a symmetry in $\Set$''.
The aim of this section is to give a precise formulation of the latter
and prove it.
% \footnote{which reminds me of the following: my lecturer in cosmology once tried to publish a paper about rotating black holes, only to have it rejected because it turned out that it was his universe, not the black hole, that was rotating}

%which is equivalent to saying that $X$ is the universal \covering
Let $G$ be a group.
Recall from \cref{def:principaltorsor} the principal torsor
$\princ G:\BG\to\Set: z\mapsto (\sh_G\eqto z)$.
Since $\princ G (\sh_G)\defequi \USymG$, $\princ G$ restricts to
a pointed function $\BG\ptdto \BSG_{\USymG}$, \ie 
classifies a homomorphism from $G$ to the permutation group 
$\SG_{\USymG}\jdeq \Aut_{\Set}(\USymG)$,
denoted by\footnote{The letter $\rho$ commemorates the word ``regular''} 
$$\rho_G:\Hom(G,\SG_{\USymG}).$$

\begin{theorem}[Cayley]
  \label{lem:allgpsarepermutationgps}
  For all groups $G$, $\rho_G$ is a monomorphism.\footnote{By
  \cref{def:typeofmono}, $\rho_G$ is a monomorphism means 
  that the induced map $\USym\rho_G$ from the symmetries of $\sh_G$ in 
  $\BG_\div$ to the symmetries of $\USymG$ in $\Set$ is an injection, 
  \ie ``any symmetry is a symmetry in $\Set$''.}  
\end{theorem}

\begin{proof}
  In view of \cref{def:typeofmono} we need to show that 
  $\B\rho_G\jdeq\princ G :\BG \to \BSG_{\USymG}$ is a \covering.
  Under the pointed equivalence
  $$\pathsp{\blank}:\BG\ptdto (\typetorsor_G,\princ G)$$ of
  \cref{lem:BGbytorsor}, $\princ G$ is transported to\footnote{
  See \cref{xca:evP_isPrG}.} to the
  evaluation map
  $$\mathrm{ev}_{\sh_G}:\conncomp{(\BG\to\Set)}{\princ G}\ptdto
  \conncomp{\Set}{\USymG},\qquad
  \mathrm{ev}_{\sh_G}(E)\defeq E(\sh_G).$$
  We must show that the preimages
  $\inv{\ev_{\sh_G}}(X)$ for $X:\Sigma_{\USymG}$ are sets.  This
  fiber is equivalent to
  $\sum_{E:\conncomp{(\BG\to\Set)}{\princ G}}(X\eqto E(\sh_G))$ which,
  being a subtype, is a
  set precisely when $\sum_{E:\BG\to\Set}(X\eqto E(\sh_G))$ is a set.
  The latter is the type of pointed maps from $\BG$ to $(\Set,X)$
  and hence a set by \cref{lem:hom-is-set}, 
  in particular \cref{ft:ptd-decr-h-lev}.
\end{proof}
  Note that the above theorem yields that 
  $(G,\rho_G,!)$ is a monomorphism into $\SG_{\USymG}$.
  In other words, $G$ is a subgroup of $\SG_{\USymG}$.

\begin{xca}\label{xca:evP_isPrG}
\MB{New:} Show that $\princ G$ and ${\ev_{\sh_G}}\circ{\pathsp{\blank}}$ are
equal as pointed maps.
\end{xca}

\begin{remark}\label{rem:CayleyOversize}
\MB{New:} In many cases, the set $\USymG$ used in \cref{lem:allgpsarepermutationgps} is larger than necessary for
obtaining the symmetries in $G$ as symmetries of a set.
A case in point is the group $\SG_3$, where the symmetries \emph{are}
already symmetries of a set, namely of the set $\bn3$. However,
$\USG_3\jdeq(\bn3\eqto\bn3)$ is a $6$-element set. 
Let's take a closer look at where and how this happens in the proof.

As stated in \cref{xca:evP_isPrG}, the map $\princ G: 
\BG\ptdto\conncomp{\Set}{\USymG}$ classifying the monomorphism $\rho_G$ 
is decomposed as an equivalence $\pathsp{\blank}$
followed by the evaluation map $\ev_{\sh_G}$.
This is depicted in the following diagram, where the
second line shows the induced maps on the symmetries.
   \[
     \begin{tikzcd}
     \BG \ar[r,equivr,"\pathsp{\blank}"] & 
     (\typetorsor_G,\princ G) \ar[r,"\ev_{\sh_G}"]&
     \conncomp{\Set}{\USymG}\\
     \USymG \ar[r,equivr,"\pathsp{\blank}"] &   %\loops?
     (\princ G \eqto \princ G) \ar[r,"\ev_{\sh_G}"]&  %\loops?
     (\USymG \eqto \USymG)  
     \end{tikzcd}
   \]
Let $\PP$ be the $G$-set given by 
$\PP(z)\defeq(\princ G(z) \eqto \princ G(z))$ for all $z:\BG$.
By function extensionality, $\princ G \eqto \princ G$ is
equivalent to $\prod_{z:\BG}\PP(z)$, the type of
invariant maps of $\PP$.
By \cref{lem:fixpts-are-fixed}\ref{it:ev-is-inj}, such invariant maps,
and hence the corresponding symmetries of $\princ G$, are uniquely
determined by their value at $\sh_G$.\footnote{%
This is an alternative way to understand that $\ev_{\sh_G}$,
and hence $\princ G$, classifies a monomorphism.}

Note that the underlying set of $\PP$ is 
$\PP(\sh_G)\jdeq(\USymG \eqto \USymG)$.
However, $\PP$ has more structure than its underlying set.
\cref{lem:fixpts-are-fixed}\ref{it:ev-is-eq-on-inv} characterizes
exactly the invariant maps of $\PP$ as corresponding via $\ev_{\sh_G}$
with fixed elements of $\USymG \eqto \USymG$. In other words,
$\ev_{\sh_G}$ forgets about the extra structure of $\PP$
and sends invariant maps of $\PP$ to fixed permutations of $\USymG$.
For example, in the case of $\SG_3$, we go in total from permutations
of $\bn3$ to fixed permutations of the $6$-element set $\bn3\eqto\bn3$.

In \cref{xca:PP-fixed-permutations} you are asked to explore
the abstract group of fixed permutations of $\USymG$.
\end{remark}

\begin{xca}\label{xca:PP-fixed-permutations} \MB{New:}
Let conditions be as in \cref{rem:CayleyOversize}.
By analyzing transport in the type family $\princ G(\blank)$,
show that a permutation $\pi$ of $\USymG$
is fixed if and only if $\pi(gg')=g\pi(g')$ for all $g,g':\USymG$.
Show that the fixed permutations of $\USymG$ form an abstract group
and that evaluation of such a permutation at $\refl{\sh_G}$
yields an abstract isomorphism from this group to $\abstr(G)$.
\end{xca}
  

\begin{xca} \MB{MB doesn't understand:}
    Given a group $G$ we defined in \cref{sec:groupssubperm} a monomorphism from $G$ to the permutation group $\Aut_{\USymG}(\Set)$. Write out the corresponding subgroup of $\Aut_{\USymG}(\Set)$.
  \end{xca}

\section{The lemma that is not Burnside's}
\label{sec:burnsides-lemma}
\label{lem:burnsides-lemma}

\begin{example}\label{exa:prep-burnside}
Since the lemma to come is about counting orbits and
elements of orbits, we start by elaborating an example.
Recall from \cref{ex:cyclicgroups} the cyclic group $\CG_4$.
Let $X: \BCG_4\to\Set$ be the $\CG_4$-set mapping any $(A,f):\BCG_4$
to $A\to\bn2$. Then the underlying set of $X$ is $\bn4\to\bn2$,
\ie binary sequences of length $4$. The group action induced by $X$
cyclically rotates such sequences, by $0,1,2$ or $3$ positions.\footnote{%
Use \cref{cor:id-m-cycle}, univalence, and \cref{lem:trp-in-function-type}.}

By \cref{cor:orbit-equiv}, the set of orbits $X/\CG_4$ is
equivalent to the quotient of $\bn4\to\bn2$ induced by 
$[\blank] : (\bn4\to\bn2) \to X/\CG_4$ from \cref{lem:[]0-surj-on-orbits}.
As also stated by that lemma, the equivalence class of any $x:\bn4\to\bn2$
consists precisely of all cyclic rotations of $x$. Clearly, 
$0000$ and $1111$ have singleton equivalence classes.
The equivalence class of $0001$ (resp.\ $0111$) consists of all four binary 
sequences with exactly one $1$ (resp.\ $0$).  Before you start thinking that
swapping $0$'s and $1$'s gives a new equivalence class, consider 
$0101$ that forms an equivalence class together with $1010$.
Finally, $0011$ forms an equivalence class together with $1001$,
$1100$ and $0110$. Thus we have distributed all 16 sequences over
six orbits, as in the left column of \cref{fig:C4-action-on-4-bits}.

\begin{margintable}\label{fig:C4-action-on-4-bits}
  \footnotesize
\begin{tabular}{ccc} \toprule
 orbit & stabilizers \\ \midrule
 $0000$ & $0,1,2,3$ \\
 $1111$ & $0,1,2,3$ \\
 $0001,0010,0100,1000$ & $0$ \\
 $0111,1011,1101,1110$ & $0$ \\
 $0101,1010$ & $0,2$ \\
 $1100,0110,0011,1001$ & $0$ \\ \bottomrule
\end{tabular}
\caption{\label{fig:C4-action-on-4-bits}
Underlying sets of orbits and the stabilizers of their elements.}
\end{margintable}

In the right column of \cref{fig:C4-action-on-4-bits}, we have
given in each row the respective stabilizing symmetries in $\BCG_4$.
\cref{xca:[x]=[y]-implies-||Gx=Gy||} tells us that it doesn't matter
too much which element in the orbit one chooses.\footnote{%
  Here it matters even less since $\CG_4$ is abelian.}
For the cardinality $\Card((\CG_4)_x)$ of the finite stabilizer groups, 
the particular $x$ one chooses within in each orbit is irrelevant,
but may vary from orbit to orbit. Now we can observe something
interesting: the product $\Card({\CG_4}\cdot {x}) \times \Card((\CG_4)_x)$
(\ie in each row, the number of elements on the left times that on the right)
is equal to $\Card(\CG_4) = 4$, for each $x$ in the underlying set of $X$.
This follows from Lagrange's Theorem, in particular \cref{xca:lagrange},
applied with $G\jdeq\CG_4$ and taking for $X$ the underlying $\CG_4$-set
of $[x]$, which is transitive.

Another observation in \cref{fig:C4-action-on-4-bits} is that, 
since there are six orbits and the orbits induce a disjoint partition
of $\bn4\to\bn2$, there are in total 24 pairs $(g,x)$ with $g\cdot x = x$.
This insight leads to the following lemma.
\end{example}


\begin{lemma}
  \label{lem:burnside}
  Let $G$ be a finite group and let $X:\BG\to\Set$ be a finite $G$-set.
  Define $X^g = \setof{x:X(\sh_G)}{g\cdot x = x}$ for any $g:\USymG$.
  Then each $X^g$, the sum type $\sum_{g:\USymG} X^g$, and the set of orbits $X/G$
  are finite sets, and we have
  \[
    \Card\Bigl(\sum_{g:\USymG} X^g\Bigr) = \Card(X/G) \times \Card(G).
  \]
\end{lemma}
\begin{proof}
  We first need to make sure that the sets involved are finite. 
  Finite sets are decidable sets, see \cref{xca:finsets-decidable}.
  Hence each $X^g$ is a finite set, as it is a decidable subset of $X(\sh_G)$.%
 \footnote{%
  A subset of a finite set is not necessarily finite itself:
  Let $p$ be a proposition and consider $\bn1_p\defeq\sum_{x:\bn1}p$,
  the subset of $\bn1$ defined by the predicate that is constant $p$.
  If $\bn1_p$ is a finite set, then we have 
  $\Card(\bn1_p) :\NN$, and we can prove that
  $p$ holds if and only if $\Card(\bn1_p) = 1$.
  Since equality in $\NN$ is decidable, this would mean that
  we can decide $p$. In fact we have that $\bn1_p$ is a finite set
  if and only if $p$ is decidable.
  In general, if $S$ is a finite set, then every
  decidable predicate on $S$ defines a finite subset of $S$.
  Similarly, the quotient of a finite set modulo a decidable
  equivalence relation is finite, see \cref{xca:dec-quot-finite-set}.
 }
  Finiteness of $\sum_{g:\USymG} X^g$ follows from 
  \cref{xca:sum-over-finite-set}. Regarding the set of orbits,
  note that \cref{cor:orbit-equiv} yield that $X/G$ is equivalent
  to the quotient of $X(\sh_G)$ modulo the equivalence relation
  $\exists_{g:\USymG} x=g\cdot y$. The latter proposition is decidable by
  searching for such a $g$. Now apply \cref{xca:dec-quot-finite-set}.
  
  Since the main statement (displayed) of the lemma is a proposition, 
  we may assume that we have an equivalence to a standard finite set of 
  the form $\bn n$, for every finite set at hand.  
  Rearranging sums and writing $X(\sh_G)$ as the sum of fibers
  of $[\blank]: X(\sh_G)\to X/G$ gives equivalences
  \[
    \sum_{g:\USymG} X^g \equivto \sum_{x:X(\sh_G)} \USymG_x
    \equivto \sum_{O:X/G} \sum_{x : X_O(\sh_G)} \USymG_x.
  \]
  Note that the last type in the chain above reflects how
  we counted in \cref{fig:C4-action-on-4-bits}: for every orbit,
  and every element in the underlying set of that orbit,
  we counted the stabilizers of that element.
  
  We aim to apply Lagrange's \cref{con:lagrange} with subgroups 
  defined by $X_O$ and
  $x_O : X_O(\sh_G)$, for any orbit $O:X/G$. These points $x_O$ can
  be obtained as the `least' $x:X(\sh_G)$ such that $O=[x]$,
  where `least' means: corresponding to the smallest number under the
  equivalence of $X(\sh_G)$ with a standard finite set.
  We also have to give functions
  $f_O : \prod_{y:X_O(\sh_G)}\sum_{g:\USymG} g\cdot_{X_O} y = x_O$,
  for every $O:X/G$.
  Such functions are obtained by using the transitivity of $X_O$
  in combination with the equivalence between $\USymG$ and a 
  standard finite set: we can simply take the `least' $g:\USymG$
  such that $g\cdot_{X_O} y = x_O$.
  Applying \cref{con:lagrange}, in particular \cref{cor:lagrange-dep-sum},
  we get an equivalence between $\USymG$ and $\sum_{x : X_O(\sh_G)} \USymG_x$. 
  We conclude that $\Card(\sum_{g:\USymG} X^g) = \Card(X/G) \times \Card(G)$,
  using \cref{xca:sum-over-finite-set}.
\end{proof}

As a first application of Burnside's Lemma, we not the following
number-theoretic consequence, which falls out when we consider
the analog of~\cref{exa:prep-burnside} for the case of $\CG_p$ acting
on base-$n$ sequences of length $p$.

\begin{theorem}[Fermat's Little Theorem]
  For any prime $p$ and natural number $n$, we have $p \mid {n^p-n}$.
\end{theorem}
\begin{proof}
  Consider the action $X : \BCG_p \to \Set$ of the cyclic group $\CG_p$ on
  a set of size $n^p$ given by
  \[
    X(S,t) \defeq (S \to \bn n),
  \]
  for any $p$-cycle $(S,t)$. The underlying set is the type of functions
  $\bn p \to \bn n$, which is finite of cardinality $n^p$.

  Now apply Burnside's~\cref{lem:burnside}. The stabilizer subgroup
  of a function $f : \bn p \to \bn n$ is either trivial or all of $\CG_p$.
  In the former case, $f$ is one of the $n$ constant functions,
  and all the other $n^p-n$ possible functions are free.
  We get:
  \[
    \Card\Bigl(\sum_{g:\UCG_p} X^g\Bigr)
    = np + (n^p-n) = \Card(X/\CG_p) \times \Card(\CG_p),
  \]
  and since $\Card(\CG_p)=p$, we conclude that $p$ divides $n^p-n$.
\end{proof}

\begin{xca}\label{xca:dec-quot-finite-set}\MB{Where?}
Let $X$ be a finite set and $R:X\to X\to\Prop$ a decidable
equivalence relation. Show that the quotient $X/R$ is a finite set.
\end{xca}

\begin{xca}\label{xca:sum-over-finite-set}
Let $X$ be a finite set and $f:X\to\NN$ a function.
Define the arithmetical sum $(\sum_{x:X} f(x)):\NN$.
Consider now a family $F: X\to\FinSet$ of finite sets. Show that the sum type
$\sum_{x:X}F(x)$ is a finite set with cardinality $\sum_{x:X} \Card(F(x))$.
\MB{Hint:} Key is the invariance of the sum under permutation of $X$.
You could do the second part first, but you can also get the first part
as a nice application of a fixed element:
Define the $\SG_{\Card(X)}$-set $A(Y)\defeq(Y\to\NN)\to\NN$
for all $Y:\BSG_{\Card(X)}$. Show that summation is a fixed
element of $A(\bn n)$ (\cref{def:fixed-free}).
Now apply \cref{lem:fixpts-are-fixed}.
%Alternative solution:
%Define the type $\Sub^d(X)$ of \emph{decidable} predicates on $X$. 
%Consider the type $A\defeq\bigl(\prod_{Y:\Sub^d(X)}(Y\to\NN)\bigr)\to\NN$
%of functions that `aggregate' the values of a function $g:Y\to\NN$ over 
%the subset $X_Y$ of $X$. Define the predicate $\Sigma_A: (A\to\Prop)$
%that singles out the function(s) in $A$ that aggregate by summation:
%$\Sigma^X_A(G)\defeq$
%\[
%((\prod_{g:\bn0\to\NN}G(\bn0,g)=0))\times
%(\prod_{Y:\Sub^d(X)}\prod_{g:Y\to\NN}\prod_{y:Y}
%G(Y,g)=f(y)+G(Y_{{\neq}y},g_{{\neq}y})).
%\]
%Now show that $\Tot(\Sigma^X_A)$ is contractible.
\end{xca}


%%% Local Variables:
%%% mode: LaTeX
%%% fill-column: 144
%%% latex-block-names: ("lemma" "theorem" "remark" "definition" "corollary" "fact" "properties" "conjecture" "proof" "question" "proposition" "exercise")
%%% TeX-master: "book"
%%% TeX-command-extra-options: "-fmt=macros"
%%% compile-command: "make book.pdf"
%%% End:
