\chapter{Rings, fields and vector spaces}
\label{ch:fields}

In this chapter we will extend the hierarchy of algebraic structures from 
monoids (\cref{def:monoid}) and 
groups (\cref{def:typegroup}) to
rings (\cref{def:abstractring}),
fields (\cref{def:field}), and
vector spaces (\cref{def:vectorspace}).
Of all these structures there are several varieties, 
satisfying additional properties, such as 
abelian groups (\cref{sec:abelian-groups}),
non-trivial rings (\cref{def:non-trivial-ring}),
commutative rings (\cref{def:commutative-ring}),
....

Quotients; subspaces (= ?). Bases and so. Dual space; orthogonality. (all of this depends on good implementations of subobjects). Eigen-stuff. Characteristic polynomials; Hamilton-Cayley.

\section{Rings, abstract and concrete}

A ring is an algebraic structure that consists of a group and a
monoid that share the same underlying set. The interaction between
the respective operations is governed by laws that are called
the distributivity laws. We will start by defining rings abstractly.
The standard example of a (commutative) ring is the
ring of the integers with addition as
group operation and multiplication as monoid operation.
Note that multiplication in a ring need not be commutative.\footnote{%
In contrast, in \cref{xca:ring-group-abelian} you are asked to prove
that the group of a ring is always abelian, as a consequence of the
extra structure and properties.} 
We follow the convention that the group data are denoted
by $0,\,+,\,-$ and the monoid data by $1,\,\cdot\,$.

\begin{definition}\label{def:abstractring}
An \emph{abstract ring} $\mathscr R$ consists of an abstract group 
$(R,0,+,-)$ and a monoid $(R,1,\cdot)$ with the
same underlying set $R$. Moreover, the following equations should hold,
for all $a,b,c : R$.
    \begin{enumerate}[ref=\ref{def:abstractring} (\alph*)]
    \item\label{ring:ldistr-law} $a \cdot (b + c) = a \cdot b + a \cdot c$, (the \emph{left distributive law})
    \item\label{ring:rdistr-law} $(a + b) \cdot  c) = a \cdot c + b \cdot c$, (the \emph{right distributive law})
    \end{enumerate}
The abstract ring $\mathscr R$ is called \emph{commutative} if its
multiplication $\cdot$ is commutative, and \emph{non-trivial} if $0\neq 1$.
\end{definition}

The abstract group $(R,0,+,-)$ is called the \emph{(additive) group}
of $\mathscr R$, and the monoid $(R,1,\cdot)$ the 
\emph{(multiplicative) monoid} of $\mathscr R$.

\begin{xca}\label{xca:ring-group-abelian}
Let $\mathscr R$ be an abstract ring. Show that the additive group 
of $\mathscr R$ is abelian. Hint: elaborate $(a+1)\cdot(b+1)$.
\end{xca}

Note that, for any abstract ring $\mathscr R$ and element $a:R$,
both the left multiplication function $(a\cdot\blank)$ 
and the right multiplication function $(\blank\cdot a)$ 
are abstract homomorphisms of the additive group of $\mathscr R$ to itself.
Note that these functions provide two ways to write the product $a\cdot b$.
These observations lead to the following definition of a 
ring in our set up.

\begin{definition}\label{def:ring}
An \emph{ring} $R$ consists of an group $G$ together with
a symmetry $1_R : \USymG$ and two maps $\ell,r: \USymG\to\Hom(G,G)$
from the set of symmetries in $G$ to the set of homomorphisms from
$G$ to $G$.
Given $g:\USymG$, we write $\ell_g$ for the homomorphism $\ell(g)$ and 
$r_g$ for $r(g)$.
Moreover, the following equations should hold.
    \begin{enumerate}[ref=\ref{def:ring} (\alph*)]
    \item\label{ring:unit-laws} $\ell_{1_R} = \id_G = r_{1_R}$ (the \emph{multiplicative unit laws})
    \item\label{ring:lr-coherence-law} $\USyml_g(h) = \USymr_h(g)$, for all $g,h : \USymG$ (the \emph{coherence law})
        \item\label{ring:associativity-law} $\USyml_g(\USyml_h(i)) = \USymr_i(\USymr_h(g))$, for all $g,h,i : \USymG$ (the \emph{associativity law})
    \end{enumerate}
The ring $R$ is called \emph{commutative} if $\ell=r$, 
and \emph{non-trivial} if $\refl{\sh_G}\neq 1_R$.
\end{definition}
The coherence law above allows us to abbreviate both 
$\USyml_g(h)$ and $\USymr_h(g)$ by $g\cdot h$. We will do this when
no confusion can occur. Then, \eg the associativity law above
gets its usual form: $g\cdot(h\cdot i) = (g\cdot h)\cdot i$.
Moreover, using the same notation, $\ell=r$ 
amounts to $g\cdot h = h\cdot g$, for all $g,h : \USymG$,
as could be expected from the abstract case.

We proceed by giving the standard example of the integers as a ring
in the sense of \cref{def:ring}.
\begin{example}
Consider the group $\ZZ$ classified by the circle.
Using the same notation $\ZZ$ also for the ring, take $1_\ZZ \defeq\Sloop$
and $\ell: (\base\eqto\base)\to\Hom(\ZZ,\ZZ)$ defined as follows.
For every $g:\base\eqto\base$, let $\ell_g$ be the homomorphism
classified by the map $\B\ell_g(\base)\defeq\base$, 
$\B\ell_g(\Sloop)\defis g$, and pointed by reflexivity.\footnote{%
The reader may recognize the degree $m$
map from \cref{def:mfoldS1cover} as a special case.}
Take $r\defeq\ell$. Now the unit laws, the coherence law and
the associativity law can easily be verified. It follows that
$(\ZZ,1_\ZZ,\ell,!)$ is a non-trivial commutative ring.
\end{example}

\MB{TBD define type of (abstract) rings, prove equivalence, define
ring homomorphisms, delooping etc. No interesting difficulties
expected before we come to fields.}


\begin{definition}
Given a commutative ring $R$, an element $e:R$ is \textbf{invertible} if there exists an element $a:R$ such that $e \cdot a = 1$ and $a \cdot e = 1$:
$$\mathrm{isInvertible}(e) := \left\Vert\sum_{a:R} (e \cdot a = 1) \times (a \cdot e = 1)\right\Vert$$
\end{definition}

\begin{theorem}
In any nontrivial commutative ring $R$, $0$ is always a non-invertible element. 
$$\mathrm{isNonTrivialCRing}(R) \to \neg \mathrm{isInvertible}(0)$$
\end{theorem}

\begin{proof}
Suppose that $0$ is invertible. Then there exists an element $a:R$ such that $a \cdot 0 = 1$. However, due to the absorption properties of $0$ and the fact that $R$ is a set, $a \cdot 0 = 0$. This implies that $0 = 1$, which contradicts the fact that $0 \neq 1$ in a nontrivial commutative ring. Thus, $0$ is a non-invertible element in any nontrivial commutative ring $R$. 
\end{proof}

\begin{definition}
A nontrivial commutative ring $R$ is a \textbf{field} if and only if the type of all non-invertible elements in $R$ is contractible:
$$\mathrm{isField}(R) := \mathrm{isNonTrivialCRing}(R) \times \mathrm{isContr}\left(\sum_{x:R} \neg \mathrm{isInvertible}(x)\right)$$ 
Equivalently, $R$ is a field if and only if every non-invertible element is equal to zero. 
\end{definition}

\begin{remark}
In other parts of the constructive mathematics literature, such as in Peter Johnstone's \textit{Rings, Fields, and Spectra}, this is called a "residue field". However, in this book we shall refrain from using the term "residue field" for our definition, since that contradicts the usage of "residue field" in other parts of mathematics, such as in algebraic geometry. 
\end{remark}

\begin{definition}
A field is \textbf{discrete} if every element is either invertible or equal to zero. 
$$\mathrm{isDiscreteField}(R) := \mathrm{isField}(R) \times \prod_{a:R} \Vert(a = 0) \amalg \mathrm{isInvertible}(a)\Vert$$ 
\end{definition}

\begin{definition}
A nontrivial commutative ring $R$ is a \textbf{local ring} if for every element $a:R$ and $b:R$, if the sum $a + b$ is invertible, then either $a$ is invertible or $b$ is invertible. 
$$\mathrm{isLocalRing}(R) := \mathrm{isNonTrivialCRing}(R) \times \prod_{a:R} \prod_{b:R} \mathrm{isInvertible}(a + b) \to \Vert\mathrm{isInvertible}(a) \amalg \mathrm{isInvertible}(b)\Vert$$ 
\end{definition}

\begin{definition}
A field $R$ is \textbf{Heyting} if it is also a local ring. 
$$\mathrm{isHeytingField}(R) := \mathrm{isField}(R) \times \mathrm{isLocalRing}(R)$$ 
\end{definition}

References used in this section: 
\begin{itemize}
\item Emmy Noether, \textit{Ideal Theory in Rings}, Mathematische Annalen 83 (1921)
\item Henri Lombardi, Claude Quitté, \textit{Commutative algebra: Constructive methods (Finite projective modules)}
\item Peter Johnstone, \textit{Rings, Fields, and Spectra}, Journal of Algebra 49 (1977) 238-260
\end{itemize}

\section{vector spaces}

\begin{definition}
Given a field $K$, a $K$-\textbf{vector space} is an abelian group $V$ with a bilinear function $(-)(-):K \times V \to V$ called \textbf{scalar multiplication} such that $1 v = v$ and for all elements $a:K$, $b:K$, and $v:V$, $(a \cdot b) v = a (b v)$. 
\end{definition}

\begin{definition}
A $K$-\textbf{linear map} between two $K$-vector spaces $V$ and $W$ is a group homomorphism $h:V \to W$ which also preserves scalar multiplication: for all elements $a:K$ and $v:V$, $f(a v) = a f(v)$. 
\end{definition}

\begin{definition}
Given a field $K$ and a set $S$, the \textbf{free $K$-vector space} on $S$ is the homotopy initial $K$-vector space $V$ with a function $i:S \to V$: for every other $K$-vector space $W$ with a function $j:S \to W$, the type of linear maps $h:V \to W$ such that for all elements $s:S$, $h(i(s)) = j(s)$ is contractible. 
\end{definition}

\begin{definition}
Given a field $K$ and a natural number $n$, an \textbf{$n$-dimensional $K$-vector space} is a free $K$-vector space on the finite type $\mathrm{Fin}(n)$. 
\end{definition}

\section{the general linear group as automorphism group}
\section{determinants\titledagger}
\section{examples: rationals, polynomials, adding a root, field extensions}
\section{ordered fields, real-closed fields, pythagorean fields, euclidean fields}
\section{complex fields, quadratically closed fields, algebraically closed fields}

%%% Local Variables:
%%% mode: latex
%%% TeX-master: "book"
%%% End:
