\chapter{Rings, fields and vector spaces}
\label{ch:fields}

In this chapter we will extend the hierarchy of algebraic structures from 
monoids (\cref{def:monoid}) and 
groups (\cref{def:typegroup}) to
rings (\cref{def:abstractring}),
fields (\cref{def:field}), and
vector spaces (\cref{def:vectorspace}).
Of all these structures there are several varieties, 
satisfying additional properties, such as 
abelian groups (\cref{sec:abelian-groups}),
non-trivial rings (\cref{def:non-trivial-ring}),
commutative rings (\cref{def:commutative-ring}),
....

Quotients; subspaces (= ?). Bases and so. Dual space; orthogonality. (all of this depends on good implementations of subobjects). Eigen-stuff. Characteristic polynomials; Hamilton-Cayley.

\section{Rings, abstract and concrete}

A ring is an algebraic structure that consists of a group and a
monoid that share the same underlying set. The interaction between
the respective operations is governed by laws that are called
the distributivity laws. We will start by defining rings abstractly.
The standard example of a (commutative) ring is the
ring of the integers with addition as
group operation and multiplication as monoid operation.
Note that multiplication in a ring need not be commutative.\footnote{%
In contrast, in \cref{xca:ring-group-abelian} you are asked to prove
that the group of a ring is always abelian, as a consequence of the
extra structure and properties.} 
We follow the convention that the group data are denoted
by $0,\,+,\,-$ and the monoid data by $1,\,\cdot\,$.

\begin{definition}\label{def:abstractring}
An \emph{abstract ring} $\mathscr R$ consists of an abstract group 
$(R,0,+,-)$ and a monoid $(R,1,\cdot)$ with the
same underlying set $R$. Moreover, the following equations should hold,
for all $a,b,c : R$.
    \begin{enumerate}[ref=\ref{def:abstractring} (\alph*)]
    \item\label{ring:ldistr-law} $a \cdot (b + c) = a \cdot b + a \cdot c$ (the \emph{left distributive law})
    \item\label{ring:rdistr-law} $(a + b) \cdot  c = a \cdot c + b \cdot c$ (the \emph{right distributive law})
    \end{enumerate}
The abstract ring $\mathscr R$ is called \emph{commutative} if its
multiplication $\cdot$ is commutative, and \emph{non-trivial} if $0\neq 1$.
\end{definition}

The abstract group $(R,0,+,-)$ is called the \emph{(additive) group}
of $\mathscr R$, and the monoid $(R,1,\cdot)$ the 
\emph{(multiplicative) monoid} of $\mathscr R$.

\begin{xca}\label{xca:ring-group-abelian}
Let $\mathscr R$ be an abstract ring. Show that the additive group 
of $\mathscr R$ is abelian. Hint: elaborate $(a+1)\cdot(b+1)$.
\end{xca}

We will now elaborate an approach to rings that is more in line
with our set up. The obvious first step is to replace the abstract
additive group by a (concrete) group. The multiplicative monoid
poses some challenge, since monoids have no concrete counterpart in our set up.
However, for any abstract ring $\mathscr R$ and element $a:R$,
the left multiplication function $(a\cdot\blank)$ is an abstract homomorphism 
of the additive group $(R,0,+,-)$ of $\mathscr R$ to itself.\footnote{%
The same is true for the right multiplication function $(\blank\cdot a)$,
which will play a role later, in \cref{rem:concrabsring}.}
Even more so, the map $a\mapsto(a\cdot\blank)$ is an abstract homomorphism
from $(R,0,+,-)$ to the abstract group $\absHom_{\ptw}(R,R)$
of abstract homomorphisms from $(R,0,+,-)$ to itself.\footnote{%
The operations of $\absHom_{\ptw}(R,R)$ are pointwise,
induced by $(R,0,+,-)$. It is an abstract abelian group by
\cref{xca:abs-homgroup} and \cref{xca:abstract-group-of-maps}.}

Given that we have replaced $(R,0,+,-)$ by an abelian group $G:\Group$,
the plan is to deloop $\absHom_{\ptw}(\abstr(G),\abstr(G))$. 
Denoting the result of the delooping by $\Hom(G,G)$,\footnote{%
It will always be clear from the context whether this denotes
the \emph{set} of homomorphisms from $G$ to $G$, or the \emph{group}
with this set of homomorphisms as underlying set.}
we can then define the multiplication as a homomorphism
$\mu: \Hom(G,\Hom(G,G))$.

One way of delooping $\absHom_{\ptw}(\abstr(G),\abstr(G))$ would be
to use the inverse of $\abstr$ in \cref{lem:homomabstrconcr}.
This involves torsors and, though equivalent, is not close to $G$.  
A better option is to use \cref{thm:abelian-groups-weq-sc2types},
which we will do now.

Recall from \cref{thm:abelian-groups-weq-sc2types} the equivalence
$\BB$ from the type of abelian groups to the type of pointed
simply connected $2$-types. Let $G:\AbGroup$ be an abelian group.
Then the type $\BG\ptdto\BB G$ is a connected $1$-type\footnote{%
Note that the maps are pointed.}, pointed at the constant map that sends
$z:\BG$ to the point $(\BB G)_\pt\defeq (\BG_\div,\settrunc{\id_{\BG_\div}})$ 
of $\BB G$.\footnote{Itself pointed by reflexivity.}
Thus the type $\BG\ptdto\BB G$ classifies a group: 

\begin{definition}\label{def:groupHom}
Let $G:\AbGroup$ be an abelian group. Define the abelian group $\Hom(G,G)$
of homomorphisms from $G$ to $G$ as the group classified
by $\BHom(G,G) \defeq ((\BG\ptdto\BB G),(z\mapsto (\BB G)_\pt))$.
\end{definition}

The above definition of $\Hom(G,G)$ is indeed serving its purpose:

\begin{construction}\label{cons:groupHomOK}
Let $G:\AbGroup$ be an abelian group. Then we have an abstract isomorphism
from $\USym\Hom(G,G)$ to $\absHom_{\ptw}(\abstr(G),\abstr(G))$.
\end{construction}
\begin{implementation}{cons:groupHomOK}
TBD
\end{implementation}

\MB{{\color{red}CURSOR}}

There are two ways to compose them: $(a\cdot(\blank\cdot b))$
and $((a\cdot\blank)\cdot b)$. Equality of the latter two functions is
an elegant way of expressing associativity.
These observations lead to the following definition of a 
ring in our set up.

\begin{definition}\label{def:ring}
A \emph{ring} $R$ consists of a group also denoted $R$ together with
a symmetry $1_R : \USymR$ and two maps $\ell,r: \USymR\to\Hom(R,R)$
from the set of symmetries in $R$ to the set of homomorphisms from
$R$ to $R$.
Given $g:\USymR$, we write $\ell_g$ for the homomorphism $\ell(g)$ and 
$r_g$ for $r(g)$.
Moreover, the following equations should hold.
    \begin{enumerate}
    \item\label{ring:unit-laws} $\ell_{1_R} = \id_G = r_{1_R}$ (the \emph{multiplicative unit laws})
    \item\label{ring:lr-coherence-law} $(\USym\ell_g)(h) = (\USymr_h)(g)$, 
    for all $g,h : \USymR$ (the \emph{coherence law})\footnote{%
    These functions provide two ways to write the product $a\cdot b$,
see the coherence law in \cref{def:ring}\ref{ring:lr-coherence-law}.}
        \item\label{ring:associativity-law} $\ell\circ r= r\circ\ell$ (the \emph{associativity law})
    \end{enumerate}
The properties \ref{ring:unit-laws}-\ref{ring:associativity-law} 
are together denoted by $\RingProps(R,1_R,\ell,r)$.
The ring $R$ is called \emph{commutative} if $\ell=r$, 
and \emph{non-trivial} if $\refl{\sh_R}\neq 1_R$.
\end{definition}

\begin{definition}\label{def:typering}
The type of rings is defined as
\[
\typering\defeq\sum_{R:\typegroup}\sum_{1_R:\USymR}\sum_{\ell,r:\USymR\to\Hom(R,R)} \RingProps(R,1_R,\ell,r).
\]
The type $\typecommring$ of commutative rings is similar to the type
of rings but with $\RingProps(R,1_R,\ell,r) \times(\ell=r)$. 
\end{definition}

The coherence law \ref{ring:lr-coherence-law} allows us to abbreviate both 
$(\USym\ell_g)(h)$ and $(\USymr_h)(g)$ by $g\cdot h$. We will do this when
no confusion can occur. Then, $\ell=r$ 
amounts to $g\cdot h = h\cdot g$, for all $g,h : \USymG$,
as could be expected from the abstract case.

We proceed by giving the standard example of the integers as a ring
in the sense of \cref{def:ring}.
\begin{example}
Consider the group $\ZZ$ classified by the circle.
Using the same notation $\ZZ$ also for the ring, take $1_\ZZ \defeq\Sloop$
and $\ell: (\base\eqto\base)\to\Hom(\ZZ,\ZZ)$ defined as follows.
For every $g:\base\eqto\base$, let $\ell_g$ be the homomorphism
classified by the map $\B\ell_g(\base)\defeq\base$, 
$\B\ell_g(\Sloop)\defis g$, and pointed by reflexivity.\footnote{%
The reader may recognize the degree $m$
map from \cref{def:mfoldS1cover} as a special case.}
Take $r\defeq\ell$. Now the unit laws, the coherence law and
the associativity law can easily be verified. It follows that
$(\ZZ,1_\ZZ,\ell,!)$ is a non-trivial commutative ring.
\end{example}

\begin{definition}\label{def:ringhom}
Given rings $(R,1_R,\ell,r),(S,1_S,\ell',r') : \typering$,
a \emph{ring homomorphism} from $R$ to $S$ is a (group) homomorphism
$f:\Hom(R,S)$ that preserves the multiplicative unit and 
left and right multiplication. This means $\USymf(1_R) = 1_S$ and
$\USymf(\USym\ell_g(h)) = \USym\ell'_{\USymf(g)}(\USymf(h))$ for
all $g,h:\USymR$.\footnote{Also good for $r,r'$ by coherence.}
$\USymf(\USymr_g(h)) = \USymr'_{\USymf(g)}(\USymf(h))$.
\end{definition}

\begin{remark}\label{rem:ring}\MB{Experimental:}
Here we explore a definition of a ring that is even
``less abstract'' than \cref{def:ring}.

First, let $\agp G \defeq(S,0,+,-)$ be an abelian abstract group.
Then $\absHom(\agp G,\agp G)$ with pointwise operations,
denoted by $\absHom_{\ptw}(\agp G,\agp G)$, is also
an abelian abstract group, by  \cref{xca:abshomaddition} and
\cref{xca:abstract-group-of-maps}.

Now, let $\mathscr R$ be an abstract ring as in \cref{def:abstractring}.
Then the additive group of $\mathscr R$ is an abstract abelian group,
which we denote by $\agp R$. We observe that
$a\mapsto(a\cdot\blank)$ is an abstract homomorphism
from $\agp R$ to $\absHom_{\ptw}(\agp R,\agp R)$.\footnote{%
Left distributivity gives that $((a+b)\cdot r) = (a\cdot r)+(b\cdot r)$,
for all $a,b,r:R$.}

It is now possible to define a ring $R$ with the following data:
\begin{enumerate}
\item A group also denoted $R$
\item A homomorphism $1_R:\Hom(\ZZ,R)$
\item $\ell,r: \Hom(R,\concr(\absHom_{\ptw}(\abstr(R),\abstr(R))))$ \MB{???}
\item Properties: $\ell\circ 1_R = \id_\MB{???} = r\circ 1_R$, ...
\item Homomorphisms improve: $f:\Hom(R,S)$, $f\circ 1_R = 1_S$, ...\MB{???}
\end{enumerate}
\end{remark}




\MB{TBD define type of (abstract) rings, prove equivalence, define
ring homomorphisms, delooping etc. No interesting difficulties
expected before we come to fields.}





\begin{definition}
Given a commutative ring $R$, an element $e:R$ is \textbf{invertible} if there exists an element $a:R$ such that $e \cdot a = 1$ and $a \cdot e = 1$:
$$\mathrm{isInvertible}(e) := \left\Vert\sum_{a:R} (e \cdot a = 1) \times (a \cdot e = 1)\right\Vert$$
\end{definition}

\begin{theorem}
In any nontrivial commutative ring $R$, $0$ is always a non-invertible element. 
$$\mathrm{isNonTrivialCRing}(R) \to \neg \mathrm{isInvertible}(0)$$
\end{theorem}

\begin{proof}
Suppose that $0$ is invertible. Then there exists an element $a:R$ such that $a \cdot 0 = 1$. However, due to the absorption properties of $0$ and the fact that $R$ is a set, $a \cdot 0 = 0$. This implies that $0 = 1$, which contradicts the fact that $0 \neq 1$ in a nontrivial commutative ring. Thus, $0$ is a non-invertible element in any nontrivial commutative ring $R$. 
\end{proof}

\begin{definition}
A nontrivial commutative ring $R$ is a \textbf{field} if and only if the type of all non-invertible elements in $R$ is contractible:
$$\mathrm{isField}(R) := \mathrm{isNonTrivialCRing}(R) \times \mathrm{isContr}\left(\sum_{x:R} \neg \mathrm{isInvertible}(x)\right)$$ 
Equivalently, $R$ is a field if and only if every non-invertible element is equal to zero. 
\end{definition}

\begin{remark}
In other parts of the constructive mathematics literature, such as in Peter Johnstone's \textit{Rings, Fields, and Spectra}, this is called a "residue field". However, in this book we shall refrain from using the term "residue field" for our definition, since that contradicts the usage of "residue field" in other parts of mathematics, such as in algebraic geometry. 
\end{remark}

\begin{definition}
A field is \textbf{discrete} if every element is either invertible or equal to zero. 
$$\mathrm{isDiscreteField}(R) := \mathrm{isField}(R) \times \prod_{a:R} \Vert(a = 0) \amalg \mathrm{isInvertible}(a)\Vert$$ 
\end{definition}

\begin{definition}
A nontrivial commutative ring $R$ is a \textbf{local ring} if for every element $a:R$ and $b:R$, if the sum $a + b$ is invertible, then either $a$ is invertible or $b$ is invertible. 
$$\mathrm{isLocalRing}(R) := \mathrm{isNonTrivialCRing}(R) \times \prod_{a:R} \prod_{b:R} \mathrm{isInvertible}(a + b) \to \Vert\mathrm{isInvertible}(a) \amalg \mathrm{isInvertible}(b)\Vert$$ 
\end{definition}

\begin{definition}
A field $R$ is \textbf{Heyting} if it is also a local ring. 
$$\mathrm{isHeytingField}(R) := \mathrm{isField}(R) \times \mathrm{isLocalRing}(R)$$ 
\end{definition}

References used in this section: 
\begin{itemize}
\item Emmy Noether, \textit{Ideal Theory in Rings}, Mathematische Annalen 83 (1921)
\item Henri Lombardi, Claude Quitté, \textit{Commutative algebra: Constructive methods (Finite projective modules)}
\item Peter Johnstone, \textit{Rings, Fields, and Spectra}, Journal of Algebra 49 (1977) 238-260
\end{itemize}

\section{vector spaces}

\begin{definition}
Given a field $K$, a $K$-\textbf{vector space} is an abelian group $V$ with a bilinear function $(-)(-):K \times V \to V$ called \textbf{scalar multiplication} such that $1 v = v$ and for all elements $a:K$, $b:K$, and $v:V$, $(a \cdot b) v = a (b v)$. 
\end{definition}

\begin{definition}
A $K$-\textbf{linear map} between two $K$-vector spaces $V$ and $W$ is a group homomorphism $h:V \to W$ which also preserves scalar multiplication: for all elements $a:K$ and $v:V$, $f(a v) = a f(v)$. 
\end{definition}

\begin{definition}
Given a field $K$ and a set $S$, the \textbf{free $K$-vector space} on $S$ is the homotopy initial $K$-vector space $V$ with a function $i:S \to V$: for every other $K$-vector space $W$ with a function $j:S \to W$, the type of linear maps $h:V \to W$ such that for all elements $s:S$, $h(i(s)) = j(s)$ is contractible. 
\end{definition}

\begin{definition}
Given a field $K$ and a natural number $n$, an \textbf{$n$-dimensional $K$-vector space} is a free $K$-vector space on the finite type $\mathrm{Fin}(n)$. 
\end{definition}

\section{the general linear group as automorphism group}
\section{determinants\titledagger}
\section{examples: rationals, polynomials, adding a root, field extensions}
\section{ordered fields, real-closed fields, pythagorean fields, euclidean fields}
\section{complex fields, quadratically closed fields, algebraically closed fields}

%%% Local Variables:
%%% mode: latex
%%% TeX-master: "book"
%%% End:
